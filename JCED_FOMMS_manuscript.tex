%\documentclass[12pt]{article}
 %\documentclass[letterpaper,floatfix,citeautoscript,aip,jcp]{revtex4-1}
 %\documentclass[letterpaper,floatfix,citeautoscript,showkeys]{revtex4-1}
 %\documentclass[twocolumn,letterpaper,floatfix,citeautoscript,jcp]{revtex4-1}
 %\documentclass[twocolumn,letterpaper,floatfix,citeautoscript,aip,jcp]{revtex4-1}
 \documentclass[journal=jced,manuscript=article]{achemso}
 \setkeys{acs}{maxauthors=30,etalmode=truncate,articletitle=true}
 %%%%%%%%%%%%%%%%%%%%%%%%%%%%%%%%%%%%%%%%%%%%%%%%%%%%%%%%%%%%%%%%%%%%%
 %% Place any additional packages needed here.  Only include packages
 %% which are essential, to avoid problems later.
 %%%%%%%%%%%%%%%%%%%%%%%%%%%%%%%%%%%%%%%%%%%%%%%%%%%%%%%%%%%%%%%%%%%%%
 \usepackage{chemformula} % Formula subscripts using \ch{}
 \usepackage[T1]{fontenc} % Use modern font encodings
 
 %%%%%%%%%%%%%%%%%%%%%%%%%%%%%%%%%%%%%%%%%%%%%%%%%%%%%%%%%%%%%%%%%%%%%
 %% If issues arise when submitting your manuscript, you may want to
 %% un-comment the next line.  This provides information on the
 %% version of every file you have used.
 %%%%%%%%%%%%%%%%%%%%%%%%%%%%%%%%%%%%%%%%%%%%%%%%%%%%%%%%%%%%%%%%%%%%%
 %%\listfiles
 
 %%%%%%%%%%%%%%%%%%%%%%%%%%%%%%%%%%%%%%%%%%%%%%%%%%%%%%%%%%%%%%%%%%%%%
 %% Place any additional macros here.  Please use \newcommand* where
 %% possible, and avoid layout-changing macros (which are not used
 %% when typesetting).
 %%%%%%%%%%%%%%%%%%%%%%%%%%%%%%%%%%%%%%%%%%%%%%%%%%%%%%%%%%%%%%%%%%%%%
% \newcommand*\mycommand[1]{\texttt{\emph{#1}}}

\usepackage{fullpage}
\usepackage{amsfonts}
\usepackage{graphicx}
\usepackage{float}
\usepackage{amsmath}
\usepackage{chemfig}
\usepackage{indentfirst}
\usepackage{longtable}
\usepackage{array}
\usepackage{cellspace}
\usepackage{palatino}
%\usepackage{breqn}
\usepackage{amssymb}
\usepackage{verbatim}
\usepackage[hidelinks,colorlinks=false,citecolor=black,linkcolor=black]{hyperref}
\usepackage{siunitx}
\usepackage{xr}

\makeatletter
\newcommand*{\addFileDependency}[1]{% argument=file name and extension
	\typeout{(#1)}
	\@addtofilelist{#1}
	\IfFileExists{#1}{}{\typeout{No file #1.}}
}
\makeatother

\newcommand*{\myexternaldocument}[1]{%
	\externaldocument{#1}%
	\addFileDependency{#1.tex}%
	\addFileDependency{#1.aux}%
}

\myexternaldocument{JCED_FOMMS_supporting_information}

\SectionNumbersOn

% The figures are in a figures/ subdirectory.
\graphicspath{{figures/}}

%\bibliographystyle{apsrevlong}
%\bibliographystyle{apsrev}
\bibliographystyle{unsrt}

% italicized boldface for math (e.g. vectors)
\newcommand{\bfv}[1]{{\mbox{\boldmath{$#1$}}}}
% non-italicized boldface for math (e.g. matrices)
\newcommand{\bfm}[1]{{\bf #1}}          

%\newcommand{\bfm}[1]{{\mbox{\boldmath{$#1$}}}}
%\newcommand{\bfm}[1]{{\bf #1}}
\newcommand{\expect}[1]{\left \langle #1 \right \rangle} % <.> for denoting expectations over realizations of an experiment or thermal averages

\newcommand{\var}[1]{{\mathrm var}{(#1)}}
\newcommand{\x}{\bfv{x}}
\newcommand{\y}{\bfv{y}}
\newcommand{\f}{\bfv{f}}

\newcommand{\hatf}{\hat{f}}

\newcommand{\bTheta}{\bfm{\Theta}}
\newcommand{\btheta}{\bfm{\theta}}
\newcommand{\bhatf}{\bfm{\hat{f}}}
\newcommand{\Cov}[1] {\mathrm{cov}\left( #1 \right)}
\newcommand{\T}{\mathrm{T}}                                % T used in matrix transpose

\newcommand\blfootnote[1]{%
	\begingroup
	\renewcommand\thefootnote{}\footnote{#1}%
	\addtocounter{footnote}{-1}%
	\endgroup
}

\author{Richard A. Messerly}
\email{richard.messerly@nist.gov}
\affiliation{Thermodynamics Research Center, National Institute of Standards and Technology, Boulder, Colorado, 80305, United States}

\author{Mohammad Soroush Barhaghi}
\affiliation{Department of Chemical Engineering and Materials Science, Wayne State University, Detroit, Michigan 48202, United States}

\author{Jeffrey J. Potoff}
\affiliation{Department of Chemical Engineering and Materials Science, Wayne State University, Detroit, Michigan 48202, United States}

\author{Michael R. Shirts}
\affiliation{Department of Chemical and Biological Engineering, University of Colorado Boulder, Colorado, 80309, United States}

%%%%%%%%%%%%%%%%%%%%%%%%%%%%%%%%%%%%%%%%%%%%%%%%%%%%%%%%%%%%%%%%%%%%%
%% The document title should be given as usual. Some journals require
%% a running title from the author: this should be supplied as an
%% optional argument to \title.
%%%%%%%%%%%%%%%%%%%%%%%%%%%%%%%%%%%%%%%%%%%%%%%%%%%%%%%%%%%%%%%%%%%%%
%\title{Multistate Bennett Acceptance Ratio replaces histogram reweighting for vapor-liquid coexistence calculations}
%\title{Multistate Bennett Acceptance Ratio to enable rapid force field parameterization}
%\title{Multistate Bennett Acceptance Ratio as a substitute for histogram reweighting when optimizing non-bonded parameters}
%\title{Multistate reweighting provides a better alternative to histogram reweighting for coexistance calculations}
%\title{Multistate histogram-free reweighting for vapor-liquid coexistence calculations of non-simulated force field parameters}
%\title{Estimating vapor-liquid coexistence properties with histogram-free reweighting}
%\title{Histogram-free reweighting for vapor-liquid coexistence calculations of multiple force fields}
%\title{Histogram-free reweighting to estimate vapor-liquid coexistence properties of non-simulated force fields}
%\title{Histogram-free reweighting enables estimation of vapor-liquid coexistence properties for non-simulated force fields}
%\title{A histogram-free reweighting approach to estimate vapor-liquid coexistence properties for non-simulated force fields}
%\title{A histogram-free reweighting approach for estimating vapor-liquid coexistence properties of non-simulated force fields}
%\title{Estimating vapor-liquid coexistence properties of non-simulated force fields with histogram-free reweighting}
%\title{Histogram-free reweighting with Grand canonical Monte Carlo: A post-simulation approach to optimize non-bonded potentials for phase equilibria}
\title{Histogram-free reweighting with grand canonical Monte Carlo: Post-simulation optimization of non-bonded potentials for phase equilibria}
%MRS2: I'm not sure how the (\theta) notation works for force fields - why not without the parenthesis?
%RAM: The parenthesis should only appear the first time we define the \theta notation
%MRS2: I would focus on repeating/reinforcing the messages you most want to communicate - they may be getting a bit lost in the details.
%RAM: True, we need to repeat this several times. I will work on this.
%%%%%%%%%%%%%%%%%%%%%%%%%%%%%%%%%%%%%%%%%%%%%%%%%%%%%%%%%%%%%%%%%%%%%
%% Some journals require a list of abbreviations or keywords to be
%% supplied. These should be set up here, and will be printed after
%% the title and author information, if needed.
%%%%%%%%%%%%%%%%%%%%%%%%%%%%%%%%%%%%%%%%%%%%%%%%%%%%%%%%%%%%%%%%%%%%%
%\abbreviations{IR,NMR,UV}
\keywords{Multistate Bennett Acceptance Ratio (MBAR), Grand canonical Monte Carlo (GCMC), Basis functions, Mie $\lambda$-6, Lennard-Jones}

%%%%%%%%%%%%%%%%%%%%%%%%%%%%%%%%%%%%%%%%%%%%%%%%%%%%%%%%%%%%%%%%%%%%%
%% The manuscript does not need to include \maketitle, which is
%% executed automatically.
%%%%%%%%%%%%%%%%%%%%%%%%%%%%%%%%%%%%%%%%%%%%%%%%%%%%%%%%%%%%%%%%%%%%%
\begin{document}
	
	%%%%%%%%%%%%%%%%%%%%%%%%%%%%%%%%%%%%%%%%%%%%%%%%%%%%%%%%%%%%%%%%%%%%%
	%% The "tocentry" environment can be used to create an entry for the
	%% graphical table of contents. It is given here as some journals
	%% require that it is printed as part of the abstract page. It will
	%% be automatically moved as appropriate.
	%%%%%%%%%%%%%%%%%%%%%%%%%%%%%%%%%%%%%%%%%%%%%%%%%%%%%%%%%%%%%%%%%%%%%
%	\begin{tocentry}
%	\end{tocentry}

%\blfootnote{Contribution of NIST, an agency of the United States government; not subject to copyright in the United States.}
\newpage
\section*{Abstract}
%
%Reliable prediction of vapor-liquid phase equilibria with molecular simulation relies on efficient computational methods and well-parameterized force fields. Histogram reweighting (HR) is a standard approach for converting grand canonical Monte Carlo (GCMC) simulation output into vapor-liquid coexistence properties (saturated liquid density, $\rho_{\rm liq}^{\rm sat}$, saturated vapor density, $\rho_{\rm vap}^{\rm sat}$, saturated vapor pressures, $P_{\rm vap}^{\rm sat}$, and enthalpy of vaporization, $\Delta H_{\rm v}$). Due to the abundance of experimental vapor-liquid coexistence data and the sensitivity of such properties to both short- and long-range non-bonded interactions, numerous force fields have been parameterized using $\rho_{\rm liq}^{\rm sat}$, $\rho_{\rm vap}^{\rm sat}$, $P_{\rm vap}^{\rm sat}$, and/or $\Delta H_{\rm v}$. Unfortunately, computing $\rho_{\rm liq}^{\rm sat}$, $\rho_{\rm vap}^{\rm sat}$, $P_{\rm vap}^{\rm sat}$, and $\Delta H_{\rm v}$ for each proposed non-bonded parameter set (e.g., the Lennard-Jones parameters $\epsilon$ and $\sigma$) requires a large number of molecular simulations. 
%%MRS: why not charges as well as nonbonded parameters?  Also, could vary any of the epsilon's and sigmas, not just a single one.
%%RAM: I don't think in the abstract I necessarily want to enumerate all parameters. This was just a single example that does not preclude charges and multiple eps/sig
%
%We demonstrate that histogram-free reweighting can alleviate the computational burden of force field parameterization. Specifically, the Multistate Bennett Acceptance Ratio (MBAR) reweights configurations sampled from GCMC simulations performed with an initial reference parameter set $(\theta_{\rm ref})$. While MBAR is similar to the traditional HR method for computing $\rho_{\rm liq}^{\rm sat}$, $\rho_{\rm vap}^{\rm sat}$, $P_{\rm vap}^{\rm sat}$, and $\Delta H_{\rm v}$, the primary advantage of MBAR is that this approach can estimate coexistence properties for a different ``rerun'' parameter set $(\theta_{\rm rr} \neq \theta_{\rm ref})$ without simulating $\theta_{\rm rr}$ directly. MBAR thus greatly reduces the number of GCMC simulations that are required for parameterizing the non-bonded potential.

%Specifically, GCMC simulations are performed with an initial reference parameter set $(\theta_{\rm ref})$. The Multistate Bennett Acceptance Ratio (MBAR) reweights the configurations sampled from the reference simulations to calculate $\rho_{\rm liq}^{\rm sat}$, $\rho_{\rm vap}^{\rm sat}$, $P_{\rm vap}^{\rm sat}$, and $\Delta H_{\rm v}$. While MBAR is similar to the traditional HR method, the primary advantage of MBAR is that this approach can estimate coexistence properties for a different ``rerun'' parameter set $(\theta_{\rm rr} \neq \theta_{\rm ref})$ without simulating $\theta_{\rm rr}$ directly. MBAR thus greatly reduces the number of GCMC simulations that are required for parameterizing the non-bonded potential.

%The internal energy for a different ``rerun'' parameter set $(\theta_{\rm rr} \neq \theta_{\rm ref})$ is computed for each snapshot from the reference simulations. The Multistate Bennett Acceptance Ratio (MBAR) reweights these reference snapshots to estimate $\rho_{\rm liq}^{\rm sat}$, $\rho_{\rm vap}^{\rm sat}$, $P_{\rm vap}^{\rm sat}$, and $\Delta H_{\rm v}$ for $\theta_{\rm rr}$, without simulating $\theta_{\rm rr}$ directly.

%Specifically, we utilize the Multistate Bennett Acceptance Ratio (MBAR) to estimate coexistence properties for non-bonded parameters without simulating each parameter set directly. MBAR thus greatly reduces the number of GCMC simulations that are required for parameterizing the non-bonded potential.

Histogram reweighting (HR) is a standard approach for converting grand canonical Monte Carlo (GCMC) simulation output into vapor-liquid coexistence properties (saturated liquid density, $\rho_{\rm liq}^{\rm sat}$, saturated vapor density, $\rho_{\rm vap}^{\rm sat}$, saturated vapor pressures, $P_{\rm vap}^{\rm sat}$, and enthalpy of vaporization, $\Delta H_{\rm v}$). We demonstrate that a histogram-free reweighting approach, namely, the Multistate Bennett Acceptance Ratio (MBAR), is similar to the traditional HR method for computing $\rho_{\rm liq}^{\rm sat}$, $\rho_{\rm vap}^{\rm sat}$, $P_{\rm vap}^{\rm sat}$, and $\Delta H_{\rm v}$. The primary advantage of MBAR is the ability to predict phase equilibria properties for an arbitrary force field parameter set that has not been simulated directly. Thus, MBAR can greatly reduce the number of GCMC simulations that are required to parameterize a force field with phase equilibria data.

Four different applications of GCMC-MBAR are presented in this study. First, we validate that GCMC-MBAR and GCMC-HR yield statistically indistinguishable results for $\rho_{\rm liq}^{\rm sat}$, $\rho_{\rm vap}^{\rm sat}$, $P_{\rm vap}^{\rm sat}$, and $\Delta H_{\rm v}$ in a limiting test case. Second, we utilize GCMC-MBAR to optimize an individualized (compound-specific) parameter ($\psi$) for 8 branched alkanes and 11 alkynes using the Mie Potentials for Phase Equilibria (MiPPE) force field. Third, we predict $\rho_{\rm liq}^{\rm sat}$, $\rho_{\rm vap}^{\rm sat}$, $P_{\rm vap}^{\rm sat}$, and $\Delta H_{\rm v}$ for force field $j$ by simulating force field $i$, where $i$ and $j$ are common force fields from the literature. In addition, we provide guidelines for determining the reliability of GCMC-MBAR predicted values. Fourth, we develop and apply a post-simulation optimization scheme to obtain new MiPPE non-bonded parameters for cyclohexane ($\epsilon_{\rm CH_2}$, $\sigma_{\rm CH_2}$, and $\lambda_{\rm CH_2}$).

%We demonstrate that histogram-free reweighting can help reduce the number of GCMC simulations that are required to parameterize the force field non-bonded potential. Specifically, the Multistate Bennett Acceptance Ratio (MBAR) reweights configurations sampled from GCMC simulations performed with an initial reference parameter set $(\theta_{\rm ref})$. While MBAR is similar to the traditional HR method for computing $\rho_{\rm liq}^{\rm sat}$, $\rho_{\rm vap}^{\rm sat}$, $P_{\rm vap}^{\rm sat}$, and $\Delta H_{\rm v}$, the primary advantage of MBAR is that this approach can estimate coexistence properties for a different ``rerun'' parameter set $(\theta_{\rm rr} \neq \theta_{\rm ref})$ without simulating $\theta_{\rm rr}$ directly. MBAR thus greatly reduces the number of GCMC simulations that are required for parameterizing the non-bonded potential.

%Four different applications of GCMC-MBAR are presented in this study. First, we validate that GCMC-MBAR and GCMC-HR yield statistically indistinguishable results for $\rho_{\rm liq}^{\rm sat}$, $\rho_{\rm vap}^{\rm sat}$, $P_{\rm vap}^{\rm sat}$, and $\Delta H_{\rm v}$ when $\theta_{\rm rr} = \theta_{\rm ref}$. Second, we utilize GCMC-MBAR to optimize an individualized (compound-specific) parameter ($\psi$) for 8 branched alkanes and 11 alkynes using the Mie Potentials for Phase Equilibria (MiPPE) force field. Third, we predict $\rho_{\rm liq}^{\rm sat}$, $\rho_{\rm vap}^{\rm sat}$, $P_{\rm vap}^{\rm sat}$, and $\Delta H_{\rm v}$ for force field $j$ by simulating force field $i$, where $i$ and $j$ are common force fields from the literature. In addition, we provide guidelines for determining the reliability of GCMC-MBAR predicted values when $\theta_{\rm rr} \not\approx \theta_{\rm ref}$. Fourth, we develop an optimization scheme and report new MiPPE non-bonded parameters for cyclohexane ($\epsilon_{\rm CH_2}$, $\sigma_{\rm CH_2}$, and $\lambda_{\rm CH_2}$).   

%Four different applications of GCMC-MBAR are presented in this study. First, we validate that GCMC-MBAR and GCMC-HR yield statistically indistinguishable results for $\rho_{\rm liq}^{\rm sat}$, $\rho_{\rm vap}^{\rm sat}$, $P_{\rm vap}^{\rm sat}$, and $\Delta H_{\rm v}$ when $\theta_{\rm rr} = \theta_{\rm ref}$. This validation is based on previous simulation data from the literature for branched alkanes and alkynes with the Mie Potentials for Phase Equilibria (MiPPE) force field.
%
%Second, we utilize GCMC-MBAR to optimize an individualized (compound-specific) parameter, $\psi$, that scales the non-bonded energy parameter $(\epsilon)$. GCMC simulation data for the initial parameter set (MiPPE) are reweighted with MBAR to estimate $\rho_{\rm liq}^{\rm sat}$, $\rho_{\rm vap}^{\rm sat}$, $P_{\rm vap}^{\rm sat}$, and $\Delta H_{\rm v}$ for a range of $\psi$ values.
%
%Third, we predict coexistence properties of force field $j$ by simulating force field $i$, where $i$ and $j$ are common force fields from the literature. Specifically, we perform simulations with MiPPE and the Transferable Potentials for Phase Equilibria (TraPPE) force field. We estimate $\rho_{\rm liq}^{\rm sat}$, $\rho_{\rm vap}^{\rm sat}$, $P_{\rm vap}^{\rm sat}$, and $\Delta H_{\rm v}$ for MiPPE by reweighting the TraPPE simulation data and vice versa. 
%
%Fourth, we optimize new Mie $\lambda$-6 (generalized Lennard-Jones) parameters for cyclohexane ($\epsilon_{\rm CH_2}$, $\sigma_{\rm CH_2}$, and $\lambda_{\rm CH_2}$). This post-simulation optimization is achieved in two stages. In the first stage, we perform GCMC simulations with the TraPPE force field (where $\lambda_{\rm CH_2} = 12$) and apply MBAR to estimate coexistence properties over a wide range of $\epsilon_{\rm CH_2}$ and $\sigma_{\rm CH_2}$ parameter sets for $\lambda_{\rm CH_2} = $ 12, 14, 16, 18, and 20. In the second stage, additional simulations are performed with the optimal parameter sets for each value of $\lambda_{\rm CH_2}$ and the overall optimal parameter set is determined by again estimating $\rho_{\rm liq}^{\rm sat}$, $\rho_{\rm vap}^{\rm sat}$, $P_{\rm vap}^{\rm sat}$, and $\Delta H_{\rm v}$ with MBAR.   

% potentials for cyclohexane. where only the Lennard-Jones parameters are simulated directly. % without performing any additional simulations.

\section{Introduction} \label{sec: Introduction}

A key use of molecular simulation is the ability to accurately and efficiently estimate vapor-liquid phase equilibria properties, i.e., saturated liquid density $(\rho_{\rm liq}^{\rm sat})$, saturated vapor density $(\rho_{\rm vap}^{\rm sat})$, saturated vapor pressures $(P_{\rm vap}^{\rm sat})$, and enthalpy of vaporization $(\Delta H_{\rm v})$. The accuracy of coexistence estimates depends on the underlying molecular model (a.k.a., force field, potential model, or Hamiltonian) while the computational efficiency depends primarily on the simulation methods, software, and hardware.

Several simulation approaches exist for computing vapor-liquid coexistence properties \cite{Pana2000}. These include Gibbs Ensemble Monte Carlo (GEMC) \cite{Pana2000,Stubbs2004}, grand canonical Monte Carlo coupled with histogram reweighting (GCMC-HR) \cite{Pana2000,Potoff1999,Stubbs2004}, two-phase molecular dynamics (2$\phi$MD) \cite{Fern2007}, and isothermal-isochoric integration (ITIC) \cite{Mostafa2018}. The improved efficiency of these methods has greatly enabled the development of accurate force fields \cite{TraPPE,TAMie,Mie,AUA4,Mess4}. However, parameterization of non-bonded interactions with vapor-liquid coexistence calculations over a wide range of temperatures remains an arduous and time-consuming task. For example, recent studies have implemented an exhaustive grid-based search optimization by performing GCMC-HR simulations with hundreds of non-bonded parameter sets. \cite{Mick_Mie,Potoff_branched,Barhaghi2017}

%Due to the abundance of experimental vapor-liquid coexistence data and the sensitivity of such properties to both short- and long-range non-bonded interactions, numerous force fields \cite{TraPPE,Mie,TAMie,AUA4,Mess4} have been parameterized using $\rho_{\rm liq}^{\rm sat}$, $\rho_{\rm vap}^{\rm sat}$, $P_{\rm vap}^{\rm sat}$, and/or $\Delta H_{\rm v}$. Although the development of accurate force fields has been greatly enabled by the improved efficiency of simulation methods, parameterization of non-bonded interactions with vapor-liquid coexistence calculations remains an arduous and time-consuming task \cite{TraPPE,TAMie,Mie}.

% Advantages and disadvantages exist for each method. For example, GEMC and GCMC require insertion moves that are computationally inefficient for complex molecular structures with high density liquid phases. Several advanced simulation techniques are available to overcome this challenge \cite{ConBias}, which has enabled GEMC and GCMC-HR to be the primary methods of choice for vapor-liquid coexistence calculations.

%Some clear advantages and disadvantages exist for GCMC-HR compared with GEMC. For example, one advantage of GCMC-HR is the higher precision \cite{GEMC_GCMC}. Furthermore, coexistence properties can be computed at temperatures that are not simulated directly. However, GEMC is arguably more straightforward in that simulations are performed only at the desired saturation temperatures $(T^{\rm sat})$. By contrast, GCMC-HR requires a series of GCMC simulations for a single $T^{\rm sat}$. This set includes a near-critical simulation that ``bridges'' the vapor and liquid phases. Obtaining the appropriate chemical potential $(\mu)$ for this bridge simulation is a cumbersome and, typically, iterative process (although more advanced methods exist to obtain a good initial estimate for $\mu$ \cite{Hemmen2015}).

%Another disadvantage of GCMC-HR compared to GEMC is that GCMC-HR requires more post-processing (i.e., histogram reweighting), while simple block averaging is typically sufficient for GEMC. 

%Histogram reweighting (and more generally, configuration reweighting) is an important tool in many fields of molecular simulation. In fact, it has long since been known that it is possible to estimate properties for state $j$ by reweighting configurations that were sampled with state $i$. \cite{McDonald1967,Card1970,Wood1968,Pana2000} For example, umbrella sampling simulations are often processed using the weighted histogram analysis method (WHAM) to compute free energy differences between states. A popular alternative to WHAM is the Multistate Bennett Acceptance Ratio (MBAR) \cite{chodera:jctc:2007,shirts-chodera:jcp:2008:mbar}, which is readily available in the \textit{pymbar} package.

%MRS2: sounds a bit off.  We are presenting a general approach, and what the particular objective in this study is is somewhat irrelevant. 
%Our primary objective is to reduce the 
%We can significantly
%RAM: OK, so it is not our objective but I think it is what motivated the research
The primary motivation for this work is to reduce the computational cost of optimizing non-bonded parameters with vapor-liquid phase equilibria properties. This is achieved by substituting histogram reweighting with the Multistate Bennett Acceptance Ratio (MBAR), \cite{chodera:jctc:2007,shirts-chodera:jcp:2008:mbar} a histogram-free reweighting schema. The proposed GCMC-MBAR method for calculating phase equilibria is identical to the traditional GCMC-HR approach except that MBAR reweights configurations rather than histograms. The benefit of this simple modification is that GCMC-MBAR can estimate phase equilibria for non-bonded parameter sets that have not been simulated directly. While storing configuration files is significantly more memory intensive than storing histogram files (and scales as the number of molecules), this additional storage load can be alleviated greatly by utilizing basis functions (see Section \ref{sec: Basis functions}).

% we perform GCMC simulations and reweight the configurations with MBAR to predict vapor-liquid coexistence properties for non-bonded parameter sets that are not simulated directly.  substitute traditional HR with MBAR to obtain a GCMC-MBAR method for computing vapor-liquid coexistence properties. Because MBAR reweights configurations rather than histograms, GCMC-MBAR can estimate coexistence properties for non-bonded parameter sets that have not been simulated directly. 
%
%MBAR is a histogram-free reweighting approach, that we use in place of the standard HR  which is substituted for HR in the standard GCMC-HR approach.
%
%In this study, we substitute HR with MBAR for the GCMC-HR approach of computing vapor-liquid coexistence properties. Section \ref{sec: GCMC-HR and GCMC-MBAR} demonstrates that MBAR and HR are mathematically equivalent (in the limit of zero bin width) while Section \ref{sec: Results} shows that they are also numerically equivalent (to within statistical uncertainties). Boulougouris et al. demonstrate how to combine HR with GEMC (GEMC-HR) to estimate saturation properties at non-simulated temperatures \cite{Boulougouris2010}, therefore, MBAR could alternatively be applied to GEMC simulations.

%this study presents how MBAR can be used with GCMC simulations
%
%  HR is more commonly applied to GCMC than GEMC simulations (for an example of GEMC-HR see Ref. \citenum{Boulougouris2010}) is not unique to GCMC simulations (see Ref. \citenum{Boulougouris2010} where HR is applied to GEMC-HR) As histogram reweighting is already an essential tool for GCMC-HR, we substitute HR with MBAR  in the GCMC-HR approach of computing vapor-liquid coexistence properties. Section \ref{sec: GCMC-HR and GCMC-MBAR} demonstrates that MBAR and HR are mathematically equivalent (in the limit of zero bin width) while Section \ref{sec: Results} shows that they are also numerically equivalent (to within statistical uncertainties). Boulougouris et al. demonstrate how to combine HR with GEMC (GEMC-HR) to estimate saturation properties at non-simulated temperatures \cite{Boulougouris2010}, therefore, MBAR could alternatively be applied to GEMC simulations.

%Similar to the HR MBAR could alternatively be applied to GEMC simulations in an approach similar to the histogram reweighting GEMC  that of Boulougouris et al. demonstrate how to combine HR with GEMC (GEMC-HR) to estimate saturation properties at non-simulated temperatures \cite{Boulougouris2010}, therefore, MBAR could alternatively be applied to GEMC simulations.

%Histogram reweighting (and more generally, configuration reweighting) is an important tool in many fields of molecular simulation. In fact, it has long since been known that it is possible to estimate properties for state $j$ by reweighting configurations that were sampled with state $i$. \cite{McDonald1967,Card1970,Wood1968,Pana2000} For example, umbrella sampling simulations are often processed using the weighted histogram analysis method (WHAM) to compute free energy differences between states. A popular alternative to WHAM is the Multistate Bennett Acceptance Ratio (MBAR) \cite{chodera:jctc:2007,shirts-chodera:jcp:2008:mbar}, which is readily available in the \textit{pymbar} package.

%MRS: describing published papers below, should use past tense. 
%RAM: I like to think of the papers in present tense. If the readers goes to Messerly et al. the manuscript will still demonstrate ... it has not stopped demonstrating after being published
%MRS: Do you need ``reference'' in quotation marks?  Seems like it doesn't need it.
%RAM: I think the first time it is helpful
%MRS2: really only meaningful term in context of GROMACS
%(``rerun'') 
%RAM: We have actually adopted this terminology in GOMC as well. It is important for the sub-script "rr" and so I prefer to keep it. 
%MRS3: but it might not be clear to other people - it's a term whose meaning is nonobvious. How about:
%RAM: OK, I rewrote this so it is more clear what rerun means
In related studies, Messerly et al.~demonstrate how to combine MBAR with ITIC (MBAR-ITIC) to optimize generalized Lennard-Jones (Mie $\lambda$-6) potentials \cite{Postdoc_1,Postdoc_2}. For MBAR-ITIC, a series of simulations are performed with a constant number of molecules, constant volume, and constant temperature $(NVT)$ along a supercritical isotherm and liquid density isochore(s) with a ``reference'' force field(s) $(\theta_{\rm ref})$. Subsequently, the energies and forces are recomputed for the configurations sampled with $\theta_{\rm ref}$ using a different ``rerun'' force field $(\theta_{\rm rr})$. MBAR reweights the reference configurations to estimate the internal energy $(U)$ and pressure $(P)$ at each $(T$, $\rho)$ state point for $\theta_{\rm rr}$, without directly running simulations with $\theta_{\rm rr}$. ITIC then converts the MBAR $U$ and $P$ estimates into vapor-liquid phase equilibria properties for $\theta_{\rm rr}$~\cite{Mostafa_Diss,Mostafa2018}.

% This is followed by a ``rerun'' step where the energies and forces are recomputed for the configurations sampled with $\theta_{\rm ref}$ using a different force field $(\theta_{\rm rr})$.
% The configurations sampled with $\theta_{\rm ref}$ are then ``rerun'' by recomputing the energies and forces with a different force field $(\theta_{\rm rr})$.
%The energies and forces for the configurations sampled with $\theta_{\rm ref}$ are then recomputed with a ``rerun'' force field $(\theta_{\rm rr})$, rather than running a direct simulation with $\theta_{\rm rr}$.
%Instead, the energies and forces are simply recomputed using the ``rerun'' force field $(\theta_{\rm rr})$, rather than running an actual molecular simulation. , i.e. where the generated configurations are simply reevaluated with the new force field, rather than run with them.

% with only single stored snapshots from simulations with $(\theta_{\rm ref})$ evaluated using simulations with $(\theta_{\rm rr})$.
%In this study, we substitute HR with MBAR for the GCMC-HR approach of computing vapor-liquid coexistence properties. Section \ref{sec: GCMC-HR and GCMC-MBAR} demonstrates that MBAR and HR are mathematically equivalent (in the limit of zero bin width) while Section \ref{sec: Results} shows that they are also numerically equivalent (to within statistical uncertainties). Boulougouris et al. demonstrate how to combine HR with GEMC (GEMC-HR) to estimate saturation properties at non-simulated temperatures \cite{Boulougouris2010}, therefore, MBAR could alternatively be applied to GEMC simulations.

%and example scripts are included as Supporting Information to promote future implementation.  

%In this study, we utilize an alternative to histogram reweighting, namely, the Multistate Bennett Acceptance Ratio (MBAR) . MBAR is readily available in the \textit{pymbar} package and example scripts are included as Supporting Information to promote future implementation. Section \ref{sec: GCMC-HR and GCMC-MBAR} demonstrates that MBAR and HR are mathematically equivalent (in the limit of zero bin width) while Section \ref{sec: Results} shows that they are also numerically equivalent (to within statistical uncertainties). Note that Boulougouris et al. combine HR with GEMC (GEMC-HR) to estimate saturation properties at non-simulated temperatures \cite{Boulougouris2010}. Therefore, although we apply MBAR to GCMC simulations, the approach can also be applied to GEMC simulations.

%In this study, we utilize the Multistate Bennett Acceptance Ratio (MBAR) grand canonical Monte Carlo (GCMC-MBAR) as a substitute for GCMC-HR. We demonstrate that MBAR and HR are mathematically equivalent (in the limit of zero bin width) as well as practically equivalent (to within statistical uncertainties). Note that Boulougouris et al. demonstrate how HR can be applied to GEMC output (GEMC-HR) to estimate saturation properties at non-simulated temperatures \cite{Boulougouris2010}. Therefore, although we apply MBAR to GCMC simulations, the approach can also be applied to GEMC simulations.
%
%
%
%Although GEMC appears to be slightly more popular amongst simulation practitioners, this study utilizes the GCMC approach. GCMC-HR has been shown to

%In this study we demonstrate how the Multistate Bennett Acceptance Ratio is mathematically equivalent to histogram reweighting 
%%% Old version
%Although GCMC-HR has a higher precision than GEMC, GEMC remains a more popular method amongst simulation practitioners. There are at least two potential reasons why GEMC has grown in popularity relative to GCMC-HR. First, GEMC is more straightforward in that, GEMC simulations are performed directly at the desired saturation temperature $(T^{\rm sat})$. By contrast, GCMC-HR requires a series of GCMC simulations for a single $T^{\rm sat}$ (although this is also an advantage of GCMC-HR as estimates can be obtained at any $T^{\rm sat}$ value without additional simulations). This set includes a near-critical simulation that ``bridges'' the vapor and liquid phases. Obtaining an initial guess for the chemical potential $(\mu)$ of this bridge simulation is a cumbersome and, typically, iterative process, although more advanced methods do exist to obtain a good initial estimate for $\mu$ (e.g., Hemmen et al. \cite{Hemmen2015}). 
%
%A likely second reason for increased popularity of GEMC is that GCMC-HR requires a great deal of post-processing (i.e., histogram reweighting), while simple block averaging is typically sufficient for GEMC. In this study, we introduce an alternative to histogram reweighting, namely, the Multistate Bennett Acceptance Ratio (MBAR) \cite{chodera:jctc:2007,shirts-chodera:jcp:2008:mbar}. MBAR is readily available in the \textit{pymbar} package and example scripts are included as Supporting Information to promote future implementation.
%
%Histogram reweighting (and more generally, configuration reweighting) is an important tool in many fields of molecular simulation. In fact, it has long since been known that it is possible to estimate properties for state ``j'' by reweighting configurations that were sampled with state ``i.'' \cite{McDonald1967,Card1970,Wood1968,Pana2000} For example, umbrella sampling simulations are often processed using the weighted histogram analysis method (WHAM) to compute free energy differences between states. In addition, Boulougouris et al. demonstrated how histogram reweighting can be applied to GEMC output to estimate saturation properties at non-simulated temperatures (analogous to GCMC-HR) \cite{Boulougouris2010}.
%
%Although the development of accurate force fields has been greatly enabled by the efficiency of the aforementioned simulation methods (e.g., GEMC, GCMC-HR), parameterization of non-bonded interactions with vapor-liquid coexistence calculations remains an arduous and time-consuming task \cite{TraPPE,TAMie,Mie}. GCMC-MBAR not overly serves as a substitute for MBAR-HR, but MBAR can also be used to estimate properties for non-simulated parameter sets. For example, recently, Messerly et al. demonstrated how MBAR coupled with ITIC (MBAR-ITIC) enables rapid force field parameterization by estimating coexistence properties for non-simulated parameter sets \cite{Postdoc_1,Postdoc_2}. 

%
%Although histogram reweighting requires additional analysis steps, the benefits of histogram reweighting are clear. For example

%Although histogram reweighting approaches are common in some fields of molecular simulation, e.g., WHAM is commonly used for computing free energies, most open-source Monte Carlo codes do not include a HR tool and, thus, in-house post-processing codes abound. In this study, we introduce an alternative to histogram reweighting, namely, the Multistate Bennett Acceptance Ratio (MBAR). MBAR is readily available in the \textit{pymbar} package and example scripts are included as Supporting Information to promote future implementation.

%, with a near-critical simulation that ``bridges'' the vapor and liquid phases.

%For example, the exponential-6 model of Errington and Panagiotoupoulos
%The method outlined in this study is similar in spirit to the ``Hamiltonian scaling'' (HS) approach utilized with GEMC (BLANK) and GCMC-HR (BLANK).

%A closely related method to histogram reweighting, and one that is similar in spirit to the method outlined in the present study, is ``Hamiltonian scaling'' (HS). Despite Hamiltonian scaling grand canonical Monte Carlo (HS-GCMC) proving to be a powerful tool to obtain coexistence curves for multiple force fields from a single set of simulations, it has yet to gain widespread popularity. This is likely due to the added complexity of the algorithm, where the prescribed $\mu$ and $T$ change during the coarse of the GCMC simulation, depending on which Hamiltonian (force field) is being sampled. Furthermore, the post-processing requires a slightly more complicated form of histogram reweighting. Also, HS requires a decision be made \textit{a priori} regarding which Hamiltonians are to be tested. By contrast, MBAR does not require any modification of the simulation procedure, the post-processing is essentially unchanged, and the Hamiltonians need not be selected prior to the simulations.
 
%Substituting the standard HR approach with MBAR is not the primary purpose of this study. Rather, we demonstrate how GCMC-MBAR can also estimate coexistence properties for non-simulated parameter sets, which can greatly accelerate force field parameterization. In a similar study, Messerly et al. demonstrate how to combine MBAR with ITIC (MBAR-ITIC) to optimize Mie $\lambda$-6 (generalized Lennard-Jones) potentials \cite{Postdoc_1,Postdoc_2}. For MBAR-ITIC, a series of $NVT$ simulations along an isotherm and isochores are performed with a ``reference'' force field $(\theta_{\rm ref})$. MBAR computes the internal energy $(U)$ and pressure $(P)$ (or compressibility factor, $Z$) for each $T-\rho$ state point with a non-simulated (``rerun'') force field $(\theta_{\rm rr})$. ITIC then converts the $U$ and $P$ values into vapor-liquid coexistence properties \cite{Mostafa_Diss,Mostafa2018}.

%, where $U$ and $P$ are estimated by performing $NVT$ simulations and reweighting the configurations with MBAR.

% at numerous temperatures and densities, \cite{Mostafa_Diss,Mostafa2018} these values are estimated by performing $NVT$ simulations and reweighting the configurations with MBAR.

% for parameter sets near the ``reference'' parameter set from which configurations are sampled 

The results from Messerly et al.~demonstrate that MBAR-ITIC is most reliable in the local domain, i.e., when $\theta_{\rm rr} \approx \theta_{\rm ref}$ \cite{Postdoc_1}. Furthermore, MBAR-ITIC performs best for changes in the non-bonded well-depth parameter $(\epsilon)$ while it performs significantly worse for large changes in the non-bonded size $(\sigma)$ and repulsive exponent $(\lambda)$ parameters. Because molecular configurations in condensed phases depend strongly on short-range interactions, this poor ``overlap'' is primarily observed when $\sigma_{\rm rr} \not\approx \sigma_{\rm ref}$ or $\lambda_{\rm rr} \not\approx \lambda_{\rm ref}$. The degree of overlap can be quantified by the number of effective snapshots $(K^{\rm eff}_{\rm snaps})$, which is essentially the number of non-negligible configurations that contribute to the estimated ensemble averages. Poor overlap (low $K^{\rm eff}_{\rm snaps}$) is especially problematic for MBAR-ITIC as a large number of snapshots is needed to obtain precise estimates of $P$ in the liquid phase, due to large fluctuations in $P$ at high densities.

%This ``poor overlap'' when $\theta_{\rm rr} \not\approx \theta_{\rm ref}$

% which are essential to obtain reasonable values of $\rho_{\rm liq}^{\rm sat}$.

%MBAR not only serves as a substitute for histogram reweighting in the standard GCMC-HR approach, but GCMC-MBAR can also be used to estimate coexistence properties for non-simulated parameter sets, which can greatly accelerate force field parameterization. In a similar study, Messerly et al. demonstrate how to combine MBAR with ITIC (MBAR-ITIC) to optimize Mie $\lambda$-6 (generalized Lennard-Jones) potentials \cite{Postdoc_1,Postdoc_2}. Since ITIC requires the internal energy $(U)$ and pressure $(P)$ (or compressibility factor, $Z$) at numerous temperatures and densities, \cite{Mostafa_Diss,Mostafa2018} these values are estimated by performing $NVT$ simulations and reweighting the configurations with MBAR. 

%The results from Messerly et al. demonstrate that MBAR-ITIC is most reliable in the local domain, i.e., for parameter sets near the ``reference'' parameter set from which configurations are sampled \cite{Postdoc_1}. Furthermore, MBAR-ITIC performs best for changes in the non-bonded well-depth parameter $(\epsilon)$ while it performs significantly worse for large changes in the non-bonded size and repulsive parameters $(\sigma$ and $\lambda$, respectively$)$. This is typically referred to as poor ``overlap.''

%precise $P$ calculations in the supercritical and liquid phase, which necessitates

%MRS3: not needed?
Our initial hypothesis was that GCMC-MBAR would experience better overlap than was observed for MBAR-ITIC when $\theta_{\rm rr} \not\approx \theta_{\rm ref}$. There were two main reasons for this hypothesis. First, as opposed to the fixed density $NVT$ simulations used in ITIC, GCMC simulations sample from a wider range of configurations and energies. Second, ITIC requires larger box sizes (and, thereby, more molecules) than those typically utilized with GCMC. By utilizing fewer molecules, GCMC simulations experience larger energy fluctuations (on a percent basis) which improves the overlap between states. We also hypothesized that the impact of poor overlap would be less severe compared to MBAR-ITIC, where poor overlap leads to sporadic values of $P$ and nonsensical phase equilibria estimates.  

%  require a large number of snapshots ITIC requires precise estimates of $P$ in the liquid phase, which necessitates a large number of of snapshots for

%configurations. $(NVT)$ does not sample from a wide range of energies  leads to large energy differences for small changes in $\sigma$ and $\lambda$. The fluctuating densities of GCMC simulations should accommodate greater changes in short-range interactions. 

The method outlined in this study is similar in spirit to ``Hamiltonian scaling'' (HS), which has been applied to both GEMC \cite{Kiyohara1996} and GCMC simulations \cite{Errington1998,Exp6,Errington1999,Pana2000}. The HS approach samples from multiple force fields (Hamiltonians) in a single simulation according to a weighted sampling probability. A separate histogram is stored for each $\theta_{\rm ref}$ by scaling the fractional contribution of the combined histogram from all reference force fields. Vapor-liquid phase equilibria properties for each $\theta_{\rm ref}$ are estimated post-simulation by applying traditional histogram reweighting to the respective scaled histograms. For the GCMC implementation of Hamiltonian scaling (HS-GCMC), $\mu$ and $T$ are not stationary during the simulation, rather the current values of $\mu$ and $T$ depend on which $\theta_{\rm ref}$ is being sampled. 

%The histograms from each force field are combined and scaled for each respective parameter set $(\theta_{\rm ref})$

Despite HS-GCMC proving to be a powerful tool to optimize force field parameters \cite{Errington1998,Exp6,Errington1999,Pana2000}, it has yet to gain widespread popularity. This is likely due to the added complexity of both the simulation protocol and the histogram post-processing. Also, HS requires that a decision be made \textit{a priori} regarding which force fields are to be tested. By contrast, GCMC-MBAR does not require any modification of the simulation procedure, the post-processing is essentially unchanged compared to traditional histogram reweighting, and the non-bonded parameter sets need not be selected prior to the simulations. This final distinction is of utmost importance as GCMC-MBAR is capable of predicting phase equilibria post-simulation for any force field parameter set ($\theta_{\rm rr} = \theta_{\rm ref}$ and $\theta_{\rm rr} \neq \theta_{\rm ref}$), whereas HS-GCMC can predict phase equilibria only for the parameter sets that are tested at run time.

%Recently, Weidler and Gross proposed ``individualized,'' i.e., compound-specific, parameter sets for compounds which contain large amounts of experimental data \cite{Weidler2018}. To avoid overfitting, a one-dimensional optimization is employed which scales $\epsilon$ for all united-atom sites while not adjusting $\sigma$ or $\lambda$. MBAR is ideally suited for this ``$\epsilon$-scaling'' approach for at least two reasons. First, as mentioned previously, MBAR is most reliable when extrapolating in $\epsilon$ rather than $\sigma$ and/or $\lambda$. Second, the rate-limiting step for MBAR is recomputing the configurational energies for a different force field. Furthermore, storing millions of configuration (``snapshots'') is highly memory intensive. While basis functions (see Section \ref{sec: Basis functions}) alleviate the additional computational cost and reduce the memory load, $\epsilon$-scaling does not require storing/recomputing configurations or basis functions. Instead, the energies for each snapshot are simply multiplied by the $\epsilon$-scaling parameter.  

%   Despite HS-GCMC proving to be a powerful tool to obtain coexistence curves for multiple force fields from a single set of simulations, it has yet to gain widespread popularity. This is likely due to the added complexity of the algorithm, where the prescribed $\mu$ and $T$ change during the coarse of the GCMC simulation, depending on which Hamiltonian (force field) is being sampled. Furthermore, the post-processing requires a slightly more complicated form of histogram reweighting. Also, HS requires a decision be made \textit{a priori} regarding which Hamiltonians are to be tested. By contrast, MBAR does not require any modification of the simulation procedure, the post-processing is essentially unchanged, and the Hamiltonians need not be selected prior to the simulations.

%The results from Messerly et al. demonstrated that MBAR is accurate over a wide range of $\epsilon$ (the Lennard-Jones well-depth parameter) values but less reliable for large changes in $\sigma$ (the Lennard-Jones size parameter) and $\lambda$ (the Mie $\lambda$-6 repulsive parameter, i.e., for Lennard-Jones 12-6 $\lambda = 12$). MBAR is most reliable in the local parameter space relative to the reference parameter set from which configurations are sampled.

%Some fundamental limitations exist for the MBAR-ITIC approach. First, ITIC is ill-suited for near-critical saturation properties, i.e., ITIC is not recommended for $T^{\rm sat} > 0.85 T_{\rm c}$ ($T_{\rm c}$ is the critical temperature). Second, ITIC requires a temperature correlation for the virial coefficients of the force field. Third, the poor extrapolation of MBAR with changes in $\sigma$ and $\lambda$.

The outline for this study is the following. Section \ref{sec: Methods} provides details regarding the force fields, simulation set-up, and the HR/MBAR post-simulation analysis. Section \ref{sec: Results} presents results of GCMC-MBAR for four scenarios. Section \ref{sec: Constant theta} validates that GCMC-MBAR and GCMC-HR yield indistinguishable coexistence estimates for a fixed force field. Section \ref{sec: eps scaling} applies GCMC-MBAR to a recently proposed $\epsilon$-scaling approach. Section \ref{sec: litFF} shows how GCMC-MBAR can predict coexistence properties for force field $j$ by reweighting configurations sampled with force field $i$. Section \ref{sec: Case study} demonstrates how GCMC-MBAR can be utilized to rapidly optimize the united-atom Mie $\lambda$-6 parameters for cyclohexane. Section \ref{sec: Discussion} discusses some limitations and provides recommendations for future work. Section \ref{sec: Conclusions} reviews the primary conclusions.

%provides a comparison of GCMC-MBAR and GCMC-HR as well as various applications of GCMC-MBAR for force field parameterization. Section \ref{sec: Discussion} discusses some limitations and provides recommendations for future work. Section \ref{sec: Conclusions} presents the primary conclusions.

% for of GCMC-MBAR  validates that GCMC-MBAR yields indistinguishable results from GCMC-HR, applies GCMC-MBAR to $\epsilon$-scaling, and . We demonstrate how
%

%%% Old introduction:
%
%A key use of molecular simulation is the ability to accurately and efficiently estimate vapor-liquid coexistence properties, i.e., saturated liquid density $(\rho_{\rm liq}^{\rm sat})$, saturated vapor density $(\rho_{\rm vap}^{\rm sat})$, saturated vapor pressures $(P_{\rm vap}^{\rm sat})$, and enthalpy of vaporization $(\Delta H_{\rm v})$. The accuracy of coexistence estimates depends on the underlying molecular model (a.k.a., force field, potential model, or Hamiltonian) while the computational efficiency depends primarily on the simulation methods, software, and hardware. Due to the abundance of experimental vapor-liquid coexistence data and the sensitivity of such properties to both short- and long-range non-bonded interactions, numerous force fields have been parameterized using $\rho_{\rm liq}^{\rm sat}$, $P_{\rm vap}^{\rm sat}$, and $\Delta H_{\rm v}$. Although the development of accurate force fields has been greatly enabled by the improved efficiency of simulation methods, parameterization of non-bonded interactions with vapor-liquid coexistence calculations remains an arduous and time-consuming task \cite{TraPPE,TAMie,Mie}.
%
%Several methods exist for computing vapor-liquid coexistence properties. These include Gibbs Ensemble Monte Carlo (GEMC), two phase molecular dynamics (2$\phi$MD), isothermal-isochoric integration (ITIC), and grand canonical Monte Carlo coupled with histogram reweighting (GCMC-HR). Advantages and disadvantages exist for each method. For example, GEMC and GCMC require insertion moves that are computationally inefficient for complex molecular structures with high density liquid phases. Several advanced simulation techniques are available to overcome this challenge \cite{ConBias}, which has enabled GEMC and GCMC-HR to be the primary methods of choice for vapor-liquid coexistence calculations.
%
%Some clear advantages and disadvantages exist for GCMC-HR compared with GEMC. For example, one advantage of GCMC-HR is the higher precision \cite{GEMC_GCMC}. Furthermore, coexistence properties can be computed at temperatures that are not simulated directly. However, GEMC is arguably more straightforward in that simulations are performed only at the desired saturation temperatures $(T^{\rm sat})$. By contrast, GCMC-HR requires a series of GCMC simulations for a single $T^{\rm sat}$. This set includes a near-critical simulation that ``bridges'' the vapor and liquid phases. Obtaining the appropriate chemical potential $(\mu)$ for this bridge simulation is a cumbersome and, typically, iterative process (although more advanced methods exist to obtain a good initial estimate for $\mu$ \cite{Hemmen2015}).
%
%Another disadvantage of GCMC-HR compared to GEMC is that GCMC-HR requires more post-processing (i.e., histogram reweighting), while simple block averaging is typically sufficient for GEMC. Histogram reweighting (and more generally, configuration reweighting) is an important tool in many fields of molecular simulation. In fact, it has long since been known that it is possible to estimate properties for state $j$ by reweighting configurations that were sampled with state $i$. \cite{McDonald1967,Card1970,Wood1968,Pana2000} For example, umbrella sampling simulations are often processed using the weighted histogram analysis method (WHAM) to compute free energy differences between states. A popular alternative to WHAM is the Multistate Bennett Acceptance Ratio (MBAR) \cite{chodera:jctc:2007,shirts-chodera:jcp:2008:mbar}, which is readily available in the \textit{pymbar} package.
%
%In this study, we substitute HR with MBAR for the GCMC-HR approach of computing vapor-liquid coexistence properties. Section \ref{sec: GCMC-HR and GCMC-MBAR} demonstrates that MBAR and HR are mathematically equivalent (in the limit of zero bin width) while Section \ref{sec: Results} shows that they are also numerically equivalent (to within statistical uncertainties). Note that, as Boulougouris et al. demonstrate how to combine HR with GEMC (GEMC-HR) to estimate saturation properties at non-simulated temperatures \cite{Boulougouris2010}, MBAR could alternatively be applied to GEMC simulations.
%
%Substituting the standard HR approach with MBAR is not the primary purpose of this study. Rather, we demonstrate how GCMC-MBAR can also estimate coexistence properties for non-simulated parameter sets, which can greatly accelerate force field parameterization. In a similar study, Messerly et al. demonstrate how to combine MBAR with ITIC (MBAR-ITIC) to optimize Mie $\lambda$-6 (generalized Lennard-Jones) potentials \cite{Postdoc_1,Postdoc_2}. For MBAR-ITIC, a series of $NVT$ simulations along an isotherm and isochores are performed with a ``reference'' force field $(\theta_{\rm ref})$. MBAR computes the internal energy $(U)$ and pressure $(P)$ (or compressibility factor, $Z$) for each $T-\rho$ state point with a non-simulated (``rerun'') force field $(\theta_{\rm rr})$. ITIC then converts the $U$ and $P$ values into vapor-liquid coexistence properties \cite{Mostafa_Diss,Mostafa2018}.
%
%The results from Messerly et al. demonstrate that MBAR-ITIC is most reliable in the local domain, i.e., for parameter sets near the ``reference'' parameter set from which configurations are sampled \cite{Postdoc_1}. Furthermore, MBAR-ITIC performs best for changes in the non-bonded well-depth parameter $(\epsilon)$ while it performs significantly worse for large changes in the non-bonded size and repulsive parameters $(\sigma$ and $\lambda$, respectively$)$. This is typically referred to as poor ``overlap'' and can be quantified by the ``number of effective samples'' $(K^{\rm eff}_{\rm snaps})$, which is essentially the number of non-negligible samples that contribute to the estimated ensemble averages. Poor overlap (low $K^{\rm eff}_{\rm snaps}$) is especially problematic for ITIC as a large number of snapshots is needed to obtain precise estimates of $P$ in the liquid phase, which are essential to obtain reasonable values of $\rho_{\rm liq}^{\rm sat}$.
%
%Our initial hypothesis was that GCMC-MBAR should have better overlap over the non-bonded parameter space than what was observed for MBAR-ITIC. There are two main reasons for this hypothesis/aspiration. First, as opposed to the fixed density $NVT$ simulations, the fluctuating density of a GCMC simulation produces a wider range of configurations and energies. Second, ITIC requires accurate calculations of $U$ and $P$ in the vapor phase, which necessitate larger box sizes (and, thereby, more molecules) than those typically utilized with GCMC. By utilizing fewer molecules, GCMC simulations experience larger energy fluctuations (on a percent basis) which improves the overlap between states. We also hypothesized that the impact of poor overlap would be less severe compared to ITIC, where poor overlap leads to sporadic and nonsensical coexistence estimates.  
%
%The method outlined in this study is similar in spirit to ``Hamiltonian scaling'' (HS), which has been applied to both GEMC \cite{Kiyohara1996} and GCMC simulations \cite{Errington1998,Exp6,Errington1999,Pana2000}. The HS approach samples from multiple force fields (Hamiltonians) in a single simulation according to a weighted sampling probability. Vapor-liquid coexistence curves for each force field are estimated post-simulation by reweighting the configurations accordingly. For the grand canonical Monte Carlo implementation of Hamiltonian scaling (HS-GCMC), $\mu$ and $T$ are not stationary during the simulation, rather the current value of $\mu$ and $T$ depends on which force field is being sampled. Despite HS-GCMC proving to be a powerful tool to optimize force field parameters \cite{Errington1998,Exp6,Errington1999,Pana2000}, it has yet to gain widespread popularity. This is likely due to the added complexity of both the simulation protocol and the histogram post-processing. Also, HS requires that a decision be made \textit{a priori} regarding which force fields are to be tested. By contrast, MBAR does not require any modification of the simulation procedure, the post-processing is essentially unchanged, and the non-bonded parameter sets need not be selected prior to the simulations.
%
%Recently, Weidler and Gross proposed ``individualized,'' i.e., compound-specific, parameter sets for compounds which contain large amounts of experimental data \cite{Weidler2018}. To avoid overfitting, a one-dimensional optimization is employed which scales $\epsilon$ for all united-atom sites while not adjusting $\sigma$ or $\lambda$. MBAR is ideally suited for this ``$\epsilon$-scaling'' approach for at least two reasons. First, as mentioned previously, MBAR is most reliable when extrapolating in $\epsilon$ rather than $\sigma$ and/or $\lambda$. Second, the rate-limiting step for MBAR is recomputing the configurational energies for a different force field. Furthermore, storing millions of configuration ``snapshots'' is highly memory intensive. While basis functions (see Section \ref{sec: Basis functions}) alleviate the additional computational cost and reduce the memory load, $\epsilon$-scaling does not require storing/recomputing configurations or basis functions. Instead, the energies for each snapshot are simply multiplied by the $\epsilon$-scaling parameter.  
%
%The outline for this study is the following. Section \ref{sec: Methods} provides details regarding the force fields, simulation set-up, and post-simulation analysis with MBAR. Section \ref{sec: Results} provides a comparison of GCMC-MBAR and GCMC-HR as well as various applications of GCMC-MBAR for force field parameterization. Section \ref{sec: Discussion} discusses some limitations and provides recommendations for future work. Section \ref{sec: Conclusions} presents the primary conclusions.

%%% Original outline
%\begin{enumerate}
%	\item Accurate and efficient computation of vapor-liquid coexistence is an important but challenging task for molecular simulation
%	\item Reweighting simulation outputs between different states is a well-known and powerful tool, e.g., histogram reweighting of GCMC results
%	\item Force field parameterization with VLE data is an arduous and time-consuming task
%	\item Hamiltonian scaling (histogram reweighting for multiple force fields) allows for estimating VLE properties of multiple force fields from single set of simulations
%	\item Messerly et al. demonstrated how MBAR can be combined with ITIC to predict VLE properties. Several  Weakness of ITIC is need large systems, which is not ideal for MBAR
%	\item Gross demonstrated benefits of $\epsilon$-scaling for ``individualized'', i.e., compound-specific parameter sets
%	\item In this study, we demonstrate that:
%	\begin{enumerate}
%		\item MBAR yields indistinguishable results from histogram reweighting (HR)
%		\item Scaling epsilon is straightforward by scaling U with MBAR
%		\item MBAR can estimate VLE properties for multiple force fields simultaneously
%		\item Basis functions allow for rapid computation of VLE for non-simulated Mie parameter sets
%	\end{enumerate}
%\end{enumerate}

\section{Methods} \label{sec: Methods}

\subsection{Force fields} \label{sec: Force fields}

    The force fields utilized in this study are Mie Potentials for Phase Equilibria (MiPPE) \cite{Mie,Potoff_branched,Barhaghi2017}, Transferable Potentials for Phase Equilibria (TraPPE-UA, referred to simply as TraPPE \cite{TraPPE,Martin1999,Keasler2012}), and Nath, Escobedo, and de Pablo revised (NERD) \cite{NERD,Nath2001}. Each force field adopts a united-atom (UA) representation, where non-polar hydrogens are not modeled explicitly. 
    
    %The non-bonded potential is of the generalized Lennard-Jones (Mie $\lambda$-6) form. 

%    For computational efficiency, we use fixed bond lengths for each force field. Although the original NERD publication\cite{NERD} utilizes a harmonic bond potential, previous studies have demonstrated that fixed and flexible bonds typically yield indistinguishable vapor-liquid phase equilibria \cite{Mie,Mess3}. Furthermore, we compare our NERD results with those of Mick et al.\cite{Potoff_branched}, which were also obtained with fixed bonds.     
%    In addition to the increased computational efficiency, the simulation package utilized in this study
    
    We employ fixed bond lengths for each force field studied. Note that this is inconsistent with the original NERD force field, which was developed using a harmonic bond potential. The primary reason we utilize fixed bond lengths for the NERD potential is to allow for a valid comparison of our GCMC-MBAR values with the GCMC-HR results of Mick et al.\cite{Potoff_branched}, which were also obtained using fixed bonds. Furthermore, Section \ref{SI sec: Fixed vs flexible bonds} of Supporting Information demonstrates that the branched alkane NERD phase equilibria results obtained with fixed bonds (as reported by Mick et al.\cite{Potoff_branched}) agree with the flexible-bond results (as reported by Nath et al.\cite{Nath2001}).
    
    %Fixed bonds are also more computationally efficient in Monte Carlo simulations and, in addition, flexible bonds are not yet supported by the simulation package utilized in this study (see Section \ref{sec: Simulation set-up}).
    
    %that the fixed-bond phase equilibria values reported by Mick et al.\cite{Potoff_branched} are indistinguishable from the flexible-bond results reported by Nath et al.\cite{Nath2001} (see Section \ref{SI sec: Fixed vs flexible bonds} of Supporting Information).
    
    % and flexible bonds yield indistinguishable vapor-liquid phase equilibria
    
    % While several studies suggest that fixed and flexible bonds yield indistinguishable vapor-liquid phase equilibria \cite{Mie,Mess3,C100,Smit}, other studies demonstrate that this agreement likely depends on the thermodynamic state point \cite{Toxvaerd1989,Patel2011}.
    
%     does not yet permit simulating flexible bonds.  
    
%    Furthermore, we compare our NERD results with those of Mick et al.\cite{Potoff_branched}, which were also obtained with fixed bonds. 
    
    Angular bending interactions for each force field are evaluated using a harmonic potential:
    \begin{equation}
    u^{\rm bend} = \frac{k_\theta}{2} \left(\theta-\theta_{\rm eq}\right)^2
    \end{equation}
    where $u^{\rm bend}$ is the bending energy, $\theta$ is the instantaneous bond angle, $\theta_{\rm eq}$ is the equilibrium bond angle, and $k_\theta$ is the harmonic force constant. 
    
    Dihedral torsional interactions for each force field are determined using a cosine series:
    \begin{equation} \label{eq: tors}
    u^{\rm tors} = c_0 + c_1 \left(1+\cos{\phi}\right) + c_2 \left(1-\cos{2\phi}\right) + c_3 \left(1+\cos{3\phi}\right)
    \end{equation}
    where $u^{\rm tors}$ is the torsional energy, $\phi$ is the dihedral angle and $c_n$ are the Fourier constants. Bond lengths, $\theta_{\rm eq}$, $k_\theta$, and $c_n$ values for each force field are reported in Section \ref{SI sec: Bonded parameters} of Supporting Information.
    
    In accordance with Reference \citenum{Yiannourakou2019}, we simulate cyclohexane using the TraPPE CH$_x$-CH$_2$-CH$_2$-CH$_y$ torsional parameters instead of the TraPPE C-C-C-C six-member ring torsional parameters reported in Table 3 of Reference \citenum{Keasler2012}. This choice is made to better replicate the vapor-liquid coexistence densities reported in Reference \citenum{Keasler2012}, as we suspect there is a typographical error in Reference \citenum{Keasler2012} for the TraPPE C-C-C-C six-member ring torsional potential. Note that, subsequent to submitting this manuscript, an alternative equation to Equation \ref{eq: tors} is now provided on the official TraPPE website\cite{TraPPE_website} for the cyclohexane torsional potential (although we are still wary of a possible sign error for the $c_1$ term).
    
%    (possibly a sign error for the $c_1$ term)
    
    %  () s suspition Note that, subsequent to submission of this manuscript, the equation for the cyclohexane torsional potential was  for cyclohexane was listed on the official TraPPE website was modified. Note that subsequent to our 
    
    % After investigating the dihedral distributions for cyclohexane obtained with the TraPPE C-C-C-C six-member ring torsional parameters, we suspect there is a sign error for the $c_1$ term in Reference \citenum{Keasler2012}.
    
    %%% Old version, moved to SI    
%    For computational efficiency, we utilize fixed bond lengths for each force field, although the original NERD publication utilizes a harmonic bond potential \cite{NERD}. The bond lengths for all bonds, compounds, and force fields investigated in this study are reported in Table \ref{tab:bonds}.
    
%    The bond lengths for all bonds, compounds, and force fields simulated in this study are 0.154 nm. Although we do not simulate alkynes, the MiPPE alkyne bond lengths are also reported in Table \ref{tab:bonds} because we re-analyze the MiPPE alkyne simulation data produced by Soroush Barhaghi et al.
    
    % The bond length values required to simulate the branched alkanes, alkynes, and cyclohexane with the for each force field  
    %, but this typically does not impact vapor-liquid coexistence properties. 
    % rather than a harmonic bond potential (although this was im)
%    Fixed bond lengths are utilized in this study. The bond lengths for the TraPPE, MiPPE, and NERD force fields are 0.154 nm for all compounds studied. 

%    \begin{table}[h!]
%		\caption{Equilibrium (fixed) bond lengths $(r_{\rm eq})$ \cite{Martin1999}.} \label{tab:bonds}
%		\begin{center}
%			\begin{tabular}{|c|c|c|c|}
%				\hline
%				Bending sites & \multicolumn{3}{|c|}{$r_{\rm eq}$ (nm)} \\ \hline
%				& TraPPE & MiPPE & NERD \\ \hline
%				CH$_2$-CH$_3$ & 0.154 & 0.154 & 0.154 \\ 
%				CH$_2$-CH$_2$ & 0.154 & 0.154 & 0.154 \\ 
%				CH$_2$-CH & 0.154 & 0.154 & 0.154 \\ 
%				CH$_2$-C & 0.154 & 0.154 & 0.154 \\
%				CH-CH$_3$ & 0.154 & 0.154 & 0.154 \\ 
%				CH-CH$_2$ & 0.154 & 0.154 & 0.154 \\ 
%				CH-CH & 0.154 & 0.154 & 0.154 \\ 
%				CH-C & 0.154 & 0.154 & 0.154 \\ 
%				CH$_x$-CH-CH$_y$ & 112.0 & 112.0 & 109.5 \\
%				CH$_x$-C-CH$_y$ & 109.5 & 109.5 & 109.5 \\
%				\hline
%			\end{tabular}
%		\end{center} 
%	\end{table}

%%% Old version, moved to SI
%    \begin{table}[h!]
%		\caption{Equilibrium (fixed) bond lengths $(r_{\rm eq})$ \cite{Martin1999,Nath2001,Potoff_branched,Barhaghi2017}. CH$_x$ and CH$_y$ represent CH$_3$, CH$_2$(sp$^3$), CH(sp$^3$), or C(sp$^3$) sites.} \label{tab:bonds}
%		\begin{center}
%			\begin{tabular}{|c|c|c|c|}
%				\hline
%				Bending sites & \multicolumn{3}{|c|}{$r_{\rm eq}$ (nm)} \\ \hline
%				& TraPPE & MiPPE & NERD \\ \hline
%				CH$_x$-CH$_y$ & 0.154 & 0.154 & 0.154 \\ 
%				C(sp)-CH$_x$ & -- & 0.146 & -- \\
%				CH$\equiv$CH & -- & 0.121 & -- \\ 
%				C$\equiv$CH & -- & 0.121 & -- \\
%				\hline
%			\end{tabular}
%		\end{center} 
%	\end{table}
%    
%    Angular bending interactions for each force field are evaluated using a harmonic potential:
%    \begin{equation}
%    u^{\rm bend} = \frac{k_\theta}{2} \left(\theta-\theta_{\rm eq}\right)^2
%    \end{equation}
%    where $u^{\rm bend}$ is the bending energy, $\theta$ is the instantaneous bond angle, $\theta_{\rm eq}$ is the equilibrium bond angle, and $k_\theta$ is the harmonic force constant. Table \ref{tab:angles} provides the $\theta_{\rm eq}$ and $k_\theta$ values for each angle type and force field. Note that the $\theta_{\rm eq}$ and $k_\theta$ values are equivalent for the three force fields, with the exception of the NERD CH$_x$-CH-CH$_y$ $\theta_{\rm eq}$ value. 
%%%
 
%    with $k_\theta/k_{\rm B} = 62500$ K/rad$^2$ for all bonding angles, where $k_{\rm B}$ is the Boltzmann constant
    
%    \begin{table}[h!]
%    	\caption{Equilibrium bond angles $(\theta_{\rm eq})$ \cite{Martin1999}. CH$_i$ and CH$_j$ represent CH$_3$, CH$_2$, CH, or C sites.} \label{tab:angles}
%    	\begin{center}
%    		\begin{tabular}{|c|c|}
%    			\hline
%    			Bending sites & $\theta_{\rm eq}$ (degrees) \\ \hline
%    			CH$_i$-CH$_2$-CH$_j$ & 114.0 \\ 
%    			CH$_i$-CH-CH$_j$ & 112.0 \\
%    			CH$_i$-CH-CH$_j$, NERD & 109.5 \\ 
%    			CH$_i$-C-CH$_j$ & 109.5 \\
%    			\hline
%    		\end{tabular}
%    	\end{center} 
%    \end{table}

%    \begin{table}[h!]
%		\caption{Equilibrium bond angles $(\theta_{\rm eq})$ \cite{Martin1999}. CH$_x$ and CH$_y$ represent CH$_3$, CH$_2$, CH, or C sites.} \label{tab:angles}
%		\begin{center}
%			\begin{tabular}{|c|c|c|c|}
%				\hline
%				Bending sites & \multicolumn{3}{|c|}{$\theta_{\rm eq}$ (degrees)} \\ \hline
%				& TraPPE & MiPPE & NERD \\ \hline
%				CH$_x$-CH$_2$-CH$_y$ & 114.0 & 114.0 & 114.0 \\ 
%				CH$_x$-CH-CH$_y$ & 112.0 & 112.0 & 109.5 \\
%				CH$_x$-C-CH$_y$ & 109.5 & 109.5 & 109.5 \\
%				\hline
%			\end{tabular}
%		\end{center} 
%	\end{table}

%%% Old version, moved to SI
%%% Included alkyne parameters
%    \begin{table}[h!]
%		\caption{Equilibrium bond angles $(\theta_{\rm eq})$ and force constants $(k_\theta/k_{\rm B})$, where $k_{\rm B}$ is the Boltzmann constant \cite{Martin1999,Nath2001,Potoff_branched,Barhaghi2017}. CH$_x$ and CH$_y$ represent CH$_3$, CH$_2$(sp$^3$), CH(sp$^3$), or C(sp$^3$) sites.} \label{tab:angles}
%		\begin{center}
%			\begin{tabular}{|c|c|c|c|c|}
%				\hline
%				Bending sites & \multicolumn{3}{|c|}{$\theta_{\rm eq}$ (degrees)} & $k_\theta/k_{\rm B}$ (K/rad$^2$) \\ \hline
%				& TraPPE & MiPPE & NERD & \\ \hline
%				CH$_x$-CH$_2$-CH$_y$ & 114.0 & 114.0 & 114.0 &  62500 \\ 
%				CH$_x$-CH-CH$_y$ & 112.0 & 112.0 & 109.5 & 62500 \\
%				CH$_x$-C-CH$_y$ & 109.5 & 109.5 & 109.5 & 62500 \\
%				CH$_x$-CH$_2$-C(sp) & -- & 112 & -- & 62500 \\
%				CH$_x$-C(sp)$\equiv$CH & -- & 180 & -- & 30800 \\
%				CH$_x$-C(sp)$\equiv$C & -- & 180 & -- & 30800 \\
%				\hline
%			\end{tabular}
%		\end{center} 
%	\end{table}
%    
%    Dihedral torsional interactions for each force field are determined using a cosine series:
%    \begin{equation}
%    u^{\rm tors} = c_0 + c_1 \left(1+\cos{\phi}\right) + c_2 \left(1-\cos{2\phi}\right) + c_3 \left(1+\cos{3\phi}\right)
%    \end{equation}
%    where $u^{\rm tors}$ is the torsional energy, $\phi$ is the dihedral angle and $c_n$ are the Fourier constants listed in Table \ref{tab:torsions}. In accordance with Reference \citenum{Yiannourakou2019}, we simulate cyclohexane using the TraPPE CH$_x$-CH$_2$-CH$_2$-CH$_y$ torsional parameters instead of the TraPPE C-C-C-C six-member ring torsional parameters reported in Table 3 of Reference \citenum{Keasler2012}. This choice is made to better replicate the vapor-liquid coexistence densities reported in Reference \citenum{Keasler2012}. After investigating the dihedral distributions for cyclohexane obtained with the torsional parameters from Reference \cite{Keasler2012}, we suspect there is a sign error for at least the $c_1$ term.
    
    % reported in Reference \citenum{Keasler2012}.
    
    %, as changing the sign of the $c_1$ term for TraPPE C-C-C-C six-member rings also we obtained reliable coexistence densitie 
    
    %Alternatively, we obtained reliable coexistence densities by changing the sign of the $c_1$ term reported in Table 3 of Reference \citenum{Keasler2012}. Therefore, we suspect a sign error for the $c_1$ term reported in \citenum{Keasler2012}. We found it necessary to use th in order to replicate the vapor-liquid coexistence densities reported in Reference \citenum{Keasler2012}. Alternatively, we obtained reliable results by s found similar results We suspect a sign error the $c_1$ term reported in \citenum{Keasler2012} % with the TraPPE cyclic torsional potential.  
 
 %%% Only branched alkanes and cyclohexane   
%    \begin{table}[h!]
%    	\caption{Fourier constants $(c_n/k_{\rm B})$ in units of K \cite{Martin1999}. CH$_x$ and CH$_y$ represent CH$_3$, CH$_2$, CH, or C sites.} \label{tab:torsions}
%    	\begin{center}
%    		\begin{tabular}{|c|c|c|c|c|}
%    			\hline
%    			Torsion sites & $c_0/k_{\rm B}$ & $c_1/k_{\rm B}$ & $c_2/k_{\rm B}$ & $c_3/k_{\rm B}$ \\ \hline
%    			CH$_x$-CH$_2$-CH$_2$-CH$_y$ & 0.0 & 355.03 & -68.19 & 791.32 \\ 
%    			CH$_x$-CH$_2$-CH-CH$_y$ & -251.06 & 428.73 & -111.85 & 441.27 \\
%    			CH$_x$-CH$_2$-C-CH$_y$ & 0.0 & 0.0 & 0.0 & 461.29 \\
%    			CH$_x$-CH-CH-CH$_y$ & -251.06 & 428.73 & -111.85 & 441.27 \\
%    			\hline
%    		\end{tabular}
%    	\end{center} 
%    \end{table}

%%% Old version, moved to SI
%    \begin{table}[h!]
%		\caption{Fourier constants $(c_n/k_{\rm B})$ in units of K \cite{Martin1999,Nath2001,Potoff_branched,Barhaghi2017}. CH$_x$ and CH$_y$ represent CH$_3$, CH$_2$(sp$^3$), CH(sp$^3$), or C(sp$^3$) sites.} \label{tab:torsions}
%		\begin{center}
%			\begin{tabular}{|c|c|c|c|c|}
%				\hline
%				Torsion sites & $c_0/k_{\rm B}$ & $c_1/k_{\rm B}$ & $c_2/k_{\rm B}$ & $c_3/k_{\rm B}$ \\ \hline
%				CH$_x$-CH$_2$-CH$_2$-CH$_y$ & 0.0 & 355.03 & -68.19 & 791.32 \\ 
%				CH$_x$-CH$_2$-CH-CH$_y$ & -251.06 & 428.73 & -111.85 & 441.27 \\
%				CH$_x$-CH$_2$-C-CH$_y$ & 0.0 & 0.0 & 0.0 & 461.29 \\
%				CH$_x$-CH-CH-CH$_y$ & -251.06 & 428.73 & -111.85 & 441.27 \\
%				CH$_x$-CH$_2$-CH$_2$-C(sp) & 94.88 & 162.00 & -205.40 & 980.40 \\
%				CH$_x$-CH$_2$-C(sp)$\equiv$C(sp) & 0 & 0 & 0 & 0 \\
%				CH$_x$-CH$_2$-C(sp)$\equiv$CH(sp) & 0 & 0 & 0 & 0 \\
%				CH$_x$-C(sp)$\equiv$C(sp)-CH$_y$ & 0 & 0 & 0 & 0 \\
%				\hline
%			\end{tabular}
%		\end{center} 
%	\end{table}
    
    Non-bonded interactions between sites located in two different molecules or separated by more than three bonds within the same molecule are calculated using a Mie $\lambda$-6 potential (of which the Lennard-Jones, LJ, 12-6 is a subclass):
    \begin{equation} \label{eq:Mie}
    u^{\rm nb}(\epsilon,\sigma,\lambda;r) = \left(\frac{\lambda}{\lambda - 6}\right)\left(\frac{\lambda}{6}\right)^{\frac{6}{\lambda - 6}} \epsilon \left[\left(\frac{\sigma}{r}\right)^{\lambda} - \left(\frac{\sigma}{r}\right)^6\right]
    \end{equation} 
    where $u^{\rm nb}$ is the non-bonded energy, $\sigma$ is the distance $(r)$ where $u^{\rm nb} = 0$, $-\epsilon$ is the energy of the potential at the minimum $\left(\text{i.e., }u^{\rm nb} = -\epsilon \text{ and } \frac{\partial u^{\rm nb}}{\partial r} = 0 \text{ for } r=r_{\rm min} \right)$, and $\lambda$ is the repulsive exponent. Note that Coulombic (electrostatic) interactions are not computed because the MiPPE, TraPPE, and NERD force fields do not include charges for any of the compounds studied.
    
    The non-bonded Mie $\lambda$-6 force field parameters for MiPPE, TraPPE, and NERD are provided in Table \ref{tab:nonbonded params}. MiPPE reports a ``generalized'' (MiPPE-gen) and ``short/long'' (MiPPE-SL) CH and C parameter set. The ``short'' and ``long'' parameters are implemented when the number of carbons in the backbone is $\le 4$ and $> 4$, respectively. Also note that the NERD force field has several different CH$_3$ non-bonded parameter sets.
    
%%% Old format
%    \begin{table}[h!]
%    	\caption{Non-bonded (intermolecular) parameters for TraPPE \cite{TraPPE,Martin1999}, Potoff \cite{Mie,Potoff_branched}, and NERD \cite{NERD}. The ``short/long'' Potoff CH and C parameters are included in parentheses.} \label{tab:nonbonded params}
%    	\begin{center}
%    		\begin{tabular}{|c|c|c|c|c|c|c|}
%    			\hline
%    			United-atom & $\epsilon/k_{\rm B}$ (K) & $\sigma$ (nm) & $\lambda$ & $\epsilon/k_{\rm B}$ (K) & $\sigma$ (nm) & $\lambda$ \\ \hline
%    			\multicolumn{1}{|c}{} & \multicolumn{3}{|c}{TraPPE} & \multicolumn{3}{|c|}{Potoff (S/L)}  \\ \hline
%    			CH$_3$ & 98 (134.5)  & 0.375 (0.352) & 12 & 121.25 & 0.3783 & 16  \\ 
%    			CH$_2$ & 46 & 0.395 & 12 & 61 & 0.399 & 16 \\ 
%    			CH & 10 & 0.468 & 12 & 15 (15/14) & 0.46 (0.47/0.47) & 16\\
%    			C & 0.5 & 0.640 & 12 & 1.2 (1.45/1.2) & 0.61 (0.61/0.62) & 16\\
%    			\hline
%    			\multicolumn{1}{|c}{} & \multicolumn{3}{|c}{NERD} & \multicolumn{3}{|c|}{} \\ \hline
%    			CH$_3$ & 104.00  & 0.3910 & 12 & -- & -- & --\\ 
%                CH$_3$ (2-methylpropane) & 78.23  & 0.3880 & 12 & -- & -- & --\\ 
%                CH$_3$ (2,2-dimethylpropane) & 74.50  & 0.3910 & 12 & -- & -- & --\\ 
%                CH$_3$ & 104.00  & 0.3910 & 12 & -- & -- & --\\ 
%    			CH$_2$ & 45.80 & 0.3930 & 12 & -- & -- & -- \\ 
%    			CH & 39.70 & 0.3850 & 12 &  -- & -- & --\\
%    			C & 17.00 & 0.3910 & 12 & -- & -- & --\\
%    			\hline
%    		\end{tabular}
%    	\end{center} 
%    \end{table}

    \begin{table}[h!]
%    	\begin{threeparttable}
		\caption{Non-bonded (Mie $\lambda$-6) parameters for TraPPE \cite{TraPPE,Martin1999,Keasler2012}, MiPPE \cite{Mie,Potoff_branched,Barhaghi2017}, and NERD \cite{NERD,Nath2001}.} \label{tab:nonbonded params}
		\begin{center}
			\begin{tabular}{|c|c|c|c|}
				\hline
				United-atom & $\epsilon/k_{\rm B}$ (K) & $\sigma$ (nm) & $\lambda$ \\ \hline
				\multicolumn{4}{|c|}{MiPPE} \\ \hline
				CH$_3$ & 121.25 & 0.3783 & 16  \\ 
				CH$_2$(sp$^3$) & 61 & 0.399 & 16 \\ 
				CH(sp$^3$), gen. & 15 & 0.46 & 16\\
				C(sp$^3$), gen. & 1.2 & 0.61 & 16\\
				CH(sp$^3$), short & 15 & 0.47 & 16\\
				C(sp$^3$), short & 1.45 & 0.61 & 16\\
				CH(sp$^3$), long & 14 & 0.47 & 16\\
				C(sp$^3$), long & 1.2 & 0.62 & 16\\
				CH(sp) & 148.5 & 0.357 & 28\\
				C(sp) (1-alkyne) & 206 & 0.2875 & 16\\
				C(sp) (2-alkyne) & 118 & 0.312 & 16\\
				CH$_2$ (cyclohexane)$^a$ & 69.7 & 0.3902 & 16 \\
				\hline
                \multicolumn{4}{|c|}{TraPPE} \\ \hline
                CH$_3$ & 98 & 0.375 & 12 \\ 
                CH$_2$(sp$^3$) & 46 & 0.395 & 12\\ 
                CH(sp$^3$) & 10 & 0.468 & 12 \\
                C(sp$^3$) & 0.5 & 0.640 & 12 \\
                CH$_2$ (cyclohexane) & 52.5 & 0.391 & 12 \\
                \hline
                \multicolumn{4}{|c|}{NERD} \\ \hline
				CH$_3$ (general) & 104.00  & 0.391 & 12\\ 
				CH$_3$ (2-methylpropane) & 78.23  & 0.388 & 12\\ 
				CH$_3$ (2,2-dimethylpropane) & 74.50  & 0.391 & 12 \\  
				CH$_3$ (methyl side chain) & 70.00 & 0.385 & 12 \\
				CH$_3$ (ethyl side chain) & 83.00 & 0.382 & 12 \\
				CH$_2$(sp$^3$) & 45.80 & 0.393 & 12 \\ 
				CH(sp$^3$) & 39.70 & 0.385 & 12\\
				C(sp$^3$) & 17.00 & 0.391 & 12 \\
				\hline
				\multicolumn{4}{|l|}{$^a$ This work. See Section \ref{sec: Case study}} \\
				\hline
			\end{tabular}
		\end{center}
	    %\raggedright{$^a$ This work. See Section \ref{sec: Case study}}
%	    \begin{tablenotes}
%	    	\small
%	    	\item $^a$ This work. See Section \ref{sec: Case study}
%	    \end{tablenotes}
%	    \end{threeparttable}
	\end{table}
    
    Non-bonded parameters between two different site types (i.e., cross-interactions) are determined using Lorentz-Berthelot combining rules \cite{Allen1987} for $\epsilon$ and $\sigma$ and an arithmetic mean for the repulsive exponent $\lambda$ (as recommended in Reference \citenum{Mie}):
    \begin{equation} \label{eq:Lorentz-Berthelot_eps}
    \epsilon_{ij} = \sqrt{\epsilon_{ii} \epsilon_{jj}}
    \end{equation}
    \begin{equation} \label{eq:Lorentz-Berthelot_sig}
    \sigma_{ij} = \frac{\sigma_{ii} + \sigma_{jj}}{2}
    \end{equation}
    \begin{equation} \label{eq:Lorentz-Berthelot_lam}
    \lambda_{ij} = \frac{\lambda_{ii} + \lambda_{jj}}{2}
    \end{equation}
    where the $ij$ subscript refers to cross-interactions and the subscripts $ii$ and $jj$ refer to same-site interactions. 
    
%\begin{enumerate}
%	\item Simulations are performed for united-atom generalized Lennard-Jones (a.k.a., Mie $\lambda$-6) force fields
%	\item We investigate the TraPPE, Potoff-generalized, Potoff (S/L), and NERD force fields 
%	\item Details of force fields
%\end{enumerate}

\subsection{Simulation set-up} \label{sec: Simulation set-up}

The results presented in Sections \ref{sec: Constant theta} and \ref{sec: eps scaling} are obtained by reprocessing simulation output that were analyzed in previous studies with histogram reweighting \cite{Potoff_branched,Barhaghi2017}. New simulation results are provided in Sections \ref{sec: litFF} and \ref{sec: Case study} for 2-methylpropane, 2,2-dimethylpropane, 2,2-dimethylbutane, 3,3-dimethylhexane, 3-methyl-3-ethylpentane, 2,2,4-trimethylhexane, 2,3-dimethylbutane, 2,3,4-trimethylpentane, and cyclohexane. 

Each compound is simulated with grand canonical Monte Carlo (GCMC), i.e., constant chemical potential ($\mu$), volume $(V)$, and temperature $(T)$. A series of nine GCMC simulations are performed, two in the vapor phase, six in the liquid phase (with a temperature spacing of approximately 20 K), and one near critical which acts as the ``bridge'' between the vapor and liquid phases. A single simulation is performed at each state point $(\mu$, $V$, $T)$, with the exception of the MiPPE cyclohexane results, which are obtained from 20 independent replicate simulations. 

The system volume is constant for a given compound. The cubic box side length is 3 nm for 2-methylpropane, 2,2-dimethylpropane, 2,3-dimethylbutane, and cyclohexane, 3.5 nm for 2,3,4-trimethylpentane and 3,3-dimethylhexane, and 4 nm for 2,2,4-trimethylhexane and 3-methyl-3-ethylpentane. The prescribed $\mu$, $T$, and $V$ values for the branched alkanes are the same as those utilized in Mick et al.~and vary somewhat between force fields \cite{Potoff_branched}. All simulated state points $(\mu$, $V$, $T)$ are reported in Section \ref{SI sec: State Points} of Supporting Information. 

A low-density (less than twenty molecules) initial configuration is utilized for the vapor phase simulations, while the bridge and liquid phase simulations are initialized with a high-density (around 150 molecules) configuration. To verify that finite size effects are negligible in the low-density vapor phase, we confirm that the saturated vapor compressibility factor $(Z^{\rm sat}_{\rm vap})$ converges smoothly to 1 for each compound (see Section \ref{SI sec: Z} of Supporting Information) \cite{Nezbeda2016}.

% The $\mu-T$ values for \textit{n}-hexane and cyclohexane are determined in this study.

Each GCMC simulation performed in this study consists of an equilibration and production stage of $2 \times 10^7$ and $2.5 \times 10^7$ Monte Carlo steps (MCS), respectively. Snapshots (i.e., number of molecules, internal energy, and optionally the xyz coordinates) are stored every 200 MCS to reduce the correlation between sequential configurations. Thus, the number of snapshots $(K_{\rm snaps})$ for a single state point ($\mu$, $V$, $T$) is $1.25 \times 10^5$.


%is $2 \times 10^5$ for vapor simulations and $1.25 \times 10^5$ for liquid and ``bridge'' simulations. 

%The production stage is $4 \times 10^7$ MCS for vapor simulations and $2.5 \times 10^7$ MCS for the liquid and ``bridge'' simulations.

Displacement and rotation moves are required to thermally equilibrate the system at the simulation temperature while molecule insertion and deletion moves ensure that $\mu$ is equal to the prescribed value. Cyclohexane simulations also employ crank-shaft moves to sample internal configurations, which can be challenging for ring molecules \cite{Shah2011,Binder1979}. The type of Monte Carlo move implemented for each step is selected randomly. The displacement, rotation, and molecule swap move probabilities for branched alkanes are 30\%, 10\%, and 60\%, respectively. The move probabilities for cyclohexane are 30\%, 10\%, 40\%, and 20\% for displacement, rotation, molecule swap, and crank-shaft moves, respectively. 

All simulations utilize coupled-decoupled configurational-bias Monte Carlo (CBMC)\cite{Martin1999} to enhance the insertion acceptance rate, with 100 angle trials, 30 dihedral trials, 10 initial site trials, and 4 subsequent site trials. The move probabilities are consistent with those of Mick et al.~and Soroush Barhaghi et al.~\cite{Potoff_branched,Barhaghi2017}, while the CBMC dihedral trials and initial site trials differ slightly. Section \ref{SI sec: CBMC acceptance rates} provides an example of the CBMC acceptance rates for the different simulation state points.

Consistent with previous MiPPE studies \cite{Mie,Potoff_branched,Barhaghi2017}, we utilize a 1.0 nm non-bonded cut-off distance with analytical tail corrections for internal energy \cite{Allen1987}. Although TraPPE \cite{TraPPE,Martin1999,Keasler2012} and NERD \cite{NERD,Nath2001} were parameterized using a 1.4 nm and 1.38 nm cut-off, respectively, we compare our TraPPE and NERD validation results with those from Mick et al.~\cite{Potoff_branched}, which also utilizes a 1.0 nm cut-off.

All simulations are performed using GPU optimized Monte Carlo (GOMC) \cite{Nejahi2018} development version. Although GOMC is capable of implementing graphics processing units (GPUs), all simulations are performed with central processing units (CPUs) because of the relatively simple systems that are studied (i.e., small box sizes and no electrostatics). A description of the compiler and machine hardware is provided in Section \ref{SI sec: Machine hardware} of Supporting Information.

%, compiled using GCC with OpenMP enabled.

%CPU GCMC because all of the systmes simulated in this study are relatively simple (small molecules with no electrostatics)  systems simulated in this study, Only the MiPPE cyclohexane results are obtained with graphics processing unit (GPU) GCMC simulations on a INSERT MOHAMMAD'S MACHINE DESCRIPTION. All other systems are simulated using central processing units (CPUs) run on a Linux 4.4.0-112-generic x86\_64 on an Intel(R) Xeon(R) CPU E5-2699 v4 @ 2.20GHz machine. 

Initial configurations are generated with Packmol \cite{PACKMOL}, while psfgen is used to generate the coordinate (*.pdb) and connectivity (*.psf) files \cite{VMD}. Example GOMC input files with corresponding shell and Python scripts for preparing, running, and analyzing simulations are provided at https://github.com/ramess101/MBAR\_GCMC.

%, where a Monte Carlo cycle is defined as $N$ individual Monte Carlo steps

%\begin{enumerate}
%	\item Simulations performed by Mick et al. are reanalyzed using MBAR
%	\item Additional simulations are performed in GCMC ensemble using GPU optimized Monte Carlo (GOMC)
%	\item Simulation specifications, i.e., box size, number of steps, type of moves, etc.
%	\item State points (chemical potentials and temperatures) simulated are same as those utilized in Mick et al.
%\end{enumerate}

\subsection{Comparison between GCMC-MBAR and GCMC-HR} \label{sec: GCMC-HR and GCMC-MBAR}

%\begin{eqnarray} \label{HR prob}
%Pr(N,U) = \frac{\Omega (N,V,U) \exp (-\beta U + \beta \mu N)}{\Xi (\mu, V, \beta)}
%\end{eqnarray}
%
%****\textbf{MRSHIRTS}: I would like the MBAR and histogram reweighting equations to be consistent with each other in the limit of zero bin-width and a single reference force field. Here is how HR is presented in the literature (copied essentially verbatim):
%
%%MRS: note that f_i(N,E) is essentially a descritized density of states.
%The probability of observing $N$ particles with internal energy $U$ for a given chemical potential $(\mu)$ and inverse temperature $(\beta \equiv \frac{1}{k_{\rm B}T}$ where $k_{\rm B}$ is the Boltzmann constant) is
%%%% This is how the equation is presented in Pana2000, but I want to have similar terms between MBAR and HR
%\begin{eqnarray} \label{eq: HR prob lit}
%p(N,E;\mu,\beta) = \frac{\sum_{i=1}^{R} f_i(N,E)  \exp(-\beta E + \beta \mu N)}{\sum_{i=1}^{R} K_i \exp(-\beta_i E + \beta_i \mu_i N - C_i)}
%\end{eqnarray}
%where $f_i(N,E)$ is the probability of occurrence $N$ particles in the simulation cell with total configurational energy in the vicinity of $E$, $K_i$ is the total number of observations $(K_i = \sum_{N,E} f_i(N,E))$ for run $i$. The constants $C_i$ (also known as ``weights'') are obtained by iteration from the relationship
%\begin{eqnarray} \label{eq: Weights lit}
%\exp (C_i) = \sum_{E} \sum_{N} p(N,E;\mu_i,\beta_i)
%\end{eqnarray}
%%MRS: the C_i's are( minus) the free energies; the sum above is a descritized version of exp(-\Pi) = \int N \int E \Omega(E,N) exp(-beta E(X) + \beta \mu N) dE dN
%%RAM: I am considering redefining C_i as -C_i for simplicity
%%MRS: I don't think R is defined here.  It's the number of runs, each of which has a different $\mu$ and $\beta$, correct?
%%RAM: Yes, R is the number of runs.
%Given an initial guess for the set of weights $C_i$, Equations \ref{eq: HR prob lit} and \ref{eq: Weights lit} can be iterated until convergence. The ensemble average for a given observable is
%\begin{eqnarray}
%\langle O \rangle_{\mu,\beta} = \sum_{E} \sum_{N} p(N,E;\mu,\beta) \times O
%\end{eqnarray}
%The pressure of a system can be obtained from the following expression. If the conditions for run 1 are $(\mu_1, V, \beta_1)$ and for run 2 $(\mu_2, V, \beta_2)$, then
%\begin{eqnarray} \label{eq: press lit}
%C_2 - C_1 = \ln \frac{\Xi (\mu_2,V,\beta_2)}{\Xi (\mu_1,V,\beta_1)} = \beta_2 P_2 V - \beta_1 P_1 V
%\end{eqnarray}
%where $P$ is the pressure, since $\ln \Xi = \beta P V$. Equation \ref{eq: press lit} can be used to obtain the absolute value of the pressure for one of the two runs, provided that the absolute pressure can be estimated for the other run. Typically, this is done by performing simulations for low-density states for which the system follows the ideal gas equation of state, $P V = N k_{\rm B} T$.
%
%There are two states sampled by the run, one at low and one at high particle numbers, corresponding to the gas and liquid states. The conditions for phase coexistence are equality of temperature, chemical potential, and pressure - the first two are satisfied by construction. From Eq. 18, the integral under the probability distribution function is proportional to the pressure. In the case of two distinct phsaes, the integrals should be calculated separately under the liquid and gas peaks. The condition of equality of pressures can be satisfied by reweighting the data until this condition is met.
% 
%%\begin{eqnarray} \label{HR prob}
%%Pr(N,U|\mu,\beta) = \frac{\sum_{i=1}^{R} Pr_i(N,U)  \exp(-\beta U + \beta \mu N)}{\sum_{i=1}^{R} K_i \exp(-\beta_i U + \beta_i \mu_i N - C_i)}
%%\end{eqnarray}
%
%****\textbf{MRSHIRTS}: Here is how I reported the MBAR equations in our first publication (again this is copied essentially verbatim). Note that previously we used MBAR in the NVT ensemble and only weighted configurations from the same NVT run. In this paper we need to denote that different temperatures and chemical potentials are being weighted together.:
%
%%MRS: I think one thing that this presentation obscures that makes the comparison harder is the fact that for histograms, youtalk about simulations with different \betas or \mus, and in MBAR, you talk about it with different \thetas. 
%%Whereas for MBAR, you can do the analysis with different thetas AND/OR different \betas or \mus.   This is impossible with HR, because beta and mu only change the probability in the boltzman weight, so all data samples in the histogram N, E are reweighted the SAME.  Thus you can histogram them, and reweight them all together. This FAILS for changing theta, because there's essentially no way to histogram things for arbitrary theta (you would have to histogram by pairwise distance, which depends on pairs of particles, so - a mess).  
%%MRS: I am going to rewrite it below; you can adapt as you see fit. I can see a couple of different ways of writing it up. One might be to 1) make MBAR general to begin with (like I do here), with simulations over all mu and beta AND theta, or 2) to just write MBAR in terms of mu and beta, and THEN generalize to theta after proving the equivalence to HR for just beta and mu changes.  
%%RAM: I prefer option 1
%With MBAR the estimate of expectation $\langle O(\theta) \rangle$ of any given observable $O$ at an arbitrary set of force field parameters, temperatures, and chemical potentials $(\theta)$ of any given observable $(O)$:
%\begin{eqnarray} \label{MBAR O theta_lit}
%\langle O(\theta)\rangle = \sum_{n=1}^{N} O(\x_{n},N_n;\theta) W_{n}(\theta)
%\end{eqnarray}
%where $\x_n,N_n$ are configurations and number of particles sampled at $R$ (which can be one or more) simulation runs with reference conditions (i.e. force fields, temperatures and chemical potentials, designated by vector $\theta$). $O(\x_{n},N_n;\theta)$ is the observable value using force field, temperature, chemical potentials $\theta$ with configurations $\x_n$,$N_n$, and $W_{n}(\theta)$ is the weight of the $n^{th}$
%configuration using the vector $\theta$, calculated by using:
%\begin{eqnarray} \label{MBAR weights theta_lit}
%W_{n}(\theta) = \frac{\exp[\hat{f}(\theta)-u(\x_{n},N_n;\theta)]}{\sum\limits_{i=1}^R K_i \, \exp[\hat{f}(\theta_{\rm ref,i}) - u(\x_{n},N_n;\theta_{\rm ref,i})]}
%\end{eqnarray}
%where the reduced free energies $(\hat f(\theta))$ are calculated with:
%\begin{eqnarray} \label{MBAR free energy theta_lit}
%\hat f(\theta) &=& - \ln \sum_{n=1}^{N^{\rm tot}_{\rm snaps}}
%\frac{\exp[-u(\x_{n},N_n;\theta)]}{\sum\limits_{i=1}^R K_i \, \exp[\hat{f}(\theta_{\rm ref,i}) - u(\x_{n},N_n;\theta_{\rm ref,i})]} 
%\end{eqnarray}
%where $R$ is the number of runs at reference conditions, $K^{\rm tot}_{\rm snaps} = \sum_i K_i$ is the total number of snapshots for all $R$ reference conditions, $N_k$ are the total number of snapshots from the $i^{th}$ reference force field, $\theta_{\rm ref,i}$ is the $i^{th}$ reference (i.e. simulated) set of conditions, and $u(\x_{n};\theta) = \beta U(\x_n;\theta_\mathrm{parm})-\beta N_n \mu)$ is the reduced potential energy evaluated with $\theta$ for configuration $\x_n,N_n$, where $\theta_{\mathrm{parm}}$ are the components of theta that specifically involved the potential energy. 
%%MRS: not that there is also potentially an issue with factors relating indistinguishability of particles in the probability distribution. I think it ends up cancelling out for this particular problem, but maybe I need to go back and think about that some more. It doesn't appear in the histogram version?  Unless it's implicit?  Since I don't see it in the histogram. I will not include anything for now. 
%Note that $\hat f(\theta_{\rm ref,i})$ is required to evaluate the denominator of Equations \ref{MBAR weights theta_lit}-\ref{MBAR free energy theta_lit}. The values for $\hat f(\theta_{\rm ref,i})$ are obtained by solving a system of $R$ nonlinear equations for self-consistency, using a range of different methods. There is provably only one solution, so as long as certain criteria are met that will be discussed below, the only difference in solvers is efficiency and numerical stability.
%%MRS: note that it's technically a system of R-1 equations; one of the R is redundant. 
%
%****\textbf{MRSHIRTS}: This is my attempt at an internally consistent set of HR and MBAR equations:

%MRS: some comments that would be great to get your feedback on
%1) In the equations below, I assume R is the number of histogram bins that have nonzero occupancy?
%3) I think you should say from the start in this section that you are suppressing the theta dependence; HR has none, so you do't want to include that in the MBAR equations. So the indices i in both MBAR and HR should both be indexed by i. 

%RAM:
%1) R is the number of simulation runs
%3) Hopefully this is more clear now

%\begin{eqnarray} \label{HR prob}
%Pr(N,U|\mu,\beta) = \frac{\sum_{i=1}^{R} Pr_i(N,U)  \exp(-\beta U + \beta \mu N)}{\sum_{i=1}^{R} K_i \exp(-\beta_i U + \beta_i \mu_i N - C_i)}
%\end{eqnarray}

Converting the GCMC simulation output into vapor-liquid phase equilibria properties requires significant post-processing through reweighting. Histogram reweighting (HR) and, more generally, configuration reweighting is an important tool in many fields of molecular simulation.\cite{Ferrenberg1988,Ferrenberg1989} In fact, it has long since been known that it is possible to estimate properties for state $j$ by reweighting configurations that were sampled with state $i$. \cite{McDonald1967,Card1970,Wood1968,Pana2000} 

For example, umbrella sampling simulations are often processed using the weighted histogram analysis method (WHAM) to compute free energy differences between states \cite{Kumar1992}. WHAM (or HR) is essentially an approximation of MBAR and, therefore, MBAR should be favored for free energy calculations whenever a histogram-free approach is feasible \cite{Souaille2001,Matos2017}. In this study, we implement MBAR in Python 2.7 through the \textit{pymbar} package available at https://github.com/choderalab/pymbar. 

Before demonstrating how to compute vapor-liquid phase equilibria with GCMC-MBAR in Sections \ref{sec: MBAR} and \ref{sec: Saturation}, we review the traditional GCMC-HR approach in Section \ref{sec: HR}. We also discuss the steps of this procedure that are the same for both GCMC-HR and GCMC-MBAR in Section \ref{sec: HR and MBAR}. We refer the interested reader to the literature for derivations and more detailed discussion of the GCMC-HR equations (cf. Reference \citenum{Pana2000}).

%Reweighting is a powerful tool when estimating phase coexistence from GCMC simulations. requires significant post-processing. Before demonstrating how we implement MBAR for this purpose, we review the traditional histogram reweighting (HR) approach. We also discuss the steps of this procedure that are the same for both HR and MBAR. We refer the interested reader to the literature for derivations and more detailed discussion of the GCMC-HR equations (cf. \citenum{Pana2000}).

\subsubsection{Histogram reweighting} \label{sec: HR}


%MRS3: one notation thing with the below: it's not explicitly clear that O is a function of N and U.  But it's not an explicit function of them, since the same N and I can have different values of O.  So I'm not quite sure how to denote it.  But it's definitely not a constant, which is not really clear in the notation in \ref{eq: HR ave}.
%MRS3: fixed spelling on discritized to discretized.  Check other spelling if this one got through.
%MRS3: added below to clarify.
Histogram reweighting computes the ensemble average of a given observable $(O$, e.g., $U$ and $\rho)$ according to
\begin{eqnarray} \label{eq: HR ave}
\langle O(\mu,\beta) \rangle = \sum_{U} \sum_{N} O \times Pr(N,U;\mu,\beta)
\end{eqnarray}
where $\langle \dots \rangle$ denotes an ensemble average and $Pr(N,U;\mu,\beta)$ is the probability of observing $N$ molecules with internal energy $U$ for a given chemical potential $(\mu)$ and inverse temperature $(\beta \equiv \frac{1}{k_{\rm B}T}$, where $k_{\rm B}$ is the Boltzmann constant). The double summation is computed numerically where $U$ and $N$ are discretized into a 2-dimensional histogram. The probability is obtained with HR from\cite{Ferrenberg1988,Ferrenberg1989}
\begin{eqnarray} \label{eq: HR prob}
Pr(N,U;\mu,\beta) = \frac{\sum_{i=1}^{R} K_{\mathrm{snaps},i}(N,U)  \exp(-\beta U + \beta \mu N)}{\sum_{i=1}^{R} K_{\mathrm{snaps}, i} \exp(-\beta_i U + \beta_i \mu_i N + \hat f_i)}
\end{eqnarray}
where $R$ is the number of runs, $K_{\mathrm{snaps},i}(N,U)$ is the number of (uncorrelated) configuration ``snapshots'' in the $i^{\rm th}$ run (with $\beta_i$ and $\mu_i$) that have $N$ molecules and $U$ within the histogram bin width, $K_{\mathrm{snaps}, i}$ is the total number of snapshots for run $i$ (i.e., $K_{\mathrm{snaps}, i} = \sum_{N,U} K_{\mathrm{snaps},i}(N,U)$), and $\hat f_i$ is an estimate for the reduced free energy, which is calculated with the relationship
\begin{eqnarray} \label{eq: Weights}
\hat f(\mu,\beta) = - \ln \sum_{U} \sum_{N} \frac{\sum_{i=1}^{R} K_{\mathrm{snaps},i}(N,U) \exp(-\beta U + \beta \mu N)}{\sum_{i=1}^{R} K_{\mathrm{snaps}, i} \exp(-\beta_i U + \beta_i \mu_i N + \hat f_i)}
\end{eqnarray}
where $\hat f_i \equiv \hat f(\mu_i,\beta_i)$. Note that because $\hat f_i$ can also be viewed simply as a constant that solves the self-consistent equations, the GCMC-HR literature \cite{Pana2000} typically adopts the notation $C_i$ (or more specifically, $-C_i$) instead of $\hat f_i$. We prefer $\hat f_i$ for a clear comparison with the MBAR expressions that follow.

%%% I believe this was incorrect referring to Pr_i(N,U) as a probability and not the histogram count
%\begin{eqnarray} \label{eq: HR prob}
%Pr(N,U;\mu,\beta) = \frac{\sum_{i=1}^{R} Pr_i(N,U)  \exp(-\beta U + \beta \mu N)}{\sum_{i=1}^{R} K_{\mathrm{snaps}, i} \exp(-\beta_i U + \beta_i \mu_i N + \hat f_i)}
%\end{eqnarray}
%where $R$ is the number of runs, $Pr_i(N,U)$ is the probability of observing $N$ particles and $U$ within the histogram bin width for the $i^{\rm th}$ run with $\beta_i$ and $\mu_i$, $K_{\mathrm{snaps}, i}$ is the number of ``snapshots'' for run $i$ (i.e., $K_{\mathrm{snaps}, i} = \sum_{N,U} Pr_i(N,U)$), and $\hat f_i$ is an estimate for the reduced free energy, which is calculated with the relationship
%\begin{eqnarray} \label{eq: Weights}
%\hat f(\mu,\beta) = - \ln \sum_{U} \sum_{N} \frac{\sum_{i=1}^{R} Pr_i(N,U) \exp(-\beta U + \beta \mu N)}{\sum_{i=1}^{R} K_{\mathrm{snaps}, i} \exp(-\beta_i U + \beta_i \mu_i N + \hat f_i)}
%\end{eqnarray}
%where $\hat f_i \equiv \hat f(\mu_i,\beta_i)$. Note that because $\hat f_i$ can also be viewed simply as a constant
%%MRS3: added below to clarify.
%that solves the self-consistent equations,
%the GCMC-HR literature \cite{Pana2000} typically adopts the notation $C_i$ (or more specifically, $-C_i$) instead of $\hat f_i$. We prefer $\hat f_i$ for a clear comparison with the MBAR expressions that follow.

%MRS3: above: the i should be listed included in the definition.
%RAM: Done
%MRS3: is this correct above? The number of observations is a sum over probabilities?  Seems either wrong, or a confusing way to describe things. 
%RAM: I have had some concerns about this too. This is the notation used in the literature but it is certainly confusing. I think it is because they are defining Pr_i(N,U) as the count in the bins rather than the probability?

%%% Old version with C_i 
%\begin{eqnarray} \label{eq: HR prob}
%Pr(N,U;\mu,\beta) = \frac{\sum_{i=1}^{R} Pr_i(N,U)  \exp(-\beta U + \beta \mu N)}{\sum_{i=1}^{R} K_i \exp(-\beta_i U + \beta_i \mu_i N + C_i)}
%\end{eqnarray}
%where $Pr_i(N,U)$ is the probability of observing $N$ particles and $U$ within the histogram bin width, $R$ is the number of runs (where the i$^{\rm th}$ run corresponds to $\beta_i$ and $\mu_i$), $K_{i}$ is the number of observations (``snapshots'') for run $i$ (i.e., $K_{i} = \sum_{N,U} Pr_i(N,U)$), and $C_i$ are the ``constants'' that are calculated with the relationship
%\begin{eqnarray} \label{eq: Weights}
%C(\mu,\beta) = - \ln \sum_{U} \sum_{N} \frac{\sum_{i=1}^{R} Pr_i(N,U) \exp(-\beta U + \beta \mu N)}{\sum_{i=1}^{R} K_i \exp(-\beta_i U + \beta_i \mu_i N + C_i)}
%\end{eqnarray}
%where $C_i \equiv C(\mu_i,\beta_i)$.
% The constants $C_i$ are obtained by iteration from the relationship
%\begin{eqnarray} \label{eq: Weights}
%C_i = - \ln \sum_{U} \sum_{N} Pr(N,U;\mu_i,\beta_i)
%\end{eqnarray}

\subsubsection{Histogram-free reweighting} \label{sec: MBAR}

Equations \ref{eq: HR ave}, \ref{eq: HR prob}, and \ref{eq: Weights} only allow for reweighting simulations at a different $\beta$ and $\mu$. By contrast, MBAR can also be applied to reweight simulations for different force field parameters $(\theta)$. The analogous MBAR equation to Equation \ref{eq: HR ave} is
\begin{eqnarray} \label{eq: MBAR ave}
\langle O(\theta,\mu,\beta)\rangle = \sum_{n=1}^{K^{\rm tot}_{\rm snaps}} O(\x_{n},N_n;\theta,\mu,\beta) \times W_{n}(\theta,\mu,\beta)
\end{eqnarray}
where $(\x_n,N_n)$ are \textit{uncorrelated} configurations sampled from $i=1 \ldots R$ simulations at inverse temperature $(\beta_{i})$, chemical potential $(\mu_{i})$, and reference force field parameters $(\theta_{\mathrm{ref},i})$, and $K^{\rm tot}_{\rm snaps} \equiv \sum_{i=1}^R K_{\mathrm{snaps}, i}$ is the total number of snapshots for all $R$ runs. $W_{n}(\theta,\beta,\mu)$ is the weight of the $n^{\rm th}$ configuration for an arbitrary $\mu$, $\beta$, and $\theta$. $W_{n}$ is computed with the following expression (analogous to Equation \ref{eq: HR prob})
%The analogous MBAR equation to Equation \ref{eq: HR prob} is 
\begin{eqnarray} \label{eq: MBAR weights}
W_{n}(\theta,\beta,\mu) = \frac{\exp[\hat{f}(\theta,\beta,\mu)-u(\x_{n},N_n;\theta,\beta,\mu)]}{\sum\limits_{i=1}^R K_{\mathrm{snaps}, i} \, \exp[\hat{f}(\theta_i,\beta_{i},\mu_{i}) - u(\x_{n},N_n;\theta_i,\beta_{i},\mu_{i})]}
\end{eqnarray}
%where $\x_n,N_n$ are configurations sampled from $i=1 \ldots R$ simulations at inverse temperature $(\beta_{i})$, chemical potential $(\mu_{i})$, and force field parameters $(\theta_i)$. $W_{n}(\theta,\beta,\mu)$ is the weight of the $n^{\rm th}$ configuration in a simulation with arbitrary $\mu$, $\beta$, and $\theta$.
%$K_i$ are the total number of snapshots from the $i^{\rm th}$ run, 
%$\hat f(\theta,\beta,\mu)$ is the reduced free energy, and
where $u(\x_{n},N_n;\theta,\beta,\mu)$ is the reduced potential energy evaluated with $\theta$, $\beta$, and $\mu$ for configuration $(\x_n,N_n)$. The reduced free energy is computed with an expression analogous to Equation \ref{eq: Weights}
%The analogous MBAR equation to Equation \ref{eq: Weights} is
\begin{eqnarray} \label{eq: MBAR free energy}
\hat f(\theta,\beta,\mu) &=& - \ln \sum_{n=1}^{K^{\rm tot}_{\rm snaps}}
\frac{\exp[-u(\x_{n},N_n;\theta,\beta,\mu)]}{\sum\limits_{i=1}^R K_{\mathrm{snaps}, i} \, \exp[\hat{f}(\theta_i,\beta_{i},\mu_{i}) - u(\x_{n},N_n;\theta_i,\beta_{i},\mu_{i})]}
\end{eqnarray}
For the grand canonical ensemble, the reduced potential energy is
\begin{equation} \label{eq: reduced potential}
u(\x_{n},N_n;\theta,\beta,\mu) = \beta U(\x_n,N_n;\theta) - \beta \mu N_n
\end{equation} 

\subsubsection{Comparison between HR and MBAR} \label{sec: HR and MBAR}

The similarities between MBAR (Equations \ref{eq: MBAR weights} and \ref{eq: MBAR free energy}) and HR (Equations \ref{eq: HR prob} and \ref{eq: Weights}) are readily apparent after substituting Equation \ref{eq: reduced potential} into Equations \ref{eq: MBAR weights} and \ref{eq: MBAR free energy}. Indeed, the difference between HR and MBAR is primarily that of bookkeeping, although the histogram-free approach of MBAR does have some benefits when varying force field parameters, as discussed below. 

%$K^{\rm tot}_{\rm snaps}$ is the total number of observations (``snapshots'')
%where $\hat f(\theta,\beta,\mu)$ is the reduced free energy, $K_i$ are the total number of snapshots from the $i^{\rm th}$ run, $K^{\rm tot}_{\rm snaps} = \sum_i K_i$ is the total number of snapshots for all $R$ runs, and $u(\x_{n},N_n;\theta,\beta,\mu)$ is the reduced potential energy evaluated with $\theta$, $\beta$, and $\mu$ for configuration $\x_n,N_n$. For the grand canonical ensemble, $u(\x_{n},N_n;\theta,\beta,\mu) = \beta U(\x_n;\theta) - \beta \mu N_n$.

%%MRS: not that there is also potentially an issue with factors relating indistinguishability of particles in the probability distribution. I think it ends up cancelling out for this particular problem, but maybe I need to go back and think about that some more. It doesn't appear in the histogram version?  Unless it's implicit?  Since I don't see it in the histogram. I will not include anything for now.

%The values for $C_i$ and $\hat f$ are obtained by solving a system of $R-1$ nonlinear equations for self-consistency, using a range of different methods. There is provably only one solution, so as long as certain criteria are met that will be discussed below, the only difference in solvers is efficiency and numerical stability.
%%MRS: note that it's technically a system of R-1 equations; one of the R is redundant. 

%%MRS: added this qualitative description of the reduction. 
%These two sets of equations (HR: Equations \ref{eq: HR ave}, \ref{eq: HR prob}, and \ref{eq: Weights}, MBAR: Equations \ref{eq: MBAR ave}, \ref{eq: MBAR weights}, and \ref{eq: MBAR free energy}) can be seen as equivalent for a single force field and in the limit of infinitesimal histogram bin widths. In other words, if all $R$ simulations are performed using a single reference force field $(\theta_{\rm ref})$ and $W_{n}$ and $\hat f$ are computed with $\theta = \theta_i = \theta_{\rm ref}$, the $\theta$ dependence of Equations \ref{eq: MBAR ave}, \ref{eq: MBAR weights}, and \ref{eq: MBAR free energy} is removed. Furthermore, in the zero bin width limit no histogram contains more than 1 snapshot. In that case, $U$ and $N$ for each histogram can be taken to be the $U(x_n)$ and $N_n$ of the single observation in that histogram, and histograms with no particles can be omitted. We then observe that $Pr_i(N,U)$ is either 1 or 0, and the sum over all histograms becomes a sum over snapshots conducted in all $R$ simulation runs. Equation \ref{eq: Weights} reduces in this approximation to Equation \ref{eq: MBAR free energy}, and Equation \ref{eq: HR prob} reduces to Equation \ref{eq: MBAR weights}.
%
%%The first assumption is that all $R$ simulations are performed using a single reference force field $(\theta_{\rm ref})$ and $W_{n}$ and $\hat f$ are computed with $\theta = \theta_i = \theta_{\rm ref}$. Second,
%
%However, a key advantage of MBAR over HR is that by changing from a sum over histograms to a sum over snapshots, we are free to perform simulations with other conditions besides $\mu$ and $\beta$. For example, we can carry out simulations at different force field parameters $\theta$. 
%MRS3: adding additional information.
In the histogram context, snapshots with similar $U(\theta_{\rm ref})$ would not necessarily belong in the same $U$ histogram bin when recomputed with a different force field. Similarly, we cannot easily separate out snapshots in the same $U$ histogram that were performed with multiple $\theta_{\rm ref}$. However, if we perform sums over snapshots, we can carry out simulations at different force field parameters $(\theta_{\mathrm{ref},i})$. Furthermore, we can reevaluate the configurational energy $(U(\x_{n},N_n;\theta_{\rm rr}))$ with a range of different parameters for the relatively small expense of rerunning only a subset of uncorrelated snapshots with $\theta_{\rm rr}$.

%The key advantage of MBAR over HR is that by changing from a sum over histograms to a sum over snapshots, we are free to perform simulations with other conditions besides $\mu$ and $\beta$. For example, we can carry out simulations at different force field parameters $(\theta_{\mathrm{ref},i})$. In the histogram context, we cannot easily separate out snapshots in the same $U$ histogram that were performed with different force field parameters. Furthermore, as MBAR performs a sum over snapshots, we can reevaluate the configurational energy, $U(\x_{n},N_n;\theta_{\rm rr})$, with a range of different ``rerun'' parameter sets $(\theta_{\rm rr})$.
% range of different parameters for a relatively small expense.

If, however, all $R$ simulations are performed with a single reference force field $(\theta_{\rm ref})$ and $\theta_{\rm rr} = \theta_{\rm ref}$, these two sets of equations (HR: Equations \ref{eq: HR ave}, \ref{eq: HR prob}, and \ref{eq: Weights}, MBAR: Equations \ref{eq: MBAR ave}, \ref{eq: MBAR weights}, and \ref{eq: MBAR free energy}) can be seen as equivalent in the limit of infinitesimal histogram bin widths. In the zero bin width limit, no histogram contains more than 1 snapshot and, therefore, $U$ and $N$ for each histogram can be taken to be the $U(\x_n,N_n)$ and $N_n$ of the single observation in that histogram, while histograms with no snapshots can be omitted. Thus, $K_{\mathrm{snaps},i}(N,U)$ is either 1 or 0, and the sum over all histograms becomes a sum over snapshots conducted in all $R$ simulation runs. Equations \ref{eq: HR prob} and \ref{eq: Weights} then reduce to Equations \ref{eq: MBAR weights} and \ref{eq: MBAR free energy}, respectively.

%MRS: added this qualitative description of the reduction. 
%The $\theta$ dependence of Equations \ref{eq: MBAR ave}, \ref{eq: MBAR weights}, and \ref{eq: MBAR free energy} can be removed by performing all $R$ simulations with a single reference force field $(\theta_{\rm ref})$ and only computing $W_{n}$ and $\hat f$ for $\theta = \theta_i = \theta_{\rm ref}$. With this simplification, these two sets of equations (HR: Equations \ref{eq: HR ave}, \ref{eq: HR prob}, and \ref{eq: Weights}, MBAR: Equations \ref{eq: MBAR ave}, \ref{eq: MBAR weights}, and \ref{eq: MBAR free energy}) can be seen as equivalent in the limit of infinitesimal histogram bin widths. In the zero bin width limit, no histogram contains more than 1 snapshot and, therefore, $U$ and $N$ for each histogram can be taken to be the $U(\x_n,N_n)$ and $N_n$ of the single observation in that histogram, while histograms with no snapshots can be omitted. Thus, $Pr_i(N,U)$ is either 1 or 0, and the sum over all histograms becomes a sum over snapshots conducted in all $R$ simulation runs. Equations \ref{eq: HR prob} and \ref{eq: Weights} then reduce to Equations \ref{eq: MBAR weights} and \ref{eq: MBAR free energy}, respectively.

%For both HR and MBAR, the pressure equation is derived from $\ln \Xi = \beta P V$, where $\Xi$ is the grand partition function. The HR expression is
%\begin{eqnarray} \label{eq: HR press}
%C_1 - C_2 = \ln \frac{\Xi (\mu_2,V,\beta_2)}{\Xi (\mu_1,V,\beta_1)} = \beta_2 P_2 V - \beta_1 P_1 V
%\end{eqnarray}
%where $P$ is the pressure, and runs 1 and 2 are performed with $(\mu_1, V, \beta_1)$ and $(\mu_2, V, \beta_2)$, respectively. The analogous MBAR equation is
%\begin{eqnarray} \label{eq: MBAR press}
%\hat f_1 - \hat f_2 = \ln \frac{\Xi (\theta_2,\mu_2,V,\beta_2)}{\Xi (\theta_1,\mu_1,V,\beta_1)} = \beta_2 P_2 V - \beta_1 P_1 V
%\end{eqnarray}
%where runs 1 and 2 can also be performed using different force field parameters, $\theta_1$ and $\theta_2$, respectively. Computing the absolute pressure of $P_1$ with Equations \ref{eq: press} and \ref{eq: MBAR press} requires a reference pressure $(P_2)$, which is determined at a low-density where the ideal gas equation of state, $P V = N k_{\rm B} T$, is assumed to be accurate.

%For both HR and MBAR, the pressure is computed from
%\begin{equation} \label{eq: press}
%P = \frac{k_{\rm B} T}{V} \ln \Xi + B
%\end{equation}
%where $\Xi$ is the grand partition function and $B$ is an additive constant. For both HR and MBAR, $B$ is determined by fitting a straight line to $\ln \Xi$ with respect to $N$ at very low densities. At these low densities, the system is assumed to behave as an ideal gas and, therefore, the slope is unity and $B = \frac{k_{\rm B} T}{V} \times b$, where $b$ is the y-intercept from the straight line regression.
%
%For HR $\ln \Xi$ is proportional to $C$ while for MBAR $\ln \Xi$ is proportional to $\hat f$. Therefore, the HR pressure expression is
%\begin{eqnarray} \label{eq: HR press alt}
%P(\mu_i,\beta_i) = C(\mu_i,\beta_i) + B
%\end{eqnarray}
%while the MBAR pressure expression is
%\begin{eqnarray} \label{eq: MBAR press alt}
%P(\theta,\mu_i,\beta_i) = \hat f(\theta_i,\mu_i,\beta_i) + B
%\end{eqnarray}

Both HR and MBAR require solving a system of $R-1$ nonlinear equations for self-consistency (Equations \ref{eq: HR prob} and \ref{eq: Weights} for HR and Equations \ref{eq: MBAR weights} and \ref{eq: MBAR free energy} for MBAR). Specifically, initial guesses for $\hat f_i$ are updated iteratively until convergence. There is provably only one solution \cite{shirts-chodera:jcp:2008:mbar}, so as long as certain criteria are met that will be discussed below. Thus, although a range of different solver methods exist, the only difference is efficiency and numerical stability.
%MRS2: cite the 2008 paper on provably one solution.
%RAM: Done
%MRSHIRTS: I am not sure that we discuss the ``certain criteria.'' Could you elaborate on this some more?

\subsubsection{Computing phase equilibria} \label{sec: Saturation}

%MRS: I'm pretty sure its ``straight line'' not ``straight-line''
%RAM: Yes I think you are correct. It is certainly not "straight-line" as a noun, but I thought it would be "straight-line regression" as an adjective
For both GCMC-HR and GCMC-MBAR, the pressure is computed from
\begin{equation} \label{eq: press}
P(\theta,\beta,\mu) = \frac{k_{\rm B} T}{V} \ln \Xi + B = \hat f(\theta,\beta,\mu) + B
\end{equation}
where $\Xi$ is the grand canonical partition function and $B$ is an additive constant, which is determined by fitting a straight line to $\ln \Xi$ with respect to $N$ at very low densities. At these low densities, the system is assumed to behave as an ideal gas and, therefore, the slope is unity and $B = \frac{k_{\rm B} T}{V} \times b$, where $b$ is the y-intercept from the straight line regression.

Having determined $B$, the saturated vapor pressure $(P_{\rm vap}^{\rm sat})$ is computed with Equation \ref{eq: press} at the desired saturation temperature $(T^{\rm sat})$ and corresponding saturation chemical potential $(\mu^{\rm sat})$. $\mu^{\rm sat}$ is obtained by equating the pressures in the vapor and liquid phases at a fixed value of $T^{\rm sat}$. This is done by integrating $Pr$ (HR) or $W_{n}$ (MBAR) for the two phases separately, i.e., by dividing the snapshots into low and high density regimes. For example, the equality of pressures is satisfied for HR when
\begin{eqnarray} \label{eq: HR equal press}
\sum_{U} \sum_{N > N_{\rm c}} Pr(N,U;\mu^{\rm sat},\beta^{\rm sat}) = \sum_{U} \sum_{N \leq N_{\rm c}} Pr(N,U;\mu^{\rm sat},\beta^{\rm sat})
\end{eqnarray}
where $N_{\rm c}$ is an estimate for the number of molecules at the critical density, which serves to distinguish between snapshots that are in the vapor or liquid phases. The analogous MBAR equation is
\begin{eqnarray} \label{eq: MBAR equal press}
\sum_{n=1}^{K^{\rm liq}_{\rm snaps}} W_{n}(\theta,\beta^{\rm sat},\mu^{\rm sat}) = \sum_{n=1}^{K^{\rm vap}_{\rm snaps}} W_{n}(\theta,\beta^{\rm sat},\mu^{\rm sat})
\end{eqnarray}
where $K^{\rm liq}_{\rm snaps}$ and $K^{\rm vap}_{\rm snaps}$ are the number of liquid and vapor snapshots, respectively.

By solving Equations \ref{eq: HR equal press} (HR) or \ref{eq: MBAR equal press} (MBAR) for $\mu^{\rm sat}$, the vapor and liquid saturation densities and energies ($\rho_{\rm liq}^{\rm sat}$, $\rho_{\rm vap}^{\rm sat}$, $U_{\rm liq}^{\rm sat}$, and $U_{\rm vap}^{\rm sat}$) are also computed with a modified version of Equations \ref{eq: HR ave} (HR) or \ref{eq: MBAR ave} (MBAR), where only snapshots from the desired phase are included in the weighted average. For example, when computing $\rho_{\rm liq}^{\rm sat}$ and $U_{\rm liq}^{\rm sat}$, the double summation in Equation \ref{eq: HR ave} (HR) is performed only for $N > N_{\rm c}$ and the sum in Equation \ref{eq: MBAR ave} (MBAR) is only over $K^{\rm liq}_{\rm snaps}$ liquid snapshots.

%The saturated vapor pressure $(P_{\rm vap}^{\rm sat})$ is computed with Equations \ref{eq: HR press alt} (HR) or \ref{eq: MBAR press alt} (MBAR) at the saturation temperature $(T^{\rm sat})$ and saturation chemical potential $(\mu^{\rm sat})$, where $\mu^{\rm sat}$ is determined by equating the pressures in the vapor and liquid phases at a fixed value of $T^{\rm sat}$. The two phases are integrated separately by dividing the snapshots into low and high density regimes. For example, the equality of pressures is satisfied for HR when
%\begin{eqnarray} \label{eq: HR equal press}
%\sum_{U} \sum_{N > N_{\rm c}} Pr(N,U;\mu^{\rm sat},\beta^{\rm sat}) = \sum_{U} \sum_{N \leq N_{\rm c}} Pr(N,U;\mu^{\rm sat},\beta^{\rm sat})
%\end{eqnarray}
%where $N_{\rm c}$ is an estimate for the number of molecules at the critical density, which serves to distinguish between snapshots that are in the vapor or liquid phases. The analogous MBAR equation is
%\begin{eqnarray} \label{eq: MBAR equal press}
%\sum_{n=1}^{K^{\rm liq}_{\rm snaps}} W_{n}(\theta,\beta,\mu) = \sum_{n=1}^{K^{\rm vap}_{\rm snaps}} W_{n}(\theta,\beta,\mu)
%\end{eqnarray}
%where $N_n$ is the number of molecules in the $n^{\rm th}$ snapshot and $K^{\rm liq}_{\rm snaps}$ and $K^{\rm vap}_{\rm snaps}$ are the number of liquid and vapor snapshots, respectively.
%
%By solving Equations \ref{eq: HR equal press} (HR) or \ref{eq: MBAR equal press} (MBAR) for $\mu^{\rm sat}$, the vapor and liquid saturation densities and energies ($\rho_{\rm liq}^{\rm sat}$, $\rho_{\rm vap}^{\rm sat}$, $U_{\rm liq}^{\rm sat}$, and $U_{\rm vap}^{\rm sat}$) are also computed with a modified version of Equations \ref{eq: HR ave} (HR) or \ref{eq: MBAR ave} (MBAR) where only snapshots from the desired phase are included in the weighted average. For example, when computing $\rho_{\rm liq}^{\rm sat}$ and $U_{\rm liq}^{\rm sat}$, the double summation in Equation \ref{eq: HR ave} (HR) is performed only for $N > N_{\rm c}$ and the sum in Equation \ref{eq: MBAR ave} (MBAR) is only over $K^{\rm liq}_{\rm snaps}$ liquid snapshots.

%\begin{eqnarray} \label{eq: MBAR equal press}
%\sum_{n=1}^{K^{\rm tot}_{\rm snaps}} W_{n}(\theta,\beta,\mu) h_{\rm vap}(N_n) = \sum_{n=1}^{K^{\rm tot}_{\rm snaps}} W_{n}(\theta,\beta,\mu) h_{\rm liq}(N_n)
%\end{eqnarray}
%where $N_n$ is the number of molecules in the $n^{\rm th}$ snapshot and $h_{\rm vap}$ and $h_{\rm liq}$ are the Heaviside step functions for the vapor and liquid phases, respectively, such that
%\begin{eqnarray}
%h_{\rm vap}(N) =
% \begin{cases}
% 1 & N \leq N_{\rm c} \\
% 0 & N > N_{\rm c} 
% \end{cases}
%\end{eqnarray}
%\begin{eqnarray}
%h_{\rm liq}(N) =
%\begin{cases}
%0 & N \leq N_{\rm c} \\
%1 & N > N_{\rm c}
%\end{cases}
%\end{eqnarray}
%%In addition to computing $P_{\rm vap}^{\rm sat}$, b
%By solving Equations \ref{eq: HR equal press} (HR) or \ref{eq: MBAR equal press} (MBAR) for $\mu^{\rm sat}$, the vapor and liquid saturation densities and energies ($\rho_{\rm liq}^{\rm sat}$, $\rho_{\rm vap}^{\rm sat}$, $U_{\rm liq}^{\rm sat}$, and $U_{\rm vap}^{\rm sat}$) are also computed with a modified version of Equations \ref{eq: HR ave} (HR) or \ref{eq: MBAR ave} (MBAR) where only snapshots from the desired phase are included in the weighted average. For example, when computing $\rho_{\rm liq}^{\rm sat}$ and $U_{\rm liq}^{\rm sat}$, the double summation in Equation \ref{eq: HR ave} (HR) is performed only for $N > N_{\rm c}$ and the summand of Equation \ref{eq: MBAR ave} (MBAR) is multiplied by $h_{\rm liq}(N_n)$. 

Having computed the pressure, internal energies, and densities for the saturated vapor and saturated liquid, the enthalpy of vaporization is calculated with
\begin{equation}
\Delta H_{\rm v} = \bar U_{\rm vap}^{\rm sat} - \bar U_{\rm liq}^{\rm sat} + P_{\rm vap}^{\rm sat} (\bar V_{\rm vap}^{\rm sat} - \bar V_{\rm liq}^{\rm sat})
\end{equation}
where $\bar U$ and $\bar V$ denote molar energy and molar volume, respectively.

\subsubsection{Uncertainties in phase equilibria} \label{sec: Uncertainties}

All uncertainties in this study are reported at the 95\% confidence level. Unless otherwise stated, uncertainties for $\rho_{\rm liq}^{\rm sat}$, $\rho_{\rm vap}^{\rm sat}$, $P_{\rm vap}^{\rm sat}$, $\Delta H_{\rm v}$, and $Z_{\rm vap}^{\rm sat}$ are determined with the following bootstrap re-sampling analysis: \cite{Efron1979}

%Specifically, the MBAR post-processing procedure is repeated for 100 random subsamples of the $K^{\rm tot}_{\rm snaps}$ snapshots from a single set of GCMC simulations. 

\begin{enumerate}
	\item Perform set of GCMC simulations with single replicate at each state point $(\mu$, $V$, $T$) \label{item:Simulations}
	\item Randomly select (with replacement) a subset of the $K^{\rm tot}_{\rm snaps}$ snapshots from Step \ref{item:Simulations} \label{item:Random selection}
	\item Compute phase equilibria for random subset of Step \ref{item:Random selection} following procedure outlined in Sections \ref{sec: MBAR} and \ref{sec: Saturation} \label{item:MBAR step}
	% Follow MBAR post-processing procedure of Sections \ref{sec: MBAR} and \ref{sec: Saturation} \label{item:MBAR step} 
	\item Repeat Steps \ref{item:Random selection} and \ref{item:MBAR step} $N_{\rm sets}$ times, where $N_{\rm sets} = 100$ in this study \label{item:Loop}
	\item Generate distribution from $N_{\rm sets}$ values of $\rho_{\rm liq}^{\rm sat}$, $\rho_{\rm vap}^{\rm sat}$, $P_{\rm vap}^{\rm sat}$, $\Delta H_{\rm v}$, and $Z_{\rm vap}^{\rm sat}$
	\item Determine interval that excludes the lowest and highest $0.025 \times N_{\rm sets}$ values of distribution
\end{enumerate}
Note that it is important to compute the uncertainties for $\Delta H_{\rm v}$ and $Z_{\rm vap}^{\rm sat}$ directly, rather than applying standard propagation of errors methods that typically neglect the correlation between properties\cite{Mess}.


%$P_{\rm vap}^{\rm sat}$ can be computed with Equations \ref{eq: HR press alt} (HR) or \ref{eq: MBAR press alt} (MBAR).

%After determining the saturation chemical potential $(\mu_{\rm sat})$, the vapor and liquid properties are computed with Equations \ref{eq: HR ave} (HR) or \ref{eq: MBAR ave} (MBAR) by separating the histograms or configurations into ``vapor'' and ``liquid'' phases. Typically this designation is based on $U$ and/or $N$. For example, when computing $\rho_{\rm liq}^{\rm sat}$ and $U_{\rm liq}^{\rm sat}$ with HR, the double summation in Equation \ref{eq: HR ave} is performed for $N > N_{\rm c}$, where $N_{\rm c}$ is an estimate for the critical number of molecules.

\subsubsection{Number of effective snapshots} \label{sec: Keff}

The performance of HR and MBAR depends primarily on good phase space overlap. For HR, good overlap means that the different sets of simulated $T$ and $\mu$ sample configurations and densities that are representative of the vapor and liquid phases at $T^{\rm sat}$ and $\mu^{\rm sat}$. For MBAR, an additional requirement is that the configurations sampled with $\theta_{\rm ref}$ also represent feasible configurations for $\theta_{\rm rr}$ \cite{naden:jctc:2016,Postdoc_1}. The amount of overlap can be quantified by the number of effective snapshots $(K^{\rm eff}_{\rm snaps})$ \cite{Dybeck2016}, using Kish's formula:
\begin{eqnarray} \label{Neff_lit}
K^{\rm eff}_{\rm snaps} = \frac{\left(\sum_n W_n\right)^2}{\sum_n W_n^2}
\end{eqnarray}
which reduces to $K^{\rm eff}_{\rm snaps}  = (\sum_n W_n^2)^{-1}$ when the weights are
normalized. This has the property that when the weights are equal,
$K^{\rm eff}_{\rm snaps}  = K^{\rm tot}_{\rm snaps} $, when all but one weight is negligible, $K^{\rm eff}_{\rm snaps}  \approx
1$, and behaves appropriately for intermediate cases. Messerly et al.~\cite{Postdoc_1} propose a heuristic that MBAR estimates are sufficiently reliable to compute phase equilibria with ITIC when $K^{\rm eff}_{\rm snaps} > 50$. Section \ref{sec: litFF} tests whether this is a reasonable heuristic for GCMC-MBAR as well. 
%MRS3: was in previous paper, so I think should be ``proposed''.  Also, I think it's that the uncertainty estimates are reliable at that point?  Whether it's reliable enough depends on the needed precision.  There perhaps needs to be a bit more attention on the uncertainty; you can't just see if K_eff > 50, you also have to see if the uncertainty is sufficiently low there (but K<50 means that the uncertainty is reliable)
%RAM: I still like present tense here
%RAM: True, but I actually don't really verify the MBAR uncertainties. It is more that the uncertainties are small enough (in addition to being reliable) that the ITIC or GCMC approaches provide precise estimates of phase equilibria. To your point, I have tried to convey that the Keff>50 for MBAR statement applies to whether the estimates are precise enough for ITIC. 
%MRS3: I think fundamentaly whether it works or not is more for averages of a single property or a single free energy; the more complicated combinations of algorithms will be a consequence of whether the averages are computed accurately.  
%\begin{enumerate}
%	\item Traditionally, histogram reweighting (HR) has been applied with GCMC to calculate vapor-liquid coexistence properties
%	\item Present histogram reweighting equations
%	\item Discuss how to compute phase equilibria by equating pressures
%	\item Discuss how to compute heat of vaporization
%	\item In this study, we demonstrate how to compute VLE using GCMC-MBAR
%	\item Procedure is identical to that utilized for HR but using the MBAR equations
%	\item Present MBAR equations
%	\item MBAR for $\theta = \theta_{\rm ref}$ is mathematically equivalent to histogram reweighting in the limit of zero bin width
%	\item GCMC-MBAR allows for prediction of multiple force fields from single simulation without modifying force fields mid-simulation (i.e., Hamiltonian scaling approach)
%	\item MBAR uncertainties are computed using bootstrap resampling
%\end{enumerate}

\subsection{Basis functions} \label{sec: Basis functions}

When applying GCMC-MBAR to different force field parameter sets $(\theta_{\rm rr} \neq \theta_{\rm ref})$ it is necessary to recompute the internal energy for each snapshot ($U(\x_{n},N_n;\theta_{\rm rr})$ in Equation \ref{eq: reduced potential}). GCMC-HR typically requires millions of snapshots for precise estimates of $\rho_{\rm liq}^{\rm sat}$, $\rho_{\rm vap}^{\rm sat}$, $P_{\rm vap}^{\rm sat}$, and $\Delta H_{\rm v}$ over a wide range of $T^{\rm sat}$. The naive GCMC-MBAR approach when $\theta_{\rm rr} \neq \theta_{\rm ref}$ is to store the molecular configurations $(\x_{n})$ at each snapshot and then recompute $U(\x_{n},N_n;\theta_{\rm rr})$. Although this ``rerun'' process in GOMC is orders of magnitude faster than performing direct GCMC simulations with $\theta_{\rm rr}$, the naive approach is memory intensive and computationally expensive. Fortunately, basis functions can greatly accelerate the energy recomputation step \cite{naden:jctc:2016,Postdoc_1}. Section \ref{sec: Case study} utilizes basis functions to rapidly recompute the non-bonded energies for $\theta_{\rm rr} \neq \theta_{\rm ref}$.

Basis functions are applicable whenever the energy can be separated linearly with respect to the force field parameters. For example, the Mie $\lambda$-6 non-bonded energy is separated into a repulsive and attractive term that can be expressed as
\begin{equation} \label{Mie basis function}
u^{\rm nb}(C_6,C_\lambda;r) = C_\lambda r^{-\lambda} - C_6 r^{-6}
\end{equation} 
where $C_6$ and $C_\lambda$ are proportional to $\epsilon \sigma^6$ and $\epsilon \sigma^\lambda$, respectively. Therefore, the total non-bonded energy between all $\alpha$ and $\beta$ sites $(U_{\alpha\beta}^{\rm nb})$ is simply
\begin{equation} \label{Utotal basis function}
U_{\alpha\beta}^{\rm nb}(C_6,C_\lambda) = C_\lambda \sum_{i \neq j} r_{ij}^{-\lambda} - C_6 \sum_{i \neq j} r_{ij}^{-6} = C_\lambda \Psi_{\lambda} + C_6 \Psi_{6}
\end{equation}
where $\Psi_{\lambda} (\equiv \sum_{i \neq j} r_{ij}^{-\lambda})$ is the repulsive basis function, $\Psi_{6} (\equiv \sum_{i \neq j} r_{ij}^{-6})$ is the attractive basis function and, for simplicity, $\sum_{i \neq j}$ denotes a sum over all unique pairwise interactions. Note that, because $C_\lambda$ depends on $\lambda$, a separate repulsive basis function is required for each value of $\lambda$. For this reason, we adopt the common practice of limiting $\lambda$ to integer values.

With Equation \ref{Utotal basis function}, the total non-bonded internal energy for all interaction sites of all $K^{\rm tot}_{\rm snaps}$ snapshots can be recomputed for any $\epsilon$ and $\sigma$ with linear algebra instead of computing $u^{\rm nb}(\theta_{\rm rr};r_{ij})$ for each unique pairwise interaction. Storing $\Psi_{\lambda}$ and $\Psi_{6}$ for $K^{\rm tot}_{\rm snaps}$ snapshots also greatly reduces the memory storage load compared to storing $K^{\rm tot}_{\rm snaps}$ configurational snapshots (which must be full-precision for reliable ``rerun'' results). 

Although it is preferable to output $\Psi_{\lambda}$ and $\Psi_{6}$ at runtime, including this capability would require significant modification of GOMC. Instead, we utilize the GOMC ``rerun'' feature to indirectly compute $\Psi_{\lambda}$ and $\Psi_{6}$ post-simulation. Specifically, we recompute $U_{\alpha\beta}^{\rm nb}$ for $\sigma_{\rm rr} \neq \sigma_{\rm ref}$ with $\lambda_{\rm rr} = \lambda_{\rm ref}$ and $\epsilon_{\rm rr} = \epsilon_{\rm ref}$, such that $C_{6\rm{, rr}} \neq C_{6\rm{, ref}}$ and $C_{\lambda \rm{, rr}} \neq C_{\lambda \rm{, ref}}$. We then perform an additional ``rerun'' calculation for each value of $\lambda_{\rm rr} \neq \lambda_{\rm ref}$. Having completed these ``reruns,'' the configuration files are no longer needed. Finally, we solve a system of equations for $\Psi_{6}$ and $\Psi_{\lambda}$. 

For example, the Lennard-Jones 12-6 and Mie 16-6 basis functions ($\Psi_{6}$, $\Psi_{12}$, and $\Psi_{16}$) are obtained by solving the following expression:
\[
\begin{bmatrix}
C_{6}(\sigma_{\rm ref},\epsilon_{\rm ref}) & C_{12}(\sigma_{\rm ref},\epsilon_{\rm ref}) & 0 \\
C_{6}(\sigma_{\rm rr},\epsilon_{\rm ref}) & C_{12}(\sigma_{\rm rr},\epsilon_{\rm ref}) & 0 \\
C_{6}(\sigma_{\rm rr},\epsilon_{\rm ref}) & 0 & C_{16}(\sigma_{\rm rr},\epsilon_{\rm ref})
\end{bmatrix}
\begin{bmatrix}
\Psi_{6} \\
\Psi_{12} \\
\Psi_{16}
\end{bmatrix}
=
\begin{bmatrix}
U_{\alpha\beta}^{\rm nb}(\sigma_{\rm ref},\epsilon_{\rm ref},12) \\
U_{\alpha\beta}^{\rm nb}(\sigma_{\rm rr},\epsilon_{\rm ref},12) \\
U_{\alpha\beta}^{\rm nb}(\sigma_{\rm rr},\epsilon_{\rm ref},16)
\end{bmatrix}
\]
where $U_{\alpha\beta}^{\rm nb}(\sigma_{\rm ref},\epsilon_{\rm ref},12)$ is the reference non-bonded energy, $U_{\alpha\beta}^{\rm nb}(\sigma_{\rm rr},\epsilon_{\rm ref},12)$ is the ``rerun'' non-bonded energy for $\sigma_{\rm rr} \neq \sigma_{\rm ref}$, $\epsilon_{\rm rr} = \epsilon_{\rm ref}$, and $\lambda_{\rm rr} = \lambda_{\rm ref} = 12$, and $U_{\alpha\beta}^{\rm nb}(\sigma_{\rm rr},\epsilon_{\rm ref},16)$ is the ``rerun'' non-bonded energy for $\sigma_{\rm rr} \neq \sigma_{\rm ref}$, $\epsilon_{\rm rr} = \epsilon_{\rm ref}$, and $\lambda_{\rm rr} = 16$.

%\[
%\begin{bmatrix}
%C_{6}(\sigma_{\rm ref}) & C_{\lambda_{\rm ref}}(\sigma_{\rm ref}) & 0 \\
%C_{6}(\sigma_{\rm rr}) & C_{\lambda_{\rm ref}}(\sigma_{\rm rr}) & 0 \\
%C_{6}(\sigma_{\rm rr}) & 0 & C_{\lambda_{\rm rr}}(\sigma_{\rm rr})
%\end{bmatrix}
%\begin{bmatrix}
%\Psi_{6} \\
%\Psi_{12} \\
%\Psi_{16}
%\end{bmatrix}
%=
%\begin{bmatrix}
%U_{\alpha\beta}^{\rm nb}(\sigma_{\rm ref},12) \\
%U_{\alpha\beta}^{\rm nb}(\sigma_{\rm rr},12) \\
%U_{\alpha\beta}^{\rm nb}(\sigma_{\rm rr},16)
%\end{bmatrix}
%\]

%For example, computing basis functions for a Lennard-Jones 12-6 and a Mie 16-6 potential requires solving the following system of equations:
%\[
%\begin{bmatrix}
%C_{6}(\sigma_{\rm ref}) & C_{12}(\sigma_{\rm ref}) & 0 \\
%C_{6}(\sigma_{\rm rr}) & C_{12}(\sigma_{\rm rr}) & 0 \\
%C_{6}(\sigma_{\rm rr}) & 0 & C_{16}(\sigma_{\rm rr})
%\end{bmatrix}
%\begin{bmatrix}
%\Psi_{6} \\
%\Psi_{12} \\
%\Psi_{16}
%\end{bmatrix}
%=
%\begin{bmatrix}
%U_{\alpha\beta}^{\rm nb}(\sigma_{\rm ref},12) \\
%U_{\alpha\beta}^{\rm nb}(\sigma_{\rm rr},12) \\
%U_{\alpha\beta}^{\rm nb}(\sigma_{\rm rr},16)
%\end{bmatrix}
%\]
%which can be solved with linear algebra.

% the superscript $\lambda$) and 

%Computing $\Psi_{\lambda}$ for multiple values of $\lambda$ requires an additional ``rerun'' calculation with the corresponding $\lambda_{\rm rr}$.
%, such that $C_{\lambda \rm{, rr}} \neq C_{\lambda \rm{, ref}}$.

%% An additional rerun simulation is required for each value of $\lambda$.

%  as this  different values of $\sigma$ and $\lambda$ (and, thereby, different values of $C_{\lambda}$ and $C_6$). If we are and solve a system of equations for  with $\sigma_{\rm rr} \neq \sigma_{\rm ref}$  

%Details regarding how we compute $\Psi_{\lambda}$ and $\Psi_{6}$ with GOMC are provided as Supporting Information.

%For a given $\lambda$, the total internal energy can be recomputed for any value of $\epsilon$ and $\sigma$. As $C_\lambda$ depends on $\lambda$, a separate repulsive basis function is required for each value of $\lambda$. 

%Rather than storing configurations (which must be full-precision for reliable ``rerun'' results) for all $K^{\rm tot}_{\rm snaps}$ snapshots, the basis function approach stores $\Psi_{\lambda}$ and $\Psi_{6}$. 
%
%Furthermore, storing $\Psi_{\lambda}$ and $\Psi_{6}$ for $K^{\rm tot}_{\rm snaps}$ snapshots greatly reduces the memory storage load compared to storing $K^{\rm tot}_{\rm snaps}$ configurational snapshots (which must be full-precision for reliable ``rerun'' results). 
%
%Therefore, basis functions present significant benefits from both a memory and computational cost standpoint.
%
%  basis functions only require storing two floating point values ($\sum r^{-\lambda}$ and $\sum r^{-6}$). Therefore, basis functions greatly reduce both the memory storage load as well as the computational cost to recompute the energy for each snapshot. 

%In principle, basis functions can be output at simulation runtime. In practice, however, basis functions are computed as a post-processing step. Specifically, the 

%\begin{enumerate}
%	\item When applying MBAR to different parameter sets, $\theta \neq \theta_{\rm ref}$, it is necessary to recompute energies
%	\item Basis functions accelerate the recompute energy step by storing the repulsive and attractive contributions that can be scaled by $\epsilon$ and $\sigma$
%	\item Basis functions are computed from GOMC using the recompute feature for different $\epsilon$ and $\sigma$ and solving system of equations
%\end{enumerate}

\subsection{$\epsilon$-scaling} \label{sec: eps scaling methods}

Recently, Weidler and Gross proposed an $\epsilon$-scaling approach for converting the Transferable Anisotropic Mie (TAMie)\cite{TAMie} parameters into individualized (compound-specific) parameters (iTAMie) \cite{Weidler2018}. The philosophy for individualized parameters is that some compounds have sufficient reliable experimental data to refine the force field parameters for a specific molecule. However, refitting all non-bonded parameters simultaneously would likely lead to an underspecificed optimization. To avoid overfitting, Weidler and Gross optimize a single adjustable parameter $(\psi)$ that scales all the $\epsilon$ values in a molecule according to
\begin{equation}
\epsilon_{ii}^{\rm ind} = \psi \times \epsilon_{ii}^{\rm tran}
\end{equation}
where $\epsilon_{ii}^{\rm ind}$ is the individualized $\epsilon$ value for united-atom $ii$, $\epsilon_{ii}^{\rm tran}$ is the corresponding transferable $\epsilon$ value, and $\psi$ is a fixed value for a given compound.

%To avoid overfitting, a one-dimensional optimization is employed which scales $\epsilon$ for all united-atom sites while not adjusting $\sigma$ or $\lambda$.

%Reference \citenum{Weidler2018} proposes   

%Recently, Weidler and Gross proposed ``individualized,'' i.e., compound-specific, parameter sets for compounds which contain large amounts of experimental data \cite{Weidler2018}. To avoid overfitting, a one-dimensional optimization is employed which scales $\epsilon$ for all united-atom sites while not adjusting $\sigma$ or $\lambda$. 

%This requires applying GCMC-MBAR with $\epsilon_{\rm rr} = \psi \times \epsilon_{\rm ref}$ while $\sigma_{\rm rr} = \sigma_{\rm ref}$ and $\lambda_{\rm rr} = \lambda_{\rm ref}$.

%The philosophy for individualized parameters is that some compounds have sufficient reliable experimental data to refine the force field parameters for a specific molecule. However, refitting all non-bonded parameters simultaneously would likely lead to an underspecificed optimization and, thus, an overfit parameter set. For this reason, Weidler et al. optimize a single adjustable parameter $(\psi)$ that scales all the $\epsilon$ values according to
%\begin{equation}
%\epsilon_{ii}^{\rm ind} = \psi \epsilon_{ii}^{\rm tran}
%\end{equation}
%where $\epsilon_{ii}^{\rm ind}$ is the individualized $\epsilon$ value for united-atom $ii$, $\epsilon_{ii}^{\rm tran}$ is the corresponding transferable $\epsilon$ value, and $\psi$ is a constant value for a given compound.

GCMC-MBAR is ideally suited for this ``$\epsilon$-scaling'' approach for at least two reasons. First, MBAR is most reliable when extrapolating in $\epsilon$ rather than $\sigma$ and/or $\lambda$ \cite{Postdoc_1}. Second, because $\epsilon_{\rm rr} = \psi \times \epsilon_{\rm ref}$ while $\sigma_{\rm rr} = \sigma_{\rm ref}$ and $\lambda_{\rm rr} = \lambda_{\rm ref}$, recomputing the total non-bonded energy for each snapshot is simply 
\begin{equation}
U^{\rm nb, tot}_{\rm rr} = \psi \times U^{\rm nb, tot}_{\rm ref}
\end{equation}
where $U^{\rm nb, tot}_{\rm rr}$ and $U^{\rm nb, tot}_{\rm ref}$ are the total non-bonded energy with $\theta_{\rm rr}$ and $\theta_{\rm ref}$, respectively. Therefore,  $\epsilon$-scaling does not require storing and recomputing configurations or basis functions. 

% the rate-limiting step for GCMC-MBAR is recomputing the configurational energies for a different force field. Furthermore, storing millions of configuration (``snapshots'') is highly memory intensive. While basis functions (see Section \ref{sec: Basis functions}) alleviate the additional computational cost and reduce the memory load, $\epsilon$-scaling does not require storing/recomputing configurations or basis functions. Instead,

%Second, the rate-limiting step for GCMC-MBAR is recomputing the configurational energies for a different force field. Furthermore, storing millions of configuration (``snapshots'') is highly memory intensive. While basis functions (see Section \ref{sec: Basis functions}) alleviate the additional computational cost and reduce the memory load, $\epsilon$-scaling does not require storing/recomputing configurations or basis functions. Instead, because $\epsilon_{\rm rr} = \psi \times \epsilon_{\rm ref}$ while $\sigma_{\rm rr} = \sigma_{\rm ref}$ and $\lambda_{\rm rr} = \lambda_{\rm ref}$, recomputing the total non-bonded energy for each snapshot is simply 
%\begin{equation}
%U^{\rm nb, tot}_{\rm rr} = \psi \times U^{\rm nb, tot}_{\rm ref}
%\end{equation}
%where $U^{\rm nb, tot}_{\rm rr}$ and $U^{\rm nb, tot}_{\rm ref}$ are the total non-bonded energy with $\theta_{\rm rr}$ and $\theta_{\rm ref}$, respectively.

%MRS3: maybe provide a bit of information as to why the scoring function is constructed as it is (without necessarily endorsing this particular combination).  Seems pretty black box-y to me.  I guess some of the description is a bit later on, but perhaps a bit more context would be good?
%RAM: I now point the reader to the paragraph below
Section \ref{sec: eps scaling} applies the GCMC-MBAR $\epsilon$-scaling approach to convert the MiPPE-SL semi-transferable parameters into individualized parameters (iMiPPE) for 8 branched alkanes and 11 alkynes. For consistency with the original MiPPE-SL optimization, we use the same scoring function for the branched alkanes and alkynes as Mick et al.~\cite{Potoff_branched} and Soroush Barhaghi et al.~\cite{Barhaghi2017}, respectively
\begin{multline} \label{eq: Score}
S= \frac{1}{N_{\rm exp}} [ w_0 \sum_{j=0}^{N_{\rm exp}} APD(\rho_{\rm liq}^{\rm sat}(T^{\rm sat}_j)) + w_1 \sum_{j=0}^{N_{\rm exp}} APD(\rho_{\rm vap}^{\rm sat}(T^{\rm sat}_j)) \\ + w_2 \sum_{j=0}^{N_{\rm exp}} APD(P_{\rm vap}^{\rm sat}(T^{\rm sat}_j)) + w_3 \sum_{j=0}^{N_{\rm exp}} APD(\Delta H_{\rm v} (T^{\rm sat}_j)) \\ + w_4 \sum_{j=0}^{N_{\rm exp}-1} \frac{d APD(\rho_{\rm liq}^{\rm sat}(T^{\rm sat}_j))}{dT} + w_5 \sum_{j=0}^{N_{\rm exp}-1} \frac{d APD(\rho_{\rm vap}^{\rm sat}(T^{\rm sat}_j))}{dT} \\ + w_6 \sum_{j=0}^{N_{\rm exp}-1} \frac{d APD(P_{\rm vap}^{\rm sat}(T^{\rm sat}_j))}{dT} + w_7 \sum_{j=0}^{N_{\rm exp}-1} \frac{d APD(\Delta H_{\rm v} (T^{\rm sat}_j))}{dT} ]
\end{multline}
where $S$ is the scoring function, $N_{\rm exp}$ is the number of (pseudo-)experimental data points, $w_{x}$ are the weights for property $x$ (see below), $T^{\rm sat}_j$ are the saturation temperatures for data point $j$, and the absolute percent deviation $(APD)$ is defined as
\begin{equation} \label{eq: APD}
APD(X) = \left| \frac{X_{\rm sim} - X_{\rm exp}}{X_{\rm exp}} \right| \times 100 
\end{equation}
where $X_{\rm sim}$ and $X_{\rm exp}$ correspond to the respective simulation and (pseudo-) experimental values for property $X$ (e.g., $\rho_{\rm liq}^{\rm sat}$). The derivative terms in Equation \ref{eq: Score} are computed with
\begin{equation}
\frac{d APD(X(T^{\rm sat}_{j}))}{dT} = \frac{APD(X(T^{\rm sat}_{j+1}))-APD(X(T^{\rm sat}_{j}))}{T^{\rm sat}_{j+1} - T^{\rm sat}_j}
\end{equation}
where $j+1$ and $j$ are sequential data points ordered according to $T^{\rm sat}$ and $T^{\rm sat}_{j+1} - T^{\rm sat}_j = 10$ K.
%MRS2: maybe provide a little explanation for the weights, even though it's not your scheme? 
% different for branched alkanes and alkynes, specifically, $w_x = $ (0.6135, 0.0123, 0.2455, 0.0245, 0.0613, 0.0061, 0.0245, 0.0123) and $w_x = $ (0.757, 0, 0.152, 0, 0.076, 0, 0.015, 0) for branched alkanes and alkynes, respectively.
%RAM: I moved this below. I hope the discussion is helpful. I should check with Potoff and Mohammad to make sure it is accurate though.

In accordance with the work of Mick et al.\cite{Potoff_branched}~and Soroush Barhaghi et al.\cite{Barhaghi2017}, the target values $(X_{\rm exp})$ are not experimental data, rather they are computed values from correlations fit to experimental data. Specifically, the alkyne correlations are from the Design Institute for Physical Properties (DIPPR) \cite{DIPPR} while the branched alkane correlations are from the National Institute of Standards and Technology (NIST) Reference Fluid Properties (REFPROP) database \cite{LEMMON-RP10,Lemmon2006,Blackham2017,Gao2017}. Although REFPROP correlations exist for ethyne\cite{Gao2017}, propyne\cite{Gao2017}, and 1-butyne\cite{Polt1992}, we utilize the DIPPR correlations for consistency within the alkyne family and to be consistent with Soroush Barhaghi et al.  

We use the same weights $(w_x)$ in Equation \ref{eq: Score} as Mick et al.\cite{Potoff_branched}~and Soroush Barhaghi et al.\cite{Barhaghi2017}, namely, $w_x =$ (0.6135, 0.0123, 0.2455, 0.0245, 0.0613, 0.0061, 0.0245, 0.0123) and (0.757, 0, 0.152, 0, 0.076, 0, 0.015, 0) for branched alkanes and alkynes, respectively. The $\rho_{\rm liq}^{\rm sat}$ weights are greatest due to both the high precision of experimental data and the high importance that the MiPPE force field reproduce this property. The $P_{\rm vap}^{\rm sat}$ weights are second greatest, demonstrating the importance of this property but the slightly higher experimental uncertainties. The $\rho_{\rm vap}^{\rm sat}$ and $\Delta H_{\rm v}$ weights are 0 for alkynes as the DIPPR database does not report $\rho_{\rm vap}^{\rm sat}$ and the DIPPR $\Delta H_{\rm v}$ uncertainties are relatively large for most alkynes. Despite having weights that are an order of magnitude smaller, the derivative terms help ensure that the optimal force field has similar accuracy over the entire temperature range. 

%are simply multiplied by the $\epsilon$-scaling parameter.  

\section{Results} \label{sec: Results}

Four different applications for GCMC-MBAR are demonstrated in this study, where slightly different types of simulation output are required. Section \ref{sec: Constant theta} demonstrates that GCMC-MBAR yields consistent results to those previously reported using histogram reweighting when the force field parameters do not change, i.e., $\theta_{\rm rr} = \theta_{\rm ref}$. The same simulation output as GCMC-HR is used in this application, namely, a $2 \times K^{\rm tot}_{\rm snaps}$ array containing the number of molecules $(N)$ and the internal energy $(U)$ for all $K^{\rm tot}_{\rm snaps}$ snapshots. Section \ref{sec: eps scaling} demonstrates how these same data can be used with MBAR to perform a one-dimensional optimization with $\epsilon$-scaling. Section \ref{sec: litFF} investigates how well GCMC-MBAR performs when predicting vapor-liquid coexistence properties for force field $j$ from configurations sampled with force field $i$, where $i$ and $j$ are common force fields from the literature. In this case, a $3 \times K^{\rm tot}_{\rm snaps}$ array is required, where the additional column is the internal energy computed with force field $j$. Section \ref{sec: Case study} presents a case study for optimizing Mie $\lambda$-6 non-bonded parameters for cyclohexane. Basis functions are employed as a computationally efficient method for predicting vapor-liquid phase equilibria with hundreds of $\epsilon_{\rm CH_2}$, $\sigma_{\rm CH_2}$, and $\lambda_{\rm CH_2}$ values that are unknown at runtime. Using basis functions requires storing a $(N_{\Psi} + 1) \times K^{\rm tot}_{\rm snaps}$ array, where $N_{\Psi}$ is the number of basis functions and the $N_{\Psi} + 1$ column corresponds to the values of $N$. For example, the Mie 16-6 potential has two basis functions ($N_{\Psi} = 2$), one for $\Psi_{6}$ and one for $\Psi_{16}$. Supporting Information contains basis function files for the TraPPE and MiPPE simulations of cyclohexane.

%In this section, we present results from several different applications of GCMC-MBAR. First, we compute vapor-liquid coexistence properties in the case where the force field parameters do not change, i.e., $\theta_{\rm rr} = \theta_{\rm ref}$. Second, we perform a one-dimensional optimization in the $\epsilon$-scaling parameter, $\psi$. Third, we determine the reliability of GCMC-MBAR when $\theta_{\rm rr} \neq \theta_{\rm ref}$ for different literature force fields. Fourth, we demonstrate how GCMC-MBAR can be applied to obtain new Mie $\lambda$-6 parameters for cyclohexane.

%\subsection{$\theta_{\rm rr} = \theta_{\rm ref}$} \label{sec: Constant theta}  
\subsection{Validation of GCMC-MBAR for $\theta_{\rm rr} = \theta_{\rm ref}$} \label{sec: Constant theta} 
%MRS2: maybe a clearer section name that isn't just variables.

Section \ref{sec: GCMC-HR and GCMC-MBAR} demonstrated that MBAR and HR are mathematically equivalent in the limit of zero bin width when $\theta_{\rm rr} = \theta_{\rm ref}$. Figure \ref{fig:comparison MBAR HR} provides numerical validation that GCMC-MBAR and GCMC-HR yield indistinguishable vapor-liquid coexistence properties (tabulated GCMC-MBAR estimates and uncertainties are provided in Section \ref{SI sec: Tabulated MBAR results} of Supporting Information). The evidence for this conclusion is that the median percent deviation is approximately zero and the largest deviations are within a few percent, which are typically smaller than the combined statistical uncertainties between HR and MBAR. Note that the uncertainties are largest near the critical point, i.e., reduced temperatures $(T_{\rm r}) \approx 1$. 
%MRS: is the few percent within statistical error?  State explicitly in the text.

	\begin{figure}[htb!]
		\centering
		\includegraphics[width=6.4in]{Comparison_MBAR_HR_boxplot_CI.pdf}
		\caption{Percent deviations between coexistence properties computed using histogram reweighting (HR) and Multistate Bennett Acceptance Ratio (MBAR). The HR and MBAR results are in good agreement, i.e., within a few percent and a median percent deviation of approximately 0\%. Top-left, top-right, bottom-left, and bottom-right panels correspond to saturated liquid density, saturated vapor density, saturated vapor pressure, and enthalpy of vaporization, respectively. Middle line denotes the median deviation, boxes depict the first and third quartiles, and whiskers represent the range that contains 95\% of the data.}
		\label{fig:comparison MBAR HR}
	\end{figure}

The percent deviations shown in Figure \ref{fig:comparison MBAR HR} are averaged over the 31 branched alkanes studied by Mick et al.\cite{Potoff_branched}~ (only the MiPPE-SL data are re-evaluated) and the 11 alkynes studied by Soroush Barhaghi et al.\cite{Barhaghi2017} The GCMC-HR values used in Figure \ref{fig:comparison MBAR HR} were not recomputed in this study but were taken from the literature \cite{Potoff_branched,Barhaghi2017}. For a fair comparison between GCMC-HR and GCMC-MBAR, the GCMC-MBAR values were computed using the same raw simulation data as Mick et al.~and Soroush Barhaghi et al. However, for simplicity, we only reprocess one of the five replicate simulations.

% The GCMC-MBAR values were computed using a subset of the raw simulation data from Mick et al. and Barhaghi et al. Specifically, snapshots from only a single replicate simulation were used for the GCMC-MBAR analysis, while the GCMC-HR values reported by Mick et al. and Barhaghi et al. utilized five replicate simulations.

% he difference being that the GCMC-MBAR values were obtained from only a single replicate, whereas GCMC-HR used data from all five replicates. 

%(a), (b), (c), and (d)

%\subsection{$\epsilon_{\rm rr} = \psi \times \epsilon_{\rm ref}$} \label{sec: eps scaling}
\subsection{Post-simulation $\epsilon$-scaling with GCMC-MBAR} \label{sec: eps scaling}
%MRS2: can this start with more of a summary of what the reader should think/approach the findings?  Stating what this experiment explains would help readers follow.
%RAM: I agree. I now provide a paragraph explainng the motivation for this section.
%Figure \ref{fig:epsilon_scaling} presents the $\epsilon$-scaling results for 8 branched alkanes and 11 alkynes using the MiPPE force field as the initial force field. For consistency with the original MiPPE optimization, we use the same scoring function for the branched alkanes and alkynes as Mick et al. and Barhaghi et al., respectively
%\begin{multline} \label{eq: Score}
%S= \frac{1}{N_{\rm exp}} [ w_0 \sum_{j=0}^{N_{\rm exp}} APD(\rho_{\rm liq}^{\rm sat}(T^{\rm sat}_j)) + w_1 \sum_{j=0}^{N_{\rm exp}} APD(\rho_{\rm vap}^{\rm sat}(T^{\rm sat}_j)) \\ + w_2 \sum_{j=0}^{N_{\rm exp}} APD(P_{\rm vap}^{\rm sat}(T^{\rm sat}_j)) + w_3 \sum_{j=0}^{N_{\rm exp}} APD(\Delta H_{\rm v} (T^{\rm sat}_j)) \\ + w_4 \sum_{j=0}^{N_{\rm exp}} \frac{d APD(\rho_{\rm liq}^{\rm sat}(T^{\rm sat}_j))}{dT} + w_5 \sum_{j=0}^{N_{\rm exp}} \frac{d APD(\rho_{\rm vap}^{\rm sat}(T^{\rm sat}_j))}{dT} \\ + w_6 \sum_{j=0}^{N_{\rm exp}} \frac{d APD(P_{\rm vap}^{\rm sat}(T^{\rm sat}_j))}{dT} + w_7 \sum_{j=0}^{N_{\rm exp}} \frac{d APD(\Delta H_{\rm v} (T^{\rm sat}_j))}{dT} ]
%\end{multline}
%where $S$ is the scoring function, $N_{\rm exp}$ is the number of experimental data points, $w_{x}$ are the weights for property $x$, $T^{\rm sat}_j$ are the saturation temperatures for data point $j$, and the absolute percent deviation $(APD)$ is defined as
%\begin{equation} \label{eq: APD}
%APD(X) = \left| \frac{X_{\rm sim} - X_{\rm exp}}{X_{\rm exp}} \right| \times 100 \% 
%\end{equation}
%where $X_{\rm sim}$ and $X_{\rm exp}$ correspond to the respective simulation and (pseudo-) experimental values for property $X$ (e.g., $\rho_{\rm liq}^{\rm sat}$). Note that the weights $(w_x)$ are different for branched alkanes and alkynes, specifically, $w_x = $ (0.6135, 0.0123, 0.2455, 0.0245, 0.0613, 0.0061, 0.0245, 0.0123) and $w_x = $ (0.757, 0, 0.152, 0, 0.076, 0, 0.015, 0) for branched alkanes and alkynes, respectively.
%
%In accordance with the work of Mick et al. and Barhaghi et al., the target values $(X_{\rm exp})$ are computed with pseudo-experimental correlations. The alkyne correlations are from the Design Institute for Physical Properties (DIPPR) while the branched alkane correlations are from the National Institute of Standards and Technology (NIST) Reference Fluid Properties (REFPROP) database.

% Reference Properties (REFPROP) values) are  alkyne target values are the same as those used by Barhaghi et al., namely, they are computed with pseudo-experimental correlations from the Design Institute for Physical Properties (DIPPR). The branched alkane target values are computed with  also the same as Mick et al.

As discussed in Section \ref{sec: eps scaling methods}, $\epsilon$-scaling is used to obtain an individualized (compound-specific) parameter ($\psi$) for well-studied compounds, i.e., those with a large amount of reliable experimental data. \cite{Weidler2018} In this section, we demonstrate that GCMC-MBAR is an efficient tool for performing $\epsilon$-scaling. 

%Simulations are performed with the MiPPE-SL force field and coexistence properties are computed over a range of $\psi$ values, where $\epsilon_{\rm rr} = \psi \times \epsilon_{\rm ref}$.

Figure \ref{fig:epsilon_scaling} presents the $\epsilon$-scaling results for 8 branched alkanes and 11 alkynes using the MiPPE-SL force field as $\theta_{\rm ref}$ $(\psi = 1)$. The same simulation data as discussed in Section \ref{sec: Constant theta} are re-purposed to compute phase equilibria properties over a range of $\psi$ values (where $\epsilon_{\rm rr} = \psi \times \epsilon_{\rm ref}$) without performing any additional simulations. Optimal $\psi$ values for individualized MiPPE (iMiPPE) are provided in Section \ref{SI sec: eps scale} of Supporting Information. 

	\begin{figure}[htb!]
		\centering
		\includegraphics[width=6.4in]{Optimal_epsilon_scaling.pdf}
		\caption{One-dimensional post-simulation optimization of $\epsilon$-scaling parameter $(\psi)$ for select branched alkanes (left) and alkynes (right) with MiPPE-SL as initial force field. MBAR enables prediction of scoring function over range of $\psi$ values from configurations that were sampled with $\psi = 1$ (dashed line). Open symbols correspond to the optimal $\psi$ value for a given compound.}
		\label{fig:epsilon_scaling}
	\end{figure}

Weidler et al.~tend to characterize the individualization as being useful when the scaling is greater than 0.4\% (i.e., $|1 - \psi| > 0.004$).\cite{Weidler2018} With this rationale, 3-methylpentane is the only branched alkane that merits $\epsilon$-scaling with the MiPPE-SL force field. Similarly, the TAMie force field also found $\psi \approx 1$ for all branched alkanes, except 3-methylpentane. 

Figure \ref{fig:epsilon_scaling} shows that the MiPPE parameters require a greater degree of scaling for alkynes than for branched alkanes. A truly transferable force field should have $\psi \approx 1$ for all compounds. Although $\psi$ values for iTAMie were not reported for alkynes, the largest $\psi$ value for olefins, ethers, and ketones was $\approx 1.01$.\cite{Weidler2018} Therefore, the transferability of the MiPPE force field appears to be slightly poorer for 2-pentyne and 2-hexyne, which have an optimized $\psi > 1.01$. 

It is also interesting that only 3 out of 19 compounds require $\psi < 1$. Thus, the well-depths appear to be slightly underestimated by the MiPPE force field. By contrast, this trend was not observed in Reference \citenum{Weidler2018} for TAMie. Also, note that branched alkanes have a pronounced minimum in $S$ with respect to $\psi$, whereas the minimum is more gradual for the alkynes. We attribute this, in part, to the fact that the alkynes do not include $\Delta H_{\rm v}$ (a property that depends strongly on $\epsilon$) in the scoring function, i.e., $w_4 = 0$ and $w_7 = 0$. 

%.  is introduced should result in low deviations between experiment and simulation, it is likely that such a parameter set would be overfit and, thus, perform poorly some compounds have sufficient experimental 

% MBAR is ideally suited for $\epsilon$-scaling for at least two reasons.
 
%At least two reasons exist for why MBAR is ideally suited for $\epsilon$-scaling. First, the energies in Equation \ref{BLANK} can be scaled by $\psi$ such that the configurations do not need to be stored or recomputed. Second, MBAR is more reliable for changes in $\epsilon$ rather than changes in $\sigma$ and/or $\lambda$ \cite{Postdoc_1}. 

%\subsection{$\epsilon_{\rm ref} \neq \epsilon_{\rm rr}$, $\sigma_{\rm ref} \neq \sigma_{\rm rr}$, and $\lambda_{\rm ref} \neq \lambda_{\rm rr}$} \label{sec: litFF}
\subsection{Performance of GCMC-MBAR when varying $\epsilon$, $\sigma$, and $\lambda$} \label{sec: litFF}
%MRS3: didn't the previous section already start to address the question of when theta_rr =/= theta_ref?  Use different name?
A more demanding test of GCMC-MBAR than $\epsilon$-scaling is to vary several non-bonded parameters simultaneously, including $\sigma$ and $\lambda$. To perform a test of GCMC-MBAR when $\theta_{\rm rr} \not\approx \theta_{\rm ref}$, we estimate the phase equilibria of several branched alkanes for force field $j$ by reweighting configurations sampled with force field $i$ (denoted as $\theta_{\rm i} \Rightarrow \theta_{\rm j}$), where $i$ and $j$ are either the MiPPE-gen, MiPPE-SL, TraPPE, or NERD force fields. 

TraPPE $\Leftrightarrow$ MiPPE-gen is the most challenging test as all three non-bonded parameters ($\epsilon$, $\sigma$, and $\lambda$) for all four united-atom types (CH$_3$, CH$_2$, CH, and C) are different between the TraPPE and MiPPE-gen force fields. TraPPE $\Rightarrow$ NERD is the second most challenging test because all $\epsilon$ and $\sigma$ values are substantially different but $\lambda = 12$ for both force fields. MiPPE-gen $\Rightarrow$ MiPPE-SL is considerably easier because these two force fields only differ in the $\epsilon$ and/or $\sigma$ values for the CH and C sites. Furthermore, the difference in $\epsilon$ and $\sigma$ values between MiPPE-gen and MiPPE-SL is significantly smaller than that between TraPPE and NERD (see Table \ref{tab:nonbonded params}). Therefore, MiPPE-gen $\Rightarrow$ MiPPE-SL corresponds to $\theta_{\rm rr} \approx \theta_{\rm ref}$, which is an important scenario when fine-tuning force field parameters.

%Specifically,  we utilize GCMC-MBAR to predict coexistence properties for the NERD and MiPPE-SL force fields using configurations sampled from TraPPE and MiPPE-gen, respectively (see Figure \ref{fig:refFF_to_rrFF_lam_constant}). We also use GCMC-MBAR to predict coexistence properties for the TraPPE force field by sampling configurations with MiPPE-gen, and vice versa (see Figure \ref{fig:refFF_to_rrFF_lam12to16}).

%Because it is not possible to visualize a parameter space of greater than three dimensions, we perform this analysis of GCMC-MBAR using the TraPPE, NERD, MiPPE-gen, and MiPPE-SL force fields. Specifically, we utilize GCMC-MBAR to predict coexistence properties for the NERD and MiPPE-SL force fields using configurations sampled from TraPPE and MiPPE-gen, respectively (see Figure \ref{fig:refFF_to_rrFF_lam_constant}). We also use GCMC-MBAR to predict coexistence properties for the TraPPE force field by sampling configurations with MiPPE-gen, and vice versa (see Figure \ref{fig:refFF_to_rrFF_lam12to16}). 

%Note that all three non-bonded parameters ($\epsilon$, $\sigma$, and $\lambda$) for all four united-atom types (CH$_3$, CH$_2$, CH, and C) are different between the TraPPE and MiPPE-gen force fields. The TraPPE and NERD $\epsilon$ and $\sigma$ values are different for all four united-atom types while $\lambda = 12$ for both force fields. The MiPPE-gen and MiPPE-SL force fields only differ in the $\epsilon$ and/or $\sigma$ values for the CH and C sites. However, the difference in $\epsilon$ and $\sigma$ values for MiPPE-gen and MiPPE-SL is significantly smaller than that between TraPPE and NERD. Therefore, the MiPPE-gen $\Rightarrow$ MiPPE-SL results correspond to $\theta_{\rm rr} \approx \theta_{\rm ref}$, which is important when fine-tuning a pre-optimized force field.

%Specifically, the 2-methylpropane and 2,3-dimethylbutane parameters are the same except for $\sigma_{\rm CH}$, the 2,2-dimethylpropane parameters are the same except for $\epsilon_{\rm C}$, the 2,3,4-trimethylpentane parameters are the same except for $\epsilon_{\rm CH}$ and $\sigma_{\rm CH}$, and the 2,2,4-trimethylpentane parameters are the same except for $\epsilon_{\rm CH}$, $\sigma_{\rm CH}$, and $\sigma_{\rm C}$.

	\begin{figure}[H]
		\centering
		\includegraphics[width=6.4in]{refFF_to_rrFF_lam_constant.pdf}
		\caption{Comparison between GCMC-MBAR estimates (symbols, $\theta_{\rm rr} \neq \theta_{\rm ref}$) and MBAR-HR literature values\cite{Potoff_branched} (lines) with a constant repulsive exponent, i.e., $\lambda_{\rm rr} = \lambda_{\rm ref}$. MBAR predicts both liquid and vapor properties accurately for $\lambda_{\rm rr} = \lambda_{\rm ref}$. GCMC-MBAR estimates for the NERD and MiPPE-SL force fields are computed using configurations sampled from TraPPE and MiPPE-gen, respectively. Top-left, top-right, bottom-left, and bottom-right panels correspond to saturated liquid density, saturated vapor density, saturated vapor pressure, and enthalpy of vaporization, respectively.}
		\label{fig:refFF_to_rrFF_lam_constant}
	\end{figure}
	
	\begin{figure}[H]
		\centering
		\includegraphics[width=6.4in]{refFF_to_rrFF_lam_12to16.pdf}
		\caption{Comparison between GCMC-MBAR estimates (symbols, $\theta_{\rm rr} \neq \theta_{\rm ref}$) and MBAR-HR literature values\cite{Potoff_branched} (lines) with a non-constant repulsive exponent, i.e., $\lambda_{\rm rr} \neq \lambda_{\rm ref}$. MBAR predicts only vapor properties accurately for $\lambda_{\rm rr} \neq \lambda_{\rm ref}$. GCMC-MBAR estimates for the TraPPE force field are computed using configurations sampled from MiPPE-gen, and vice versa. Top-left, top-right, bottom-left, and bottom-right panels correspond to saturated liquid density, saturated vapor density, saturated vapor pressure, and enthalpy of vaporization, respectively.}
		\label{fig:refFF_to_rrFF_lam12to16}
	\end{figure}

Figures \ref{fig:refFF_to_rrFF_lam_constant} and \ref{fig:refFF_to_rrFF_lam12to16} compare the GCMC-MBAR predicted values for $\theta_{\rm rr} \neq \theta_{\rm ref}$ (symbols) to the literature GCMC-HR values (lines) obtained by direct simulation. Figure \ref{fig:refFF_to_rrFF_lam_constant} contains $\lambda_{\rm rr} = \lambda_{\rm ref}$ (TraPPE $\Rightarrow$ NERD and MiPPE-gen $\Rightarrow$ MiPPE-SL) while Figure \ref{fig:refFF_to_rrFF_lam12to16} corresponds to $\lambda_{\rm rr} \neq \lambda_{\rm ref}$ (TraPPE $\Leftrightarrow$ MiPPE-gen). 

The good agreement between corresponding symbols and lines in Figures \ref{fig:refFF_to_rrFF_lam_constant} and \ref{fig:refFF_to_rrFF_lam12to16} shows that MBAR is extremely reliable at predicting vapor phase properties ($\rho_{\rm vap}^{\rm sat}$ and $P_{\rm vap}^{\rm sat}$) for both $\lambda_{\rm rr} = \lambda_{\rm ref}$ and $\lambda_{\rm rr} \neq \lambda_{\rm ref}$. Figure \ref{fig:refFF_to_rrFF_lam_constant} demonstrates that GCMC-MBAR is remarkably accurate at predicting liquid phase properties ($\rho_{\rm liq}^{\rm sat}$ and $\Delta H_{\rm v}$, which depends on both phases) when $\lambda_{\rm rr} = \lambda_{\rm ref}$, even for the fairly significant differences in the TraPPE and NERD $\sigma$ values. However, Figure \ref{fig:refFF_to_rrFF_lam12to16} shows that GCMC-MBAR is unreliable for liquid phase properties when $\lambda_{\rm rr} \neq \lambda_{\rm ref}$. This undesirable behavior can be explained by the low number of effective snapshots in the liquid phase.

%In particular, note that the $\rho_{\rm liq}^{\rm sat}$ estimates in Figure \ref{fig:refFF_to_rrFF_lam12to16} are sporadic and unreliable.

%, which demonstrates that the repulsive exponent $(\lambda)$ greatly impacts the configurational overlap in the liquid phase but not the vapor phase. The poor overlap when varying $\lambda$ is consistent with the MBAR-ITIC results \cite{Postdoc_1}. However, GCMC-MBAR provides considerable improvement in predicting $P_{\rm vap}^{\rm sat}$ compared to MBAR-ITIC. For this reaon, we recommend reducing the $\rho_{\rm liq}^{\rm sat}$ weight in the scoring function when varying $\lambda$. The degree to which the weight is reduced should depend on the number of effective samples.
%MRS2: Without any reasons for the weights given, it makes no sense to talk about changing them -- will that cause some other problem  Are they arbitary nor not? Why not just say more simulations are needed instead of getting into the discussion?
%RAM: I now explain the weights a little more. But it would also be good to say that we need more simulations near these parameter sets.

%Because it is not possible to visualize a parameter space of greater than two dimensions, Figures \ref{fig:refFF_to_rrFF_lam_constant} and \ref{fig:refFF_to_rrFF_lam12to16} demonstrate the reliability of MBAR when multiple parameters are varied simultaneously. 

%For example, all three non-bonded parameters ($\epsilon$, $\sigma$, and $\lambda$) for all four united-atom types (CH$_3$, CH$_2$, CH, and C) are different between the TraPPE and Potoff-gen force fields (see Figure \ref{fig:refFF_to_rrFF_lam12to16}). The TraPPE and NERD $\epsilon$ and $\sigma$ values are different for all four united-atom types while $\lambda = 12$ for both force fields (see Figure \ref{fig:refFF_to_rrFF_lam_constant}). The Potoff-gen and Potoff-SL force fields only differ in the $\epsilon$ and/or $\sigma$ values for the CH and C sites (see Figure \ref{fig:refFF_to_rrFF_lam_constant}). Specifically, the 2-methylpropane parameters are identical except for $\sigma_{\rm CH}$, the 2,2-dimethylpropane parameters are the same except for $\epsilon_{\rm C}$, and three parameters are different for 2,2,4-trimethylpentane ($\epsilon_{\rm CH}$, $\sigma_{\rm CH}$, and $\sigma_{\rm C}$). However, the difference in $\epsilon$ and $\sigma$ values for Potoff-gen and Potoff-SL is significantly smaller than that between TraPPE and NERD.

%MRS3: It would be useful to comment how this is entirely expected, since 1) there are fewer molecules in vapor, and the configurational overlap scales with number  and 2) there are fewer instances of molecules shoved up against one another, which is where the overlap problems occur.  And 3) that the different shape/location of the repulsive wall explains why it works for lambda_theta = lambda_ref, but not lambda_theta =/= lambda_ref. I think that a little bit of this physical explanation would be useful. 
%RAM: I think the explanation is more helpful now
%MRS3: next sentence already was mentioned ( K_eff > 50).  
%Messerly et al.~report that MBAR-ITIC is reliable if $K_{\rm snaps}^{\rm eff} > 50$ \cite{Postdoc_1}.

Figure \ref{fig:Neff} demonstrates that $K_{\rm snaps}^{\rm eff}$ is typically much greater in the vapor phase $(K_{\rm snaps}^{\rm eff, vap})$ than in the liquid phase $(K_{\rm snaps}^{\rm eff, liq})$. Specifically, $K_{\rm snaps}^{\rm eff, vap} \gg 50$, while $K_{\rm snaps}^{\rm eff, liq} < 50$ when $\lambda_{\rm rr} \neq \lambda_{\rm ref}$ (TraPPE $\Leftrightarrow$ MiPPE-gen) and $K_{\rm snaps}^{\rm eff, liq} > 50$ for $\lambda_{\rm rr} = \lambda_{\rm ref}$ (MiPPE-gen $\Rightarrow$ MiPPE-SL and TraPPE $\Rightarrow$ NERD). Therefore, similar to the conclusions of Messerly et al.~\cite{Postdoc_1} for MBAR-ITIC, $K_{\rm snaps}^{\rm eff} > 50$ is a good indication that the GCMC-MBAR estimates are reliable. In comparison, MBAR-ITIC experiences even worse overlap at near-saturated liquid conditions ($\bar K_{\rm snaps, liq}^{\rm eff} \approx 1$) for TraPPE $\Rightarrow$ MiPPE-gen ($\lambda_{\rm rr} \neq \lambda_{\rm ref}$) \cite{Postdoc_1}.

The strong disparity between $K_{\rm snaps}^{\rm eff, vap}$ and $K_{\rm snaps}^{\rm eff, liq}$ when $\lambda_{\rm rr} \neq \lambda_{\rm ref}$ is expected for at least two reasons. The first reason is that the vapor phase has fewer molecules overall. $K_{\rm snaps}^{\rm eff}$ (or alternatively $W_n$) is greatest when the reduced potential energy ($u$ in Equation \ref{eq: reduced potential}) is similar for $\theta_{\rm rr}$ and $\theta_{\rm ref}$, such that the ratio of Boltzmann factors is close to unity. The reduced potential energy is proportional to the total internal energy $(U)$, which is an extensive property that depends on both $N$ and $\theta$. Therefore, even when $\theta_{\rm rr} \not\approx \theta_{\rm ref}$, small values of $N$ lead to $u(\theta_{\rm rr}) \approx u(\theta_{\rm ref})$. 

The second reason is that there are more instances of close-range interactions in the liquid phase. Varying $\lambda$ changes the slope and location of the repulsive wall, which greatly impacts the non-bonded energy at short distances (i.e., $r < \sigma$). These large changes in $U$ when $\lambda_{\rm rr} \neq \lambda_{\rm ref}$ cause $W_n \approx 0$ for the majority of liquid phase configurations. For similar reasons, the overlap in the liquid phase is markedly worse when $\sigma_{\rm rr} \not\approx \sigma_{\rm ref}$. For example, $K_{\rm snaps}^{\rm eff, liq}$ is significantly greater for MiPPE-gen $\Rightarrow$ MiPPE-SL than TraPPE $\Rightarrow$ NERD (see Figure \ref{fig:Neff}). 

%, smaller values of $N$  although large changes in $\theta$ the impact of varying $\theta$ is d have a smaller impact on $u$ (and, thereby, greater values of $W_n$ and $K_{\rm snaps}^{\rm eff}$) for larger values of $N$. fewer molecules lead to smaller absolute changes in $u$ when varying $\theta$. overlap is proportional to $\exp(-\beta N (U(\theta_{\rm rr}) - U(\theta_{\rm ref}))$, such that the overlap is less sensitive to changes in the non-bonded parameters.

% although typically $K_{\rm snaps}^{\rm eff, liq} > 50$ for both cases.

%Applying this heuristic to GCMC-MBAR helps qualify why the liquid properties are poorly estimated in some systems while the vapor properties are much more accurate.  This demonstrates that $K_{\rm snaps}^{\rm eff} > 50$ is a good indication that the GCMC-MBAR estimates are reliable.

% while  is much greater in the liquid phase than the vapor phase.  if the overlap between systems is sufficient for MBAR to be reliable.  is 

\begin{figure}[H]
	\centering
	\includegraphics[width=6.4in]{refFF_to_rrFF_Neff_alt.pdf}
	%MRS2: sometimes you say snapshots, and sometime samples.
	%RAM: I have corrected this to say snapshots throughout
	%MRS2: maybe put more of what you think readers should get from this.
	%RAM: The main point is that vapor has high overlap while liquid has low overlap and therefore MBAR is more reliable in vapor phase. I added one more sentence to convey the impact of overlap on accuracy of MBAR.
	\caption{Number of effective snapshots $(K_{\rm snaps}^{\rm eff})$ in the liquid (left panel) and vapor (right panel) phases. Good overlap $(K_{\rm snaps}^{\rm eff} \gg 50)$ is achieved in the vapor phase for each system while poor overlap in liquid phase $(K_{\rm snaps}^{\rm eff} < 50)$ is observed for $\lambda_{\rm rr} \neq \lambda_{\rm ref}$. The amount of overlap explains why GCMC-MBAR is highly reliable in both phases for  $\lambda_{\rm rr} = \lambda_{\rm ref}$ (see Figure \ref{fig:refFF_to_rrFF_lam_constant}), while GCMC-MBAR is not reliable in the liquid phase for $\lambda_{\rm rr} \neq \lambda_{\rm ref}$ (see Figure \ref{fig:refFF_to_rrFF_lam12to16}). Color scheme is the same as Figures \ref{fig:refFF_to_rrFF_lam_constant} and \ref{fig:refFF_to_rrFF_lam12to16}. Closed and open symbols correspond to $\lambda_{\rm rr} = \lambda_{\rm ref}$ and $\lambda_{\rm rr} \neq \lambda_{\rm ref}$, respectively.}
	\label{fig:Neff}
\end{figure} 

%%% Old
%The poor overlap in the liquid phase for $\lambda_{\rm rr} \neq \lambda_{\rm ref}$ is consistent with the MBAR-ITIC results \cite{Postdoc_1}. By contrast, GCMC-MBAR provides considerable improvement in predicting $\rho_{\rm vap}^{\rm sat}$ and $P_{\rm vap}^{\rm sat}$ for $\lambda_{\rm rr} \neq \lambda_{\rm ref}$ compared to what was previously observed for MBAR-ITIC. Therefore, similar to MBAR-ITIC, optimizing Mie $\lambda$-6 parameters requires performing direct simulations for each value of $\lambda$. However, as opposed to MBAR-ITIC, GCMC-MBAR can help determine where these additional simulations should be performed, i.e., GCMC-MBAR can localize the likely optimal $\epsilon$ and $\sigma$ values for $\lambda_{\rm rr} \neq \lambda_{\rm ref}$.

Although GCMC-MBAR is unreliable for $\rho_{\rm liq}^{\rm sat}$ and $\Delta H_{\rm v}$ when $\lambda_{\rm rr} \neq \lambda_{\rm ref}$, GCMC-MBAR provides considerable improvement in predicting $\rho_{\rm vap}^{\rm sat}$ and $P_{\rm vap}^{\rm sat}$ when $\lambda_{\rm rr} \neq \lambda_{\rm ref}$ compared to what was previously observed for MBAR-ITIC.\cite{Postdoc_1} Therefore, similar to MBAR-ITIC, optimizing Mie $\lambda$-6 parameters requires performing direct simulations for each value of $\lambda$. However, as opposed to MBAR-ITIC, GCMC-MBAR can help determine where these additional simulations should be performed, i.e., GCMC-MBAR can localize the likely optimal $\epsilon$ and $\sigma$ values for $\lambda_{\rm rr} \neq \lambda_{\rm ref}$.


%   and then GCMC-MBAR can opt region for various values  obtain   explore regions of parameter space where $\theta_{\rm rr} \not\approx \theta_{\rm ref}$

We recommend the following algorithm for optimizing Mie $\lambda$-6 parameters with GCMC-MBAR:
\begin{enumerate}
	\item Simulate initial reference force field $(\theta^{\langle0\rangle}_{\rm ref})$, e.g., TraPPE \label{step: Sim ref}
	\item Compute basis functions for various values of $\lambda$ \label{step: Basis Functions}
%	\item Optimize $\epsilon$ and $\sigma$ for each $\lambda$ $(\epsilon_{\rm opt}^\lambda$, $\sigma_{\rm opt}^\lambda$) by: \label{step: Opt eps,sig}
    \item Optimize $\epsilon$ and $\sigma$ for each $\lambda$ $(\epsilon^{\langle i \rangle}$, $\sigma^{\langle i \rangle}$) by: \label{step: Opt eps,sig}
	\begin{enumerate}
		\item Estimating $\rho_{\rm liq}^{\rm sat}$, $\rho_{\rm vap}^{\rm sat}$, $P_{\rm vap}^{\rm sat}$, and $\Delta H_{\rm v}$ with GCMC-MBAR \label{step: GCMC-MBAR}
		\item Minimizing scoring function $(S)$ \label{step: S min}
	\end{enumerate}
%	\item Determine optimal $\epsilon$ and $\sigma$ for various values of $\lambda$ by minimizing $S$ \label{Opt eps,sig}
	\item Perform additional simulations with $\epsilon^{\langle i \rangle}$, $\sigma^{\langle i \rangle}$ from Step \ref{step: Opt eps,sig} for each $\lambda$ \label{step: Resim opt}
	% each optimal $\epsilon$, $\sigma$, $\lambda$ \label{step: Resim opt}
%	\item Determine overall optimal parameter set $(\theta_{\rm opt})$ by: \label{step: Opt eps,sig,lam}
%	\begin{enumerate}
%		\item Estimating $\rho_{\rm liq}^{\rm sat}$, $\rho_{\rm vap}^{\rm sat}$, $P_{\rm vap}^{\rm sat}$, and $\Delta H_{\rm v}$ with GCMC-MBAR for $\lambda_{\rm rr} = \lambda_{\rm ref}$
%		\item Minimizing scoring function $(S)$
%	\end{enumerate}
%	\item Repeat Steps \ref{step: Resim opt} and \ref{step: Opt eps,sig,lam} until $\min(K_{\rm snaps}^{\rm eff}) \gg 50$ for $\theta_{\rm opt}$ 
	\item Repeat Steps \ref{step: Opt eps,sig} and \ref{step: Resim opt} until $\min(K_{\rm snaps}^{\rm eff}) \gg 50$ for $\theta^{\langle i \rangle}$
	\item Determine overall optimal parameter set $(\theta_{\rm opt})$ \label{step: Opt eps,sig,lam}
\end{enumerate}
Note that Step \ref{step: Basis Functions} is optional in that GCMC-MBAR (Step \ref{step: GCMC-MBAR}) does not require basis functions. Because the Mie $\lambda$-6 potential is amenable to basis functions, however, we implement Step \ref{step: Basis Functions} to reduce the cost of recomputing the energy for $\theta_{\rm rr}$.  

The stopping criterion is based on the minimum number of effective snapshots $\min(K_{\rm snaps}^{\rm eff})$, which always corresponds to the liquid phase at lower values of $T^{\rm sat}$ (see Figure \ref{fig:Neff}). Therefore, GCMC-MBAR estimates should be reliable over the entire range of saturation temperatures when $\min(K_{\rm snaps}^{\rm eff}) \gg 50$. If this is true for the optimal region of parameter space, any additional iterations would be ill-advised and unnecessary as the optimization would primarily be fitting to the uncertainty in the simulation output. Section \ref{sec: Case study} presents an application of this approach for cyclohexane.
%where $\theta_{\rm ref}$ in Step \ref{step: Sim ref} is 


% Section \ref{sec: Case study} demonstrates The improved reliability of GCMC-MBAR allows for  Because accurate estimates of $\rho_{\rm liq}^{\rm sat}$ and $\Delta H_{\rm v}$ are important for optimizing the non-bonded Mie $\lambda$-6 parameters, we recommend performing direct simulations for each value of $\lambda$.  Section \ref{sec: Case study} demonstrates that  
%
%Therefore, it is important to perform simulations for each value of $\lambda$ when accurate estimates  predicting $\rho_{\rm liq}^{\rm sat}$ and $\Delta H_{\rm v}$.  Therefore, when re-parameterizing a Lennard-Jones 12-6 potential to a Mie $\lambda$-6 potential, we recommend performing additional simulations for each value of $\lambda$. The  force field 

%RAM: This is no longer our recommendation. You can keep the scoring function the same. You will just want to include direct simulations for each lambda after determining the semi-optimal region.
% For this reason, we recommend reducing the $\rho_{\rm liq}^{\rm sat}$ and $\Delta H_{\rm v}$ weights $(w_0$, $w_3$, $w_4$, and $w_7)$ in the scoring function (Equation \ref{eq: Score}) when varying $\lambda$. The degree to which the weight is reduced should depend on the number of effective samples $(K_{\rm snaps}^{\rm eff})$.

\subsection{Case Study: Post-simulation optimization of cyclohexane Mie $\lambda$-6 parameters} \label{sec: Case study}

%We have demonstrated that GCMC-MBAR is accurate when $K_{\rm snaps}^{\rm eff} \gg 50$, which is typically the case in the vapor phase and in the liquid phase when $\lambda_{\rm rr} = \lambda_{\rm ref}$. 

%In this section, we present how GCMC-MBAR can rapidly convert a pre-tuned Lennard-Jones 12-6 potential (TraPPE) into a Mie $\lambda$-6 potential (MiPPE) without performing hundreds of simulations. 

%optimize the Mie $\lambda$-6 parameters when starting with a Lennard-Jones 12-6 potential (namely, TraPPE). 

%This section has shown that GCMC-MBAR is capable of converting a pre-tuned Lennard-Jones 12-6 potential (TraPPE) into a Mie $\lambda$-6 potential (MiPPE) without performing hundreds of simulations. Specifically, with only two stages of direct simulation we considered hundreds of $\epsilon_{\rm CH_2}$, $\sigma_{\rm CH_2}$, and $\lambda_{\rm CH_2}$ parameter sets.

In this section, we present how GCMC-MBAR can rapidly convert a pre-tuned Lennard-Jones 12-6 potential (TraPPE) into a Mie $\lambda$-6 potential (MiPPE) without performing hundreds of simulations. We have chosen cyclohexane for this case study as MiPPE does not yet have non-bonded parameters for this compound, while the TraPPE force field does. Also, because cyclohexane consists of a single united-atom site type (CH$_2)$, it is a convenient molecule for representing the scoring function in 2-dimensions ($\epsilon_{\rm CH_2}$ and $\sigma_{\rm CH_2}$ for a given value of $\lambda_{\rm CH_2}$). 

The scoring function is computed with the branched alkane weights $(w_{x})$ and REFPROP correlations\cite{LEMMON-RP10,Zhou2014} as target data $(X_{\rm exp})$. To avoid finite size effects in the near critical region, data are excluded for $T^{\rm sat} > 0.95 T_{\rm c}$. Specifically, $X_{\rm exp}$ consists of REFPROP $\rho_{\rm liq}^{\rm sat}$, $\rho_{\rm vap}^{\rm sat}$, $P_{\rm vap}^{\rm sat}$, and $\Delta H_{\rm v}$ values from 360 K to 520 K with 5 K intervals.

Figure \ref{fig:Score_CYC6} depicts the scoring function for the first iteration of the optimization, where GCMC simulations were performed with the TraPPE parameters (depicted as a white triangle in Figures \ref{fig:Score_CYC6} and \ref{fig:Neff_CYC6}). Similar figures have been reported in the literature using GCMC-HR \cite{Potoff_branched,Barhaghi2017}. The key difference is that the heat maps reported in the literature were obtained by performing GCMC simulations with hundreds of different parameter sets. By contrast, the results shown in Figure \ref{fig:Score_CYC6} were obtained by performing GCMC simulations with a single parameter set, namely, the TraPPE parameters. MBAR reweights these same configurations for all other parameter sets. Furthermore, $U(\theta_{\rm rr})$ is computed with basis functions, enabling the GCMC-MBAR recompute step to be extremely fast. 

%, where red denotes the optimal parameter set (i.e., the lowest values of $S$.)

%each proposed parameter set could be simulated directly in parallel.
% direct GCMC simulations could be run in parallel for each proposed parameter set.
%  and, therefore, the real time to solution would be comparable for the GCMC-MBAR and GCMC-HR. Although GCMC-MBAR still reduces the total CPU time compared with GCMC-HR, this optimization scheme negates the reduction in real time required by GCMC-MBAR .

%which would negate the computational gain of GCMC-MBAR compared with GCMC-HR.

Note that the TraPPE force field utilizes a Lennard-Jones 12-6 potential (i.e., $\lambda_{\rm TraPPE} = 12$) and, therefore, the results in the top-left panel of Figures \ref{fig:Score_CYC6} and \ref{fig:Neff_CYC6} are for the case where $\lambda_{\rm rr} = \lambda_{\rm ref} = 12$, while the other panels correspond to $\lambda_{\rm rr} \neq \lambda_{\rm ref}$. Figure \ref{fig:Neff_CYC6} shows that, as expected, the average number of effective snapshots in the liquid phase ($\bar K_{\rm snaps, liq}^{\rm eff}$) is much greater for $\lambda_{\rm rr} = \lambda_{\rm ref}$ than for $\lambda_{\rm rr} \neq \lambda_{\rm ref}$. 

When $\lambda_{\rm rr} = \lambda_{\rm ref}$, MBAR-ITIC is reliable ($\bar K_{\rm snaps, liq}^{\rm eff} > 50$) for $\sigma_{\rm rr} = \sigma_{\rm ref} \pm 0.0025$ nm \cite{Postdoc_1}, while Figure \ref{fig:Neff_CYC6} (top left panel) suggests that GCMC-MBAR is reliable over a much wider range ($\sigma_{\rm rr} \approx \sigma_{\rm ref} \pm 0.01$ nm). Although the overlap is significantly better for GCMC-MBAR compared to MBAR-ITIC for $\lambda_{\rm rr} \neq \lambda_{\rm ref}$, Figure \ref{fig:Neff_CYC6} demonstrates that GCMC-MBAR and MBAR-ITIC follow a similar trend, namely, the high $\bar K_{\rm snaps, liq}^{\rm eff}$ region corresponds to $\sigma_{\rm rr} < \sigma_{\rm ref}$ when $\lambda_{\rm rr} > \lambda_{\rm ref}$ and $\epsilon_{\rm rr} > \epsilon_{\rm ref}$. This poses a challenge for parameterization because the optimal $\sigma_{\rm CH_2}$ value is fairly constant while the optimal $\epsilon_{\rm CH_2}$ value tends to increase with respect to $\lambda_{\rm CH_2}$.

%In comparison, MBAR-ITIC experiences even worse overlap at near-saturated liquid conditions for $\lambda_{\rm rr} \neq \lambda_{\rm ref}$ (Messerly et al. report $\bar K_{\rm snaps, liq}^{\rm eff} \approx 1$) \cite{Postdoc_1}.

In addition, the smooth contours in Figure \ref{fig:Score_CYC6} for $\lambda_{\rm rr} = \lambda_{\rm ref} = 12$ and the wide range of parameters over which $\bar K_{\rm snaps}^{\rm eff, liq} \gg 50$ suggests that GCMC-MBAR is highly reliable for optimizing $\epsilon_{\rm CH_2}$ and $\sigma_{\rm CH_2}$ for a fixed value of $\lambda_{\rm CH_2}$. Even more remarkable, considering the erroneous prediction of $\rho_{\rm liq}^{\rm sat}$ and $\Delta H_{\rm v}$ for $\lambda_{\rm rr} \neq \lambda_{\rm ref}$ (see Figure \ref{fig:refFF_to_rrFF_lam12to16}), is that GCMC-MBAR predicts smooth contours for $\lambda \neq 12$ despite lower values of $\bar K_{\rm snaps}^{\rm eff, liq}$. This characteristic would be important when implementing gradient descent optimization schemes. By contrast, MBAR-ITIC yields sporadic contours for $\lambda_{\rm rr} \neq \lambda_{\rm ref}$ that are unreliable for optimization \cite{Postdoc_1}. 

%, which are completely useless for     MBAR-ITIC 

%MRS2: might not be as clear to others why remarkable.

%MRS3: above - hmm. Why does it give smooth contours even though it is looking at lambda=/= 12 values?  I missed why it's working here. I see in figure 7 that the number of effective samples is larger than in the previous tests, but is that just because in the previous tests, we were sampling over very different parameters, and here, sigma and epsilon aren't that different in the range of interest?
%RAM: I think it is more that all eps/sig are penalized equally as they all have fairly random values of rholiqsat. But I really don't have a great explanation for this.
%MRS3: can't you just say ``is important'' rather than ``would be''?
%RAM: "is important" seems to imply that we actually utilized a gradient descent optimization. "Would be" suggests that this applies to other future studies.
	\begin{figure}[htb!]
		\centering
		\includegraphics[width=6.4in]{CYC6_scoring_function_lam.pdf}
%MRS2: the upper label (lambda_CH2 = 12. etc) is bleeding into the top of the figure. Also next figure. 
%RAM: Fixed
		\caption{First iteration scoring function values with respect to $\epsilon_{\rm CH_2}$ and $\sigma_{\rm CH_2}$ for cyclohexane. GCMC-MBAR enables rapid optimization of Mie $\lambda$-6 parameters from a single reference force field $(\theta^{\langle0\rangle}_{\rm ref} = \theta_{\rm TraPPE}$, depicted as a white triangle). Top-left, top-right, bottom-left, and bottom-right panels correspond $\lambda_{\rm CH_2} = 12$, $\lambda_{\rm CH_2} = 14$, $\lambda_{\rm CH_2} = 16$, $\lambda_{\rm CH_2} = 18$, respectively. White ``X''s represent the optimal parameter sets (the lowest value of $S$) for each $\lambda_{\rm CH_2}$.} %Red denotes the optimal parameter set, i.e., the lowest value of $S$.
		\label{fig:Score_CYC6}
	\end{figure} 

	\begin{figure}[htb!]
		\centering
		\includegraphics[width=6.4in]{CYC6_Neff_lam.pdf}
		\caption{First iteration average number of effective snapshots in the liquid phase $(\bar K_{\rm snaps}^{\rm eff, liq})$ with respect to $\epsilon_{\rm CH_2}$ and $\sigma_{\rm CH_2}$ for cyclohexane. $\bar K_{\rm snaps}^{\rm eff, liq} \gg 50$ over a wide range of parameters when $\lambda_{\rm rr} = \lambda_{\rm ref} = 12$ (top-left panel), while $\bar K_{\rm snaps}^{\rm eff, liq}$ is typically less than $50$ for $\lambda_{\rm rr} \neq \lambda_{\rm ref}$ (other panels). Top-left, top-right, bottom-left, and bottom-right panels correspond $\lambda_{\rm CH_2} = 12$, $\lambda_{\rm CH_2} = 14$, $\lambda_{\rm CH_2} = 16$, $\lambda_{\rm CH_2} = 18$, respectively. Symbols are the same as Figure \ref{fig:Score_CYC6}.}% represent the optimal parameter set, i.e., the lowest value of $S$, for a given $\lambda_{\rm CH_2}$.}
		\label{fig:Neff_CYC6}
	\end{figure}

The optimal parameter sets (depicted as white ``X''s in Figures \ref{fig:Score_CYC6} and \ref{fig:Neff_CYC6}) correspond to the lowest value of $S$ for each value of $\lambda_{\rm CH_2}$. Having minimized the scoring function for $\lambda_{\rm CH_2} =$ 12, 14, 16, 18, and 20 (not shown in Figure \ref{fig:Score_CYC6}), we perform additional simulations with the optimal parameter sets ($\theta^{\langle1\rangle}$) serving as new reference parameter sets. Because GCMC-MBAR is reliable over a wide range of $\epsilon_{\rm CH_2}$ and $\sigma_{\rm CH_2}$ values for $\lambda_{\rm rr} = \lambda_{\rm ref}$, we only reweight snapshots generated with the same value of $\lambda_{\rm CH_2}$. Also, because the optimal $\epsilon_{\rm CH_2}$ and $\sigma_{\rm CH_2}$ parameters for $\lambda_{\rm CH_2} = 12$ already have $\min(K_{\rm snaps}^{\rm eff}) \gg 50$, we only consider $\lambda_{\rm CH_2} \ge 14$.

% $\lambda_{\rm CH_2} = 14$ to estimate properties for $\lambda_{\rm CH_2} \neq 14$
%MRS3: do you use all previous simulations with different parameters as reference, or just a single reference? Will be slightly better, though I don't know how much.
%RAM: Just a single reference, including multiple references seemed unnecessary as we already have very high Keff. This is discussed in the discussion section

Figure \ref{fig:Iterate_Score_CYC6} presents the results from this second iteration of the two-dimensional optimization for each $\lambda_{\rm CH_2}$. Also depicted in the $\lambda_{\rm CH_2} = 16$ panel are the TAMie cyclohexane parameters \cite{Weidler2016}. Note that the first iteration optimal parameter sets (white ``X''s) are similar to the second iteration (white ``+''s and star), which provides further evidence that the first iteration results for $\lambda_{\rm rr} \neq \lambda_{\rm ref}$ are quite reliable. We verify that the optimization has converged, i.e., $\min(K_{\rm snaps}^{\rm eff}) \gg 50$ for $\theta_{\rm opt}^{\langle2\rangle}$ (see Figure \ref{SI fig:Iterate_Neff_CYC6} in Supporting Information).  

% there is sufficient overlap $(\min(K_{\rm snaps}^{\rm eff}) \gg 50)$ between the first and second iterations to consider the optimization converged.

	\begin{figure}[htb!]
		\centering
		\includegraphics[width=6.4in]{CYC6_scoring_function_lam_iteration.pdf}
		\caption{Second iteration scoring function values with respect to $\epsilon_{\rm CH_2}$ and $\sigma_{\rm CH_2}$ for cyclohexane. Pseudo-optimal parameter sets from first iteration serve as reliable reference parameter sets for refining the optimization. Top-left, top-right, bottom-left, and bottom-right panels correspond $\lambda_{\rm CH_2} = 14$, $\lambda_{\rm CH_2} = 16$, $\lambda_{\rm CH_2} = 18$, and $\lambda_{\rm CH_2} = 20$, respectively. White star represents the overall optimal parameter set (MiPPE: $\theta^{\langle2\rangle}, \lambda_{\rm CH_2} = 16$), white ``+''s correspond to optimal parameter sets ($\theta^{\langle2\rangle}$) for $\lambda_{\rm CH_2} \neq 16$, and the white diamond is the TAMie parameter set (for $\lambda_{\rm CH_2} = 16$). White ``X''s depict the single reference force field for each $\lambda_{\rm CH_2}$ (i.e., $\lambda_{\rm rr} = \lambda_{\rm ref}$) and are the same as Figures \ref{fig:Score_CYC6} and \ref{fig:Neff_CYC6}.} %, i.e., $\lambda_{\rm rr} = \lambda_{\rm ref}$.} 
		\label{fig:Iterate_Score_CYC6}
	\end{figure} 

%\begin{figure}[htb!]
%	\centering
%%	\includegraphics[width=6.4in]{CYC6_Neff_lam.pdf}
%	\caption{Average number of effective snapshots in the liquid phase $(\bar K_{\rm snaps}^{\rm eff, liq})$ with respect to $\epsilon_{\rm CH_2}$ and $\sigma_{\rm CH_2}$ for cyclohexane. $K_{\rm snaps}^{\rm eff, liq} \gg 50$ for the optimal $\epsilon$ and $\sigma$ parameter set for each $\lambda$. Top-left, top-right, bottom-left, and bottom-right panels correspond $\lambda_{\rm CH_2} = 14$, $\lambda_{\rm CH_2} = 16$, $\lambda_{\rm CH_2} = 18$, and $\lambda_{\rm CH_2} = 12$, respectively. White star represents the optimal parameter set, i.e., the lowest value of $S$, for a given $\lambda_{\rm CH_2}$}
%	\label{fig:Iterate_Neff_CYC6}
%\end{figure}

%	\begin{figure}[htb!]
%		\centering
%			\includegraphics[width=6.4in]{CYC6_min_Neff_lam_iteration.pdf}
%		\caption{Minimum number of effective snapshots $(\min(K_{\rm snaps}^{\rm eff}))$ with respect to $\epsilon_{\rm CH_2}$ and $\sigma_{\rm CH_2}$ for cyclohexane. Optimization has converged as $\min(K_{\rm snaps}^{\rm eff}) \gg 50$ for the optimal $\epsilon_{\rm CH_2}$, $\sigma_{\rm CH_2}$, $\lambda_{\rm CH_2}$ parameter set. Top-left, top-right, bottom-left, and bottom-right panels correspond $\lambda_{\rm CH_2} = 14$, $\lambda_{\rm CH_2} = 16$, $\lambda_{\rm CH_2} = 18$, and $\lambda_{\rm CH_2} = 12$, respectively. White star represents the optimal parameter set, i.e., the lowest value of $S$, for a given $\lambda_{\rm CH_2}$}
%		\label{fig:Iterate_Neff_CYC6}
%	\end{figure}

    \begin{table}[h!]
%MRS3: maybe indicate which one of the 5 is the optimal in the table?  Perhaps give final scoring function values in the table?
		\caption{Optimal Mie $\lambda$-6 cyclohexane parameters for $\lambda_{\rm CH_2} =$ 12, 14, 16, 18, and 20. Superscript denotes the iteration stage of the optimization. Stage 0 corresponds to the TraPPE force field and Stage 2 for $\lambda_{\rm CH_2} =$ 16 is the MiPPE force field. Final column reports the optimal scoring function $(S_{\rm opt})$ for each $\lambda_{\rm CH_2}$ (computed with $\theta^{\langle1\rangle}$ for $\lambda_{\rm CH_2} =$ 12 and with $\theta^{\langle2\rangle}$ for $\lambda_{\rm CH_2} =$ 14, 16, 18, and 20).} \label{tab:lam opt}
		\begin{center}
			\begin{tabular}{|c|c|c|c|c|c|c|c|}
				\hline
				$\lambda_{\rm CH_2}$ & $\epsilon_{\rm CH_2}^{\langle0\rangle}/k_{\rm B}$ (K) & $\sigma_{\rm CH_2}^{\langle0\rangle}$ (nm) & $\epsilon_{\rm CH_2}^{\langle1\rangle}/k_{\rm B}$ (K) & $\sigma_{\rm CH_2}^{\langle1\rangle}$ (nm) & $\epsilon_{\rm CH_2}^{\langle2\rangle}/k_{\rm B}$ (K) & $\sigma_{\rm CH_2}^{\langle2\rangle}$ (nm) & $S_{\rm opt}$ \\ \hline
				12 & 52.5 & 0.391 & 53.0 & 0.394 & -- & -- & 1.79 \\ 
				14 & -- & -- & 61.5 & 0.393 & 61.5 & 0.393 & 1.03 \\ 
				16 & -- & -- & 70.0 & 0.389 & 69.7 & 0.3902 & 0.463 \\
				18 & -- & -- & 77.0 & 0.389 & 76.5 & 0.390 & 0.791 \\
				20 & -- & -- & 84.0 & 0.388 & 82.5 & 0.389 & 1.07 \\
				\hline
			\end{tabular}
		\end{center} 
	\end{table}

%MRS3: I don't understand how this indicates precision below?

Table \ref{tab:lam opt} summarizes the parameter sets obtained at each stage of the optimization. Table \ref{tab:lam opt} shows that, similar to other alkanes in the MiPPE force field, $\lambda_{\rm CH_2} = 16$ is once again found to be the optimal repulsive exponent (with the lowest $S$) when only considering even integer values. This optimal $\lambda_{\rm CH_2} = 16$ value also agrees with the TAMie force field \cite{Weidler2016}. The overall optimal parameter set $(\theta^{\langle2\rangle}$, $\lambda_{\rm CH_2} = 16$) is included in Table \ref{tab:nonbonded params} as the MiPPE cyclohexane parameters. $\epsilon_{\rm CH_2}$ and $\sigma_{\rm CH_2}$ are reported with three and four digits, respectively, consistent with other MiPPE parameters and to provide a qualitative measure of uncertainty. Note the close agreement between the MiPPE and TAMie parameters ($\epsilon_{\rm CH_2}/k_{\rm B} = 69.568$ K, $\sigma_{\rm CH_2} = 0.38967$ nm, and $\lambda_{\rm CH_2} = 16$), which were optimized with a slightly different objective function and experimental data set. 

% which were optimized based on a relative least-squares objective function that depended on different quasi-experimental data (from ) \cite{DIPPR} developed by a weighted least-squares optimization using an objective  results of Weidler et al. for the TAMie potential.) 

It is important to verify that the final parameter set does indeed provide accurate predictions of vapor-liquid coexistence properties. For this reason, after completing the optimization, we perform direct GCMC simulations with the overall optimal parameter set (MiPPE). Twenty independent replicate simulations are performed at each state point ($\mu$, $V$, $T$) to reduce and rigorously quantify the statistical uncertainties for the MiPPE cyclohexane results.

Furthermore, because we have performed the MiPPE parameterization with a relatively small box size (27 nm$^3$), it is important to test for the existence of finite-size effects. By performing additional simulations with a larger box size (42.875 nm$^3$), we conclude that finite-size effects are only significant (larger than the combined uncertainties) for $\rho_{\rm vap}^{\rm sat}$ and $\Delta H_{\rm v}$ near the critical point. Specifically, we estimate that finite-size effects for both $\rho_{\rm vap}^{\rm sat}$ and $\Delta H_{\rm v}$ are between 1\% and 2\% for $T^{\rm sat} > 480$ K (see Section \ref{SI sec: finite-size effects} in Supporting Information).
 
Figure \ref{fig: VLE cyclohexane} is provided to quantify the improved accuracy achieved for the two iterations by comparing the percent deviations for between pseudo-experimental (REFPROP) values and the zeroth iteration (TraPPE: $\theta^{\langle0\rangle}, \lambda_{\rm CH_2} = 12$), first iteration ($\theta^{\langle1\rangle}$ for $\lambda_{\rm CH_2} = 14, 16,$ and $18$), and final iteration (MiPPE: $\theta^{\langle2\rangle}, \lambda_{\rm CH_2} = 16$) parameter sets. Figure \ref{fig: VLE cyclohexane} also includes percent deviations for several of the most reliable force fields from the literature.\cite{Errington1999,Eckl2008,Bourasseau2002CYC6,Mauricio2015,Keasler2012,Yiannourakou2019,Weidler2016} Additional phase equilibria and deviation plots are provided in Section \ref{SI sec: Case study} of Supporting Information, including a detailed comparison between the MiPPE and TAMie force fields.

%$\rho_{\rm liq}^{\rm sat}$, $\rho_{\rm vap}^{\rm sat}$, $P_{\rm vap}^{\rm sat}$, and $\Delta H_{\rm v}$

%is clearly the optimal parameter set as it is the most accurate for $\rho_{\rm vap}^{\rm sat}$ and $P_{\rm vap}^{\rm sat}$ with similar performance for $\rho_{\rm liq}^{\rm sat}$ and $\Delta H_{\rm v}$ as the first iteration.

	\begin{figure}[H]
		\centering
		\includegraphics[width=5.8in]{CYC6_deviations_iterations.pdf}
		\caption{Percent deviations relative to REFPROP cyclohexane values\cite{Zhou2014} for MiPPE ($\theta^{\langle2\rangle}, \lambda_{\rm CH_2} = 16$), zeroth iteration (TraPPE: $\theta^{\langle0\rangle}$), first iterations ($\theta^{\langle1\rangle}, \lambda_{\rm CH_2} = 14, 16, 18$), and several literature force fields \cite{Errington1999,Eckl2008,Bourasseau2002CYC6,Mauricio2015,Keasler2012,Yiannourakou2019,Weidler2016}. The first iteration parameter sets provide significant improvement compared to the zeroth iteration, while the second iteration (MiPPE) performs comparably to the most accurate literature force fields. Top-left, top-right, bottom-left, and bottom-right panels correspond to saturated liquid density, saturated vapor density, saturated vapor pressure, and enthalpy of vaporization, respectively. MiPPE uncertainties (95\% confidence intervals) are obtained from 20 independent replicates. For clarity, the uncertainties for $\theta^{\langle0\rangle}$, $\theta^{\langle1\rangle}$, and Errington et al.~\cite{Errington1999}are omitted.}
		\label{fig: VLE cyclohexane}
	\end{figure}

%	\begin{figure}[H]
%		\centering
%		\includegraphics[width=5.8in]{CYC6_deviations_iterations.pdf}
%		\caption{Percent deviations relative to REFPROP cyclohexane values\cite{Zhou2014} for MiPPE ($\theta^{\langle2\rangle}, \lambda_{\rm CH_2} = 16$), zeroth iteration (TraPPE: $\theta^{\langle0\rangle}$), first iterations ($\theta^{\langle1\rangle}, \lambda_{\rm CH_2} = 14, 16, 18$), and several literature force fields \cite{Errington1999,Eckl2008,Bourasseau2002CYC6,Mauricio2015,Keasler2012,Yiannourakou2019}. The first iteration parameter sets provide significant improvement compared to the zeroth iteration, while the second iteration (MiPPE) performs comparably to the most accurate literature force fields. Top-left, top-right, bottom-left, and bottom-right panels correspond to saturated liquid density, saturated vapor density, saturated vapor pressure, and enthalpy of vaporization, respectively. MiPPE uncertainties (95\% confidence intervals) are obtained from 20 independent replicates. For clarity, the uncertainties for $\theta^{\langle0\rangle}$, $\theta^{\langle1\rangle}$, and Errington et al.~\cite{Errington1999}are omitted, while Keasler et al.~\cite{Keasler2012}did not report uncertainties.}
%		\label{fig: VLE cyclohexane}
%	\end{figure}

Even the first iteration parameters demonstrate considerable improvement compared to the zeroth iteration (TraPPE) for predicting $\rho_{\rm vap}^{\rm sat}$ and $P_{\rm vap}^{\rm sat}$ without significantly diminishing the accuracy for $\rho_{\rm liq}^{\rm sat}$ and $\Delta H_{\rm v}$. In fact, the first iteration $\lambda_{\rm CH_2} = 16$ parameter set achieves similar deviations as the TAMie force field. The second (final) iteration provides further improvement in each property compared to the first iteration. In comparison with literature force fields, MiPPE (and TAMie) are arguably the most accurate at predicting $\rho_{\rm vap}^{\rm sat}$, $P_{\rm vap}^{\rm sat}$, and $\Delta H_{\rm v}$. With the exception of the Exponential-6, \cite{Exp6} MiPPE provides similar accuracy for $\rho_{\rm liq}^{\rm sat}$ as the other literature force fields. 

%Section \ref{SI sec: TAMie comparison} of Supporting Information examines the close agreement between the MiPPE and TAMie phase equilibria.

%MRS2: not sure why this negates MBAR computational advantage, since you still need to run less simulations with MBAR?  Would not help the wall clock problem, but would reduce the total CPU time? Unclear paragraph.
%RAM: Hopefully it is clear that we are referring to the real time to solution
In this optimization example (an exhaustive 2-dimensional grid search over even integer values of $\lambda_{\rm CH_2}$), each proposed parameter set could be simulated directly in parallel. Therefore, the real time to solution for GCMC-MBAR would be similar to that of the GCMC-HR approach utilized by Mick et al.~\cite{Potoff_branched}and Soroush Barhaghi et al.~\cite{Barhaghi2017}, although GCMC-MBAR would still reduce the total CPU time. In general, however, higher dimensional optimization algorithms are performed in sequence, where each iteration proposes new parameter set(s). In this scenario, GCMC-MBAR (with basis functions) is orders of magnitude faster than the literature GCMC-HR approach, which would require performing new GCMC simulations for each iteration.

While a higher dimensional parameterization would necessitate a more sophisticated optimization scheme (Step \ref{step: S min}), the GCMC-MBAR analysis (Step \ref{step: GCMC-MBAR}) would be unchanged. If implementing basis functions with GCMC-MBAR (Step \ref{step: Basis Functions}), however, recomputing the energy for $\theta_{\rm rr}$ requires an attractive and repulsive basis function for all $ii$ (same) and $ij$ (cross) pair interaction sets. For example, propane would require 6 basis functions (an attractive and repulsive basis function for the CH$_3$-CH$_3$, CH$_3$-CH$_2$, and CH$_2$-CH$_2$ interactions). Although the memory requirement scales linearly as the number of basis functions, the storage load should still be manageable and significantly less compared to storing configurations. Furthermore, although generating these additional basis functions also requires a larger number of ``rerun'' calculations (recall Section \ref{sec: Basis functions}), the increase in computational cost is negligible compared to performing direct GCMC simulations with each proposed parameter set.

%For example, simultaneously optimizing the Mie $\lambda$-6 parameters for the two united-atom site types in propane (CH$_3$ and CH$_2$) would require both an attractive and repulsive basis function for each of the CH$_3$-CH$_3$, CH$_3$-CH$_2$, and CH$_2$-CH$_2$ interactions (6 basis functions in total).

% overhead for generating additional basis functions is negligible compared to performing rerun calculations for every proposed parameter set.

%marginal.

%remains approximately the same.  

%In addition to requiring a more sophisticated optimization scheme, a higher dimensional parameterization would also necessitate generating more basis functions. Specifically, computing the overall system energy with basis functions requires an attractive and repulsive basis function for all $ii$ (same) and $ij$ (cross) pair interaction sets. For example, simultaneous optimization of the three united-atom site types in 2-methylbutane (CH$_3$, CH$_2$, and CH) would require both an attractive and repulsive basis function for each of the CH$_3$-CH$_3$, CH$_3$-CH$_2$, CH$_3$-CH, CH$_2$-CH$_2$, CH$_2$-CH, and CH-CH interactions. The additional computational overhead to compute more basis functions, however, is marginal.

% with Percent deviations in cyclohexane property values between REFPROP, MiPPE, and s force field as well as literature force field values.
%  computed with GCMC-MBAR (for $\theta_{\rm rr} = \theta_{\rm ref}$).

%Uncertainties (95\% confidence intervals) for $\theta^{\langle0\rangle}$ and $\theta^{\langle1\rangle}$ are obtained with bootstrap re-sampling of a single set of GCMC simulations, while uncertainties for the MiPPE force field ($\theta^{\langle2\rangle}$) are obtained from 20 independent replicates.

%Keasler et al.~did not provide uncertainties. Yiannourakou et al. obtained uncertainties from block averaging, although it is unclear what type of uncertainties are reported.

%Uncertainties from this study represent 95\% confidence intervals, but are not depicted when less than one symbol size.
%When not depicted, uncertainties from this study are smaller than one symbol size.

%TraPPE uncertainties are not provided by Keasler et al. Yiannourakou et al. obtained uncertainties for TraPPE using block averaging, but it is unclear what type of uncertainties are reported, i.e., standard error or 95\% confidence intervals. Uncertainties for reference force fields (including TraPPE) are 95\% confidence intervals obtained with bootstrap re-sampling of a single set of GCMC simulations. Uncertainties for the MiPPE force field are 95\% confidence intervals from 20 independent replicate sets of GCMC simulations.

%This section has shown that GCMC-MBAR is capable of converting a pre-tuned Lennard-Jones 12-6 potential (TraPPE) into a Mie $\lambda$-6 potential (MiPPE) without performing hundreds of simulations. Specifically, with only two stages of direct simulation we considered hundreds of $\epsilon_{\rm CH_2}$, $\sigma_{\rm CH_2}$, and $\lambda_{\rm CH_2}$ parameter sets.

% simulating hundreds of different parameter sets

%\begin{enumerate}
%	\item We validate that MBAR and HR are statistically indistinguishable with sufficient data by re-analyzing the simulation results of Mick et al. and Barhaghi et al. utilizing MBAR
%	%	\begin{enumerate}
%	%		\item Evaluate all of the compounds that Mohammad has U and N values for (branched alkanes and alkynes) and which have good experimental data
%	%		\item Compare MBAR results with either Potoff's or my own HR results (might be better to use my own for self consistency)
%	%	\end{enumerate}
%	%    \item Validation of the basis function approach
%	\item Epsilon scaling for all the compounds that Mohammad has U and N values for (branched alkanes and alkynes) and which have good experimental data
%	\item We estimate MiPPE generalized and NERD VLE from TraPPE simulations, MiPPE S/L from MiPPE generalized, and TraPPE from MiPPE generalized
%	\item For $\lambda_{\rm ref} = 12$ and $\lambda_{\rm rr} = 16$, GCMC-MBAR predicts vapor density, vapor pressure, and heat of vaporization more accurately than liquid density
%	\item For $\lambda_{\rm ref} = 12$ and $\lambda_{\rm rr} = 12$, i.e., computing NERD from TraPPE simulations, GCMC-MBAR predicts all four properties accurately    
%	\item We present how basis functions allow for rapid computation of wide range of parameter sets:
%	\begin{enumerate}
%%		\item \textit{n}-hexane
%		\item 2-methylpropane
%		\item 2,2-dimethylpropane
%		\item cyclopentane or cyclohexane
%	\end{enumerate}
%	\item We provide supporting information with basis functions for several branched alkanes with TraPPE and MiPPE force fields
%\end{enumerate}
%
%\subsection{Figures}
%
%
%\begin{enumerate}
%	\item Percent deviation between MBAR and HR results for rholiq, rhovap, Psat, and DeltaHv
%	\item Comparison between MBAR bootstrapping and analytical uncertainties and HR uncertainties (?)
%	\item Scaling of epsilon post-simulation for branched alkanes and alkynes
%	\item Prediction of VLE for $\lambda_{\rm ref} \neq \lambda_{\rm rr}$
%	\item Prediction of VLE for $\lambda_{\rm ref} = \lambda_{\rm rr}$
%%	\item Two-D scans of scoring functions for $\epsilon-\sigma$ of CH3 (a) and CH2 (b) for \textit{n}-hexane
%	\item Two-D scans of scoring functions for $\epsilon-\sigma$ of CH3 (a) and CH (b) for 2-methylpropane
%	\item Two-D scans of scoring functions for $\epsilon-\sigma$ of CH3 (a) and C (b) for 2,2-dimethylpropane
%	\item Two-D scans of scoring functions for $\epsilon-\sigma$ of CH2 for cyclopentane or cyclohexane (reference is TraPPE)
%\end{enumerate}

\section{Discussion} \label{sec: Discussion}

The results presented in this study were obtained by performing simulations with only a single reference force field. As shown in previous studies, a logical approach for improving the performance of MBAR is to include additional reference force fields \cite{Postdoc_1,Postdoc_2}. For example, in Section \ref{sec: Case study}, we did not utilize the snapshots from $\lambda_{\rm CH_2} = 14$ when computing properties for $\lambda_{\rm CH_2} = 16$ and vice versa. Using multiple reference force fields would certainly increase the number of effective snapshots. However, the top-left panel of Figure \ref{fig:Neff_CYC6} demonstrates that a single reference provides sufficient overlap over a wide region of $\epsilon$ and $\sigma$ parameter sets when $\lambda_{\rm rr} = \lambda_{\rm ref}$. For this reason, we deemed it unnecessary to combine the snapshots from $\theta^{\langle0\rangle}$ and the four $\theta^{\langle1\rangle}$ reference simulations.

% For this reason, it was unnecessary to combine the snapshots 

%MRS3: isn't increasing the number of effective snapshots a good thing?

% For example, it could be helpful to simulate multiple $\sigma_{\rm ref}$ and $\lambda_{\rm ref}$ values 

%For example, in Section \ref{sec: Case study} we could sample from multiple $\sigma_{\rm ref}$ and $\lambda_{\rm ref}$ to reduce 

%For example, we could include a reference parameter set for each value of $\lambda_{\rm CH_2}$ to increase $K_{\rm snaps}^{\rm eff, liq}$ and, thereby, improve the reliability of liquid phase property estimates.

As molecular insertion moves are frequently rejected in high density systems, GCMC simulations are typically not reliable at low saturation temperatures $(T^{\rm sat} < 0.65 T_{\rm c})$. Because ITIC does not suffer from this low-temperature limitation, we recommend combining the MBAR-ITIC and GCMC-MBAR methods when predicting vapor-liquid coexistence properties from near-triple-point to near-critical-point conditions.

Although GCMC-HR is a standard approach for computing vapor-liquid coexistence, HR has also been applied to GEMC simulations (GEMC-HR) \cite{Boulougouris2010}. Therefore, while the present study presents how MBAR can be applied to GCMC simulations, an analogous GEMC-MBAR approach is worth investigating in future work.

%The main objective of GCMC-MBAR is to efficiently estimate phase equilibria properties for many force field parameter sets $(\theta)$ by limiting the number of molecular simulations. This is accomplished by maximizing the information content extracted from each simulation by storing configurations (or basis functions) instead of histograms. As mentioned in Section \ref{sec: Introduction}, Hamiltonian scaling (HS) is an alternative method to achieve the same objective. HS-GCMC samples multiple $\theta_{\rm ref}$ in a single simulation and reweights histograms from each $\theta_{\rm ref}$ to i for  parameter set are reweighted . We believe a rigorous comparison of the respective efficiencies of these two methods merits future consideration.

% of each histograms from each $\theta_{\rm ref}$ are combined after reweighting (scaling) and combines the histograms from each $\theta_{\rm ref}$ after scaling (reweighting) the contribution of  reweighting (scaling).

% applying an appropriate scaling to ) and reweights the histogram , scales the contribution of all snapshots from the combined histogram of each $\theta_{\rm ref}$.

%In GCMC-MBAR, this is accomplished by storing configurations (or basis functions) instead of histograms. By contrast, HS-GCMC samples multiple $\theta_{\rm ref}$ in a single simulation and combines the histograms obtained from each $\theta_{\rm ref}$ (after scaling the importance of each snapshot for the $\theta_{\rm ref}$ of interest).

%The main objective of GCMC-MBAR is to efficiently estimate phase equilibria properties for many force field parameter sets $(\theta)$ by limiting the number of molecular simulations. As mentioned in Section \ref{sec: Introduction}, Hamiltonian scaling (HS) is an alternative method to achieve the same objective. In both approaches, the number of simulations are reduced by maximizing the information content extracted from each simulation. In HS-GCMC, this is accomplished by sampling multiple $\theta_{\rm ref}$ in a single simulation and combining the histograms obtained from each $\theta_{\rm ref}$ (after scaling the importance of each snapshot for the $\theta_{\rm ref}$ of interest). By contrast, information content is maximized in GCMC-MBAR by simply storing configurations (or basis functions) instead of histograms. We believe a rigorous comparison of the respective efficiencies of these two methods merits future consideration. Furthermore, we suggest a possible synergy between the two methods that could combine their efficiencies. Specifically, a hybrid HS-GCMC-MBAR approach would store configurations (or basis functions) that are sampled from multiple $\theta_{\rm ref}$ in a single simulation and, subsequently, apply MBAR instead of histogram reweighting as the post-simulation analysis.

%The main objective of GCMC-MBAR is similar to that of Hamiltonian scaling (HS), namely, to efficiently estimate phase equilibria properties for many force field parameter sets $(\theta)$ by maximizing the information content extracted from each simulation. Despite some apparent similarities between HS-GCMC and GCMC-MBAR, the means by which these two methods accomplish this objective are quite distinct. Both methods reweight from the mixture distribution of multiple $\theta_{\rm ref}$ \cite{arxivShirts2017}. The difference is that GCMC-MBAR generates this mixture distribution by combining configurations (or basis functions) sampled from independent simulations, while HS-GCMC samples directly from this mixture distribution in a single simulation and stores a combined (and scaled) histogram for each $\theta_{\rm ref}$. To put it simply, HS-GCMC improves efficiency by modifying the sampling (simulation) approach, whereas GCMC-MBAR simply modifies the post-simulation data analysis (reweighting). Therefore, a possible synergy exists between the two methods by employing HS for sampling and MBAR for data analysis.

The main objective of GCMC-MBAR is similar to that of Hamiltonian scaling (HS), namely, to efficiently estimate phase equilibria properties for many force field parameter sets $(\theta)$ by maximizing the information content extracted from each simulation. Despite some apparent similarities between HS-GCMC and GCMC-MBAR, the means by which these two methods accomplish this objective are quite distinct. Both methods take advantage of the concept of a \textit{mixture distribution} that corresponds to multiple $\theta$.\cite{arxivShirts2017} The difference is that GCMC-MBAR generates this mixture distribution by combining configurations (or basis functions) sampled from independent simulations, while HS-GCMC samples directly from this mixture distribution in a single simulation and stores a combined (and scaled) histogram for each $\theta_{\rm ref}$. Thus, HS-GCMC improves efficiency by modifying the sampling, whereas GCMC-MBAR simply modifies the post-simulation data analysis. Because HS-GCMC uses histograms in the analysis, in contrast with GCMC-MBAR, only the parameters that are simulated in the original mixture distribution can be combined. A possible synergy, beyond the scope of this study, could exist between the two methods by employing HS for sampling and MBAR for data analysis.

The present study has focused on the van der Waals Mie $\lambda$-6 non-bonded parameters $(\epsilon$, $\sigma$, and $\lambda$). However, GCMC-MBAR can also be applied to parameterize electrostatic non-bonded parameters, e.g., point charges $(q)$ \cite{naden:jctc:2016}. While the number of effective snapshots should still provide a reasonable estimate of reliability for $q_{\rm rr} \neq q_{\rm ref}$, we recommend that future work test the range of $q_{\rm rr}$ over which GCMC-MBAR is reliable.

%   $q_{\rm rr} \neq $  We recommend performing similar testing 

%, compared to the traditional histogram reweighting approach (GCMC-HR) 

%Specifically, this hybrid HS-GCMC-MBAR approach would store configurations (or basis functions) that are sampled from multiple $\theta_{\rm ref}$ in a single simulation and, subsequently, predict phase equilibria for arbitrary parameter sets $(\theta_{\rm rr} \neq \theta_{\rm ref})$ by applying MBAR as the post-simulation analysis.

%The difference is that GCMC-MBAR generates this mixture distribution from independent simulations while HS-GCMC samples directly from this mixture distribution in a single simulation. To put it simply, HS-GCMC improves efficiency by modifying the sampling (simulation) approach and storing a separate (scaled) histogram for each $\theta_{\rm ref}$, whereas GCMC-MBAR simply modifies the post-simulation data analysis compared to the traditional histogram reweighting approach (GCMC-HR) by storing configurations (or basis functions). 

%data output (storing configurations or basis functions) and post-simulation data analysis compared to the traditional histogram reweighting approach (GCMC-HR).

%Specifically, HS samples directly from the mixture distribution of multiple $\theta_{\rm ref}$ in a single simulation. By contrast, GCMC-MBAR reweights from the mixture distribution of $\theta_{\rm ref}$ generated in independent simulations.  whereas  The efficiency of HS-GCMC is related to the modified sampling approach, whereas the efficiency of GCMC-MBAR is associated with the post-simulation analysis. Therefore, a possible synergy exists between the two methods. Specifically, a hybrid HS-GCMC-MBAR approach is conceivable that would store configurations (or basis functions) that are sampled from multiple $\theta_{\rm ref}$ in a single simulation and, subsequently, apply MBAR instead of histogram reweighting as the post-simulation analysis.  HS-GCMC samples from the mixture distribution of multiple $\theta_{\rm ref}$ in a single simulation, whereas MBAR simply stores  As mentioned in Section \ref{sec: Introduction}, Hamiltonian scaling (HS) is an alternative method to achieve the same objective. In both approaches, the number of simulations are reduced . In HS-GCMC, this is accomplished by sampling multiple $\theta_{\rm ref}$ in a single simulation and combining the histograms obtained from each $\theta_{\rm ref}$ (after scaling the importance of each snapshot for the $\theta_{\rm ref}$ of interest). By contrast, information content is maximized in GCMC-MBAR by simply storing configurations (or basis functions) instead of histograms. We believe a rigorous comparison of the respective efficiencies of these two methods merits future consideration. Furthermore, we suggest a possible synergy between the two methods that could combine their efficiencies. Specifically, a hybrid HS-GCMC-MBAR approach would store configurations (or basis functions) that are sampled from multiple $\theta_{\rm ref}$ in a single simulation and, subsequently, apply MBAR instead of histogram reweighting as the post-simulation analysis.

%, a combined HS-GCMC-MBAR approach could conceivably combine the efficiency of sampling multiple $\theta_{\rm ref}$  we sug HS-GCMC-MBAR This hybrid HS-GCMC-MBAR approach could conceivably combine their efficiencies.

%Although similarities exist between HS-GCMC and GCMC-MBAR, GCMC-MBAR possesses some significant methodological and practical advantages. The primary advantage is that GCMC-MBAR is capable of predicting phase equilibria for any force field parameter set ($\theta_{\rm rr} = \theta_{\rm ref}$ and $\theta_{\rm rr} \neq \theta_{\rm ref}$), whereas HS-GCMC can predict phase equilibria only for the parameter sets that are simulated directly ($\theta_{\rm rr} = \theta_{\rm ref}$). in contrast to HS-GCMC, GCMC-MBAR does not require modifying the simulation protocol  the two methods,   Howecan be utilized for the same purpose. has is designed for the same purpose  HS-GCMC and G

%As mentioned in Section \ref{sec: Introduction}, although similarities exist between HS-GCMC and GCMC-MBAR, GCMC-MBAR possesses some significant methodological and practical advantages. The primary advantage is that GCMC-MBAR is capable of predicting phase equilibria for any force field parameter set ($\theta_{\rm rr} = \theta_{\rm ref}$ and $\theta_{\rm rr} \neq \theta_{\rm ref}$), whereas HS-GCMC can predict phase equilibria only for the parameter sets that are simulated directly ($\theta_{\rm rr} = \theta_{\rm ref}$). in contrast to HS-GCMC, GCMC-MBAR does not require modifying the simulation protocol  the two methods,   Howecan be utilized for the same purpose. has is designed for the same purpose  HS-GCMC and G
%
%The main objective of GCMC-MBAR is to efficiently estimate phase equilibria properties for many force field parameter sets $(\theta)$ by limiting the number of molecular simulations. This is accomplished by maximizing the information content extracted from each simulation by storing configurations (or basis functions) instead of histograms. Hamiltonian scaling (HS) is an alternative method to achieve the same objective, where in HS-GCMC multiple parameter sets are sampled from a single simulation. We believe a rigorous comparison of the respective efficiencies of these two methods merits future consideration.


%
%\begin{enumerate}
%	\item We recommend that future GCMC-VLE studies report the snapshots of $N$ and $U$ and/or basis functions to recompute $U$ as this allows for future force field optimization
%	\item Improvements are possible with multiple $\theta$ or simulating a range of $\mu$ values
%\end{enumerate}

\section{Conclusions} \label{sec: Conclusions}

This study demonstrates how the Multistate Bennett Acceptance Ratio (MBAR) can replace the traditional histogram reweighting approach for estimating vapor-liquid coexistence properties from grand canonical Monte Carlo simulations. MBAR and HR are mathematically equivalent in the limit of infinitesimal bin widths when the coexistence properties are computed for the reference force field. However, the primary benefit of MBAR is the ability to estimate properties for force fields that are not simulated directly, which greatly accelerates non-bonded parameterization.

% by reducing the number of simulations. 

We perform a one-dimensional $\epsilon$-scaling post-simulation optimization of several branched alkanes and alkynes. We provide empirical evidence that GCMC-MBAR is accurate when the number of effective snapshots is greater than fifty, which is typically the case in the vapor phase and in the liquid phase when $\lambda_{\rm rr} = \lambda_{\rm ref}$. We then show how GCMC-MBAR can rapidly parameterize a family of Mie $\lambda$-6 potentials starting with a pre-tuned Lennard-Jones 12-6 potential (TraPPE). Specifically, with only two stages of direct GCMC simulation we consider hundreds of $\epsilon_{\rm CH_2}$, $\sigma_{\rm CH_2}$, and $\lambda_{\rm CH_2}$ parameter sets. The optimized Mie 16-6 parameters for cyclohexane form the most recent contribution to the Mie Potentials for Phase Equilibria (MiPPE) force field.

\section*{Acknowledgments}

We would like to express our gratitude to Dr. J. Richard Elliott for several invaluable discussions regarding force field optimization. We are also appreciative of the internal review provided by Yauheni (Eugene) Paulechka, Andrei F. Kazakov, Daniel G. Friend, and Marcia L. Huber of the National Institute of Standards and Technology (NIST).

This research was performed while Richard A. Messerly held a National Research Council (NRC) Postdoctoral Research Associateship at NIST. Jeffrey J. Potoff and Mohammad Soroush Barhaghi acknowledge funding from NSF OAC-1642406. Some of the computations in this work were performed with resources from the Grid Computing initiative at Wayne State University. 

Commercial equipment, instruments, or materials are identified only in order to adequately specify certain procedures. In no case does such identification imply recommendation or endorsement by NIST, nor does it imply that the products identified are necessarily the best available for the intended purpose.

Contribution of NIST, an agency of the United States government; not subject to copyright in the United States.

\section*{Supporting information}

Section \ref{SI sec: Bonded parameters} reports the bonded parameters. Section \ref{SI sec: Fixed vs flexible bonds} compares phase equilibria for fixed and flexible bonds. Section \ref{SI sec: CBMC acceptance rates} reports CBMC acceptance rates. Section \ref{SI sec: Machine hardware} discusses hardware details. Section \ref{SI sec: eps scale} provides the tabulated $\epsilon$-scaling results. Section \ref{SI sec: Case study} contains additional tabulated values and figures for the cyclohexane optimization. Section \ref{SI sec: Z} depicts the compressibility factor of the vapor phase. Section \ref{SI sec: Tabulated MBAR results} contains the tabulated values for the MBAR validation. Section \ref{SI sec: State Points} reports the simulation state points.

\bibliography{JCED_FOMMS_references}

%\section{Supporting Information}
%
%\subsection{MBAR VLE estimates}
%
%Provide tables of MBAR estimates
%
%\subsection{Basis functions}
%
%\begin{enumerate}
%	\item Validation that basis functions give accurate energies
%\end{enumerate}
%
%\subsection{Raw data}
%
%\begin{enumerate}
%	\item Comparison of 2-D histograms for TraPPE and MiPPE. MBAR overlap, possible? Probably not without rerunning the simulations.
%\end{enumerate}

\clearpage
\newpage

\section*{TOC graphic}

	\begin{figure}[H]
		\centering
		\includegraphics[width=8cm]{TOC_graphic.pdf}
		\caption{for Table of Contents use only}
	\end{figure}



\end{document}
