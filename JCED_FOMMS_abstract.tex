\documentclass[12pt]{article}
%\documentclass[letterpaper,floatfix,citeautoscript,aip,jcp]{revtex4-1}
%\documentclass[letterpaper,floatfix,citeautoscript,showkeys]{revtex4-1}
%\documentclass[twocolumn,letterpaper,floatfix,citeautoscript,jcp]{revtex4-1}
%\documentclass[twocolumn,letterpaper,floatfix,citeautoscript,aip,jcp]{revtex4-1}
%\documentclass[journal=jced,manuscript=article]{achemso}
%\setkeys{acs}{maxauthors=30,etalmode=truncate,articletitle=true}
%%%%%%%%%%%%%%%%%%%%%%%%%%%%%%%%%%%%%%%%%%%%%%%%%%%%%%%%%%%%%%%%%%%%%
%% Place any additional packages needed here.  Only include packages
%% which are essential, to avoid problems later.
%%%%%%%%%%%%%%%%%%%%%%%%%%%%%%%%%%%%%%%%%%%%%%%%%%%%%%%%%%%%%%%%%%%%%
\usepackage{chemformula} % Formula subscripts using \ch{}
\usepackage[T1]{fontenc} % Use modern font encodings

%%%%%%%%%%%%%%%%%%%%%%%%%%%%%%%%%%%%%%%%%%%%%%%%%%%%%%%%%%%%%%%%%%%%%
%% If issues arise when submitting your manuscript, you may want to
%% un-comment the next line.  This provides information on the
%% version of every file you have used.
%%%%%%%%%%%%%%%%%%%%%%%%%%%%%%%%%%%%%%%%%%%%%%%%%%%%%%%%%%%%%%%%%%%%%
%%\listfiles

%%%%%%%%%%%%%%%%%%%%%%%%%%%%%%%%%%%%%%%%%%%%%%%%%%%%%%%%%%%%%%%%%%%%%
%% Place any additional macros here.  Please use \newcommand* where
%% possible, and avoid layout-changing macros (which are not used
%% when typesetting).
%%%%%%%%%%%%%%%%%%%%%%%%%%%%%%%%%%%%%%%%%%%%%%%%%%%%%%%%%%%%%%%%%%%%%
% \newcommand*\mycommand[1]{\texttt{\emph{#1}}}

\usepackage{fullpage}
\usepackage{amsfonts}
\usepackage{graphicx}
\usepackage{float}
\usepackage{amsmath}
\usepackage{chemfig}
\usepackage{indentfirst}
\usepackage{longtable}
\usepackage{array}
\usepackage{cellspace}
\usepackage{palatino}
%\usepackage{breqn}
\usepackage{amssymb}
\usepackage{verbatim}
\usepackage[hidelinks,colorlinks=false,citecolor=black,linkcolor=black]{hyperref}
\usepackage{siunitx}
\usepackage{xr}

%\title{}

\begin{document}

\section*{Abstract}

%Reliable prediction of vapor-liquid phase equilibria with molecular simulation relies on efficient computational methods and well-parameterized force fields. Histogram reweighting (HR) is a standard approach for converting grand canonical Monte Carlo (GCMC) simulation output into vapor-liquid coexistence properties (saturated liquid density, $\rho_{\rm liq}^{\rm sat}$, saturated vapor density, $\rho_{\rm vap}^{\rm sat}$, saturated vapor pressures, $P_{\rm vap}^{\rm sat}$, and enthalpy of vaporization, $\Delta H_{\rm v}$). Due to the abundance of experimental vapor-liquid coexistence data and the sensitivity of such properties to both short- and long-range non-bonded interactions, numerous force fields have been parameterized using $\rho_{\rm liq}^{\rm sat}$, $\rho_{\rm vap}^{\rm sat}$, $P_{\rm vap}^{\rm sat}$, and/or $\Delta H_{\rm v}$. Unfortunately, computing $\rho_{\rm liq}^{\rm sat}$, $\rho_{\rm vap}^{\rm sat}$, $P_{\rm vap}^{\rm sat}$, and $\Delta H_{\rm v}$ for each proposed non-bonded parameter set (e.g., the Lennard-Jones parameters $\epsilon$ and $\sigma$) requires a large amount of molecular simulations. 
%
%We demonstrate that histogram-free reweighting can alleviate the computational burden of force field parameterization. Specifically, the Multistate Bennett Acceptance Ratio (MBAR) reweights configurations sampled from GCMC simulations performed with an initial reference parameter set $(\theta_{\rm ref})$. While MBAR is similar to the traditional HR method for computing $\rho_{\rm liq}^{\rm sat}$, $\rho_{\rm vap}^{\rm sat}$, $P_{\rm vap}^{\rm sat}$, and $\Delta H_{\rm v}$, the primary advantage of MBAR is that this approach can estimate coexistence properties for a different ``rerun'' parameter set $(\theta_{\rm rr} \neq \theta_{\rm ref})$ without simulating $\theta_{\rm rr}$ directly. MBAR thus greatly reduces the amount of GCMC simulations that are required for parameterizing the non-bonded potential.
%
%Four different applications of GCMC-MBAR are presented in this study. First, we validate that GCMC-MBAR and GCMC-HR yield statistically indistinguishable results for $\rho_{\rm liq}^{\rm sat}$, $\rho_{\rm vap}^{\rm sat}$, $P_{\rm vap}^{\rm sat}$, and $\Delta H_{\rm v}$ when $\theta_{\rm rr} = \theta_{\rm ref}$. Second, we utilize GCMC-MBAR to optimize an individualized (compound-specific) parameter ($\psi$) for 8 branched alkanes and 11 alkynes using the Mie Potentials for Phase Equilibria (MiPPE) force field. Third, we predict $\rho_{\rm liq}^{\rm sat}$, $\rho_{\rm vap}^{\rm sat}$, $P_{\rm vap}^{\rm sat}$, and $\Delta H_{\rm v}$ for force field $j$ by simulating force field $i$, where $i$ and $j$ are common force fields from the literature. In addition, we provide guidelines for determining the reliability of GCMC-MBAR predicted values when $\theta_{\rm rr} \not\approx \theta_{\rm ref}$. Fourth, we develop an optimization scheme and report new MiPPE non-bonded parameters for cyclohexane ($\epsilon_{\rm CH_2}$, $\sigma_{\rm CH_2}$, and $\lambda_{\rm CH_2}$).   

Histogram reweighting (HR) is a standard approach for converting grand canonical Monte Carlo (GCMC) simulation output into vapor-liquid coexistence properties (saturated liquid density, $\rho_{\rm liq}^{\rm sat}$, saturated vapor density, $\rho_{\rm vap}^{\rm sat}$, saturated vapor pressures, $P_{\rm vap}^{\rm sat}$, and enthalpy of vaporization, $\Delta H_{\rm v}$). We demonstrate that a histogram-free reweighting approach, namely, the Multistate Bennett Acceptance Ratio (MBAR), is similar to the traditional HR method for computing $\rho_{\rm liq}^{\rm sat}$, $\rho_{\rm vap}^{\rm sat}$, $P_{\rm vap}^{\rm sat}$, and $\Delta H_{\rm v}$. The primary advantage of MBAR is the ability to predict phase equilibria properties for an arbitrary force field parameter set that has not been simulated directly. Thus, MBAR can greatly reduce the number of GCMC simulations that are required to parameterize a force field with phase equilibria data.

Four different applications of GCMC-MBAR are presented in this study. First, we validate that GCMC-MBAR and GCMC-HR yield statistically indistinguishable results for $\rho_{\rm liq}^{\rm sat}$, $\rho_{\rm vap}^{\rm sat}$, $P_{\rm vap}^{\rm sat}$, and $\Delta H_{\rm v}$ in a limiting test case. Second, we utilize GCMC-MBAR to optimize an individualized (compound-specific) parameter ($\psi$) for 8 branched alkanes and 11 alkynes using the Mie Potentials for Phase Equilibria (MiPPE) force field. Third, we predict $\rho_{\rm liq}^{\rm sat}$, $\rho_{\rm vap}^{\rm sat}$, $P_{\rm vap}^{\rm sat}$, and $\Delta H_{\rm v}$ for force field $j$ by simulating force field $i$, where $i$ and $j$ are common force fields from the literature. In addition, we provide guidelines for determining the reliability of GCMC-MBAR predicted values. Fourth, we develop and apply a post-simulation optimization scheme to obtain new MiPPE non-bonded parameters for cyclohexane ($\epsilon_{\rm CH_2}$, $\sigma_{\rm CH_2}$, and $\lambda_{\rm CH_2}$).

\end{document}