\documentclass[journal=jctc,manuscript=article]{achemso}
\setkeys{acs}{articletitle = true}
 
%%%%%%%%%%%%%%%%%%%%%%%%%%%%%%%%%%%%%%%%%%%%%%%%%%%%%%%%%%%%%%%%%%%%%
%% Place any additional packages needed here.  Only include packages
%% which are essential, to avoid problems later.
%%%%%%%%%%%%%%%%%%%%%%%%%%%%%%%%%%%%%%%%%%%%%%%%%%%%%%%%%%%%%%%%%%%%%
\usepackage{chemformula} % Formula subscripts using \ch{}
\usepackage[T1]{fontenc} % Use modern font encodings

%%%%%%%%%%%%%%%%%%%%%%%%%%%%%%%%%%%%%%%%%%%%%%%%%%%%%%%%%%%%%%%%%%%%%
%% If issues arise when submitting your manuscript, you may want to
%% un-comment the next line.  This provides information on the
%% version of every file you have used.
%%%%%%%%%%%%%%%%%%%%%%%%%%%%%%%%%%%%%%%%%%%%%%%%%%%%%%%%%%%%%%%%%%%%%
%%\listfiles

%%%%%%%%%%%%%%%%%%%%%%%%%%%%%%%%%%%%%%%%%%%%%%%%%%%%%%%%%%%%%%%%%%%%%
%% Place any additional macros here.  Please use \newcommand* where
%% possible, and avoid layout-changing macros (which are not used
%% when typesetting).
%%%%%%%%%%%%%%%%%%%%%%%%%%%%%%%%%%%%%%%%%%%%%%%%%%%%%%%%%%%%%%%%%%%%%
% \newcommand*\mycommand[1]{\texttt{\emph{#1}}}

\usepackage{fullpage}
\usepackage{amsfonts}
\usepackage{graphicx}
\usepackage{float}
\usepackage{amsmath}
\usepackage{chemfig}
\usepackage{indentfirst}
\usepackage{longtable}
\usepackage{array}
\usepackage{cellspace}
\usepackage{palatino}
%\usepackage{breqn}
\usepackage{amssymb}
\usepackage{verbatim}
\usepackage[hidelinks,colorlinks=false,citecolor=black,linkcolor=black]{hyperref}
\usepackage{siunitx}
\usepackage{xr}
%\usepackage{bibentry}
\usepackage{verbatim}

%\DefineVerbatimEnvironment%
%	{verbatimprog}%
%	{Verbatim}%
%	{fontsize=\footnotesize}%

\newenvironment{myequation}{%
\addtocounter{equation}{-1}
\refstepcounter{defcounter}
\renewcommand\theequation{SI.\thedefcounter}
\begin{equation*}}
{\end{equation*}}

\renewcommand{\thefigure}{SI.\arabic{figure}}

\renewcommand{\thepage}{SI.\arabic{page}}

\renewcommand{\thesection}{SI.\Roman{section}}

\renewcommand{\thetable}{SI.\Roman{table}}

\makeatletter
\newcommand*{\addFileDependency}[1]{% argument=file name and extension
	\typeout{(#1)}
	\@addtofilelist{#1}
	\IfFileExists{#1}{}{\typeout{No file #1.}}
}
\makeatother

\newcommand*{\myexternaldocument}[1]{%
	\externaldocument{#1}%
	\addFileDependency{#1.tex}%
	\addFileDependency{#1.aux}%
}

\myexternaldocument{JCED_FOMMS_manuscript}

\SectionNumbersOn

% The figures are in a figures/ subdirectory.
\graphicspath{{figures/}}

%\bibliographystyle{apsrevlong}
%\bibliographystyle{apsrev}
\bibliographystyle{unsrt}
%\bibliographystyle{acs}

% italicized boldface for math (e.g. vectors)
\newcommand{\bfv}[1]{{\mbox{\boldmath{$#1$}}}}
% non-italicized boldface for math (e.g. matrices)
\newcommand{\bfm}[1]{{\bf #1}}          

%\newcommand{\bfm}[1]{{\mbox{\boldmath{$#1$}}}}
%\newcommand{\bfm}[1]{{\bf #1}}
\newcommand{\expect}[1]{\left \langle #1 \right \rangle} % <.> for denoting expectations over realizations of an experiment or thermal averages

\newcommand{\var}[1]{{\mathrm var}{(#1)}}
\newcommand{\x}{\bfv{x}}
\newcommand{\y}{\bfv{y}}
\newcommand{\f}{\bfv{f}}

\newcommand{\hatf}{\hat{f}}

\newcommand{\bTheta}{\bfm{\Theta}}
\newcommand{\btheta}{\bfm{\theta}}
\newcommand{\bhatf}{\bfm{\hat{f}}}
\newcommand{\Cov}[1] {\mathrm{cov}\left( #1 \right)}
\newcommand{\T}{\mathrm{T}}                                % T used in matrix transpose

\author{Richard A. Messerly}
\email{richard.messerly@nist.gov}
\affiliation{Thermodynamics Research Center, National Institute of Standards and Technology, Boulder, Colorado, 80305, United States}

\author{Mohammad S. Barhaghi}
\affiliation{Department of Chemical Engineering and Materials Science, Wayne State University, Detroit, Michigan 48202, United States}

\author{Jeffrey J. Potoff}
\affiliation{Department of Chemical Engineering and Materials Science, Wayne State University, Detroit, Michigan 48202, United States}

\author{Michael R. Shirts}
\affiliation{Department of Chemical and Biological Engineering, University of Colorado, Boulder, Colorado, 80309, United States}

%%%%%%%%%%%%%%%%%%%%%%%%%%%%%%%%%%%%%%%%%%%%%%%%%%%%%%%%%%%%%%%%%%%%%
%% The document title should be given as usual. Some journals require
%% a running title from the author: this should be supplied as an
%% optional argument to \title.
%%%%%%%%%%%%%%%%%%%%%%%%%%%%%%%%%%%%%%%%%%%%%%%%%%%%%%%%%%%%%%%%%%%%%
%\title{Multistate Bennett Acceptance Ratio replaces histogram reweighting for vapor-liquid coexistence calculations}
%\title{Multistate Bennett Acceptance Ratio to enable rapid force field parameterization}
%\title{Multistate Bennett Acceptance Ratio as a substitute for histogram reweighting when optimizing non-bonded parameters}
%\title{Multistate reweighting provides a better alternative to histogram reweighting for coexistance calculations}
%\title{Multistate histogram-free reweighting for vapor-liquid coexistence calculations of non-simulated force field parameters}
%\title{Estimating vapor-liquid coexistence properties with histogram-free reweighting}
%\title{Histogram-free reweighting for vapor-liquid coexistence calculations of multiple force fields}
\title{Supporting information: Histogram-free reweighting to estimate vapor-liquid coexistence properties of non-simulated force fields}

%%%%%%%%%%%%%%%%%%%%%%%%%%%%%%%%%%%%%%%%%%%%%%%%%%%%%%%%%%%%%%%%%%%%%
%% Some journals require a list of abbreviations or keywords to be
%% supplied. These should be set up here, and will be printed after
%% the title and author information, if needed.
%%%%%%%%%%%%%%%%%%%%%%%%%%%%%%%%%%%%%%%%%%%%%%%%%%%%%%%%%%%%%%%%%%%%%
%\abbreviations{IR,NMR,UV}
\keywords{MBAR, Monte Carlo, Grand Canonical, Vapor-liquid equilibria}

%%%%%%%%%%%%%%%%%%%%%%%%%%%%%%%%%%%%%%%%%%%%%%%%%%%%%%%%%%%%%%%%%%%%%
%% The manuscript does not need to include \maketitle, which is
%% executed automatically.
%%%%%%%%%%%%%%%%%%%%%%%%%%%%%%%%%%%%%%%%%%%%%%%%%%%%%%%%%%%%%%%%%%%%%
\begin{document}

\section{State Points} \label{State Points}

\begin{table}[htb!]
	\caption{State points simulated for 2-methylpropane with the TraPPE force field.}
	\begin{center}
		\begin{tabular}{|c|c|c|}
			\hline
			$T$ (K) & $\mu$ (K) & $L$ (nm) \\ \hline
			350	&	-3120	&	3.0	\\
			380	&	-3120	&	3.0	\\
			405	&	-3117	&	3.0	\\
			380	&	-2980	&	3.0	\\
			350	&	-2880	&	3.0	\\
			320	&	-2790	&	3.0	\\
			290	&	-2705	&	3.0	\\
			260	&	-2645	&	3.0	\\
			230	&	-2600	&	3.0	\\
			200	&	-2570	&	3.0	\\
			\hline
		\end{tabular}
	\end{center}
\end{table}

\begin{table}[htb!]
	\caption{State points simulated for 2,2-dimethylpropane with the TraPPE force field.}
	\begin{center}
		\begin{tabular}{|c|c|c|}
			\hline
			$T$ (K) & $\mu$ (K) & $L$ (nm) \\ \hline
			380	&	-3405	&	3.0	\\
			410	&	-3405	&	3.0	\\
			440	&	-3405	&	3.0	\\
			410	&	-3250	&	3.0	\\
			380	&	-3140	&	3.0	\\
			350	&	-3037	&	3.0	\\
			330	&	-2970	&	3.0	\\
			300	&	-2900	&	3.0	\\
			270	&	-2820	&	3.0	\\
			\hline
		\end{tabular}
	\end{center}
\end{table}

\begin{table}[htb!]
	\caption{State points simulated for 2,2-dimethylbutane with the TraPPE force field.}
	\begin{center}
		\begin{tabular}{|c|c|c|}
			\hline
			$T$ (K) & $\mu$ (K) & $L$ (nm) \\ \hline
			420	&	-3860	&	3.5	\\
			450	&	-3860	&	3.5	\\
			480	&	-3860	&	3.5	\\
			450	&	-3719	&	3.5	\\
			420	&	-3600	&	3.5	\\
			400	&	-3524	&	3.5	\\
			380	&	-3450	&	3.5	\\
			360	&	-3368	&	3.5	\\
			340	&	-3288	&	3.5	\\
			310	&	-3280	&	3.5	\\
			\hline
		\end{tabular}
	\end{center}
\end{table}

\begin{table}[htb!]
	\caption{State points simulated for 2,3-dimethylbutane with the TraPPE force field.}
	\begin{center}
		\begin{tabular}{|c|c|c|}
			\hline
			$T$ (K) & $\mu$ (K) & $L$ (nm) \\ \hline
			440	&	-4015	&	3.0	\\
			470	&	-4015	&	3.0	\\
			500	&	-4011	&	3.0	\\
			470	&	-3845	&	3.0	\\
			440	&	-3735	&	3.0	\\
			410	&	-3635	&	3.0	\\
			380	&	-3555	&	3.0	\\
			350	&	-3480	&	3.0	\\
			320	&	-3415	&	3.0	\\
			\hline
		\end{tabular}
	\end{center}
\end{table}

\begin{table}[htb!]
	\caption{State points simulated for 3,3-dimethylhexane with the TraPPE force field.}
	\begin{center}
		\begin{tabular}{|c|c|c|}
			\hline
			$T$ (K) & $\mu$ (K) & $L$ (nm) \\ \hline
			500	&	-4670	&	3.5	\\
			530	&	-4670	&	3.5	\\
			560	&	-4670	&	3.5	\\
			520	&	-4476	&	3.5	\\
			490	&	-4370	&	3.5	\\
			460	&	-4268	&	3.5	\\
			430	&	-4164	&	3.5	\\
			400	&	-4039	&	3.5	\\
			370	&	-3925	&	3.5	\\
			\hline
		\end{tabular}
	\end{center}
\end{table}

\begin{table}[htb!]
	\caption{State points simulated for 3-methyl-3-ethylpentane with the TraPPE force field.}
	\begin{center}
		\begin{tabular}{|c|c|c|}
			\hline
			$T$ (K) & $\mu$ (K) & $L$ (nm) \\ \hline
			500	&	-4785	&	4.0	\\
			550	&	-4785	&	4.0	\\
			580	&	-4785	&	4.0	\\
			550	&	-4636	&	4.0	\\
			520	&	-4520	&	4.0	\\
			490	&	-4400	&	4.0	\\
			460	&	-4280	&	4.0	\\
			430	&	-4160	&	4.0	\\
			410	&	-4080	&	4.0	\\
			390	&	-3990	&	4.0	\\
			\hline
		\end{tabular}
	\end{center}
\end{table}

\begin{table}[htb!]
	\caption{State points simulated for 2,3,4-trimethylpentane with the TraPPE force field.}
	\begin{center}
		\begin{tabular}{|c|c|c|}
			\hline
			$T$ (K) & $\mu$ (K) & $L$ (nm) \\ \hline
			480	&	-4740	&	3.5	\\
			520	&	-4740	&	3.5	\\
			565	&	-4735	&	3.5	\\
			530	&	-4549	&	3.5	\\
			500	&	-4436	&	3.5	\\
			470	&	-4337	&	3.5	\\
			440	&	-4241	&	3.5	\\
			410	&	-4182	&	3.5	\\
			380	&	-4090	&	3.5	\\
			350	&	-4020	&	3.5	\\
			\hline
		\end{tabular}
	\end{center}
\end{table}

\begin{table}[htb!]
	\caption{State points simulated for 2,2,4-trimethylpentane with the TraPPE force field.}
	\begin{center}
		\begin{tabular}{|c|c|c|}
			\hline
			$T$ (K) & $\mu$ (K) & $L$ (nm) \\ \hline
			480	&	-4600	&	4.0	\\
			530	&	-4600	&	4.0	\\
			560	&	-4600	&	4.0	\\
			530	&	-4450	&	4.0	\\
			500	&	-4330	&	4.0	\\
			470	&	-4210	&	4.0	\\
			440	&	-4090	&	4.0	\\
			410	&	-3960	&	4.0	\\
			380	&	-3840	&	4.0	\\
			\hline
		\end{tabular}
	\end{center}
\end{table}

\begin{table}[htb!]
	\caption{State points simulated for cyclohexane with the TraPPE force field.}
	\begin{center}
		\begin{tabular}{|c|c|c|}
			\hline
			$T$ (K) & $\mu$ (K) & $L$ (nm) \\ \hline
			450	&	-4350	&	3.0	\\
			500	&	-4350	&	3.0	\\
			550	&	-4350	&	3.0	\\
			500	&	-4120	&	3.0	\\
			460	&	-3977	&	3.0	\\
			410	&	-3790	&	3.0	\\
			350	&	-3562	&	3.0	\\
			\hline
		\end{tabular}
	\end{center}
\end{table}

\begin{table}[htb!]
	\caption{State points simulated for cyclohexane with the $\lambda^{(1)}_{\rm CH_2} = 14$ force field.}
	\begin{center}
		\begin{tabular}{|c|c|c|}
			\hline
			$T$ (K) & $\mu$ (K) & $L$ (nm) \\ \hline
			450	&	-4389	&	3.0	\\
			500	&	-4389	&	3.0	\\
			550	&	-4389	&	3.0	\\
			500	&	-4164	&	3.0	\\
			460	&	-4033	&	3.0	\\
			410	&	-3891	&	3.0	\\
			360	&	-3780	&	3.0	\\
			\hline
		\end{tabular}
	\end{center}
\end{table}

\begin{table}[htb!]
	\caption{State points simulated for cyclohexane with the $\lambda^{(1)}_{\rm CH_2} = 16$ force field.}
	\begin{center}
		\begin{tabular}{|c|c|c|}
			\hline
			$T$ (K) & $\mu$ (K) & $L$ (nm) \\ \hline
			450	&	-4367	&	3.0	\\
			500	&	-4367	&	3.0	\\
			550	&	-4367	&	3.0	\\
			500	&	-4149	&	3.0	\\
			460	&	-4024	&	3.0	\\
			410	&	-3893	&	3.0	\\
			360	&	-3792	&	3.0	\\
			\hline
		\end{tabular}
	\end{center}
\end{table}

\begin{table}[htb!]
	\caption{State points simulated for cyclohexane with the $\lambda^{(1)}_{\rm CH_2} = 18$ force field.}
	\begin{center}
		\begin{tabular}{|c|c|c|}
			\hline
			$T$ (K) & $\mu$ (K) & $L$ (nm) \\ \hline
			450	&	-4370	&	3.0	\\
			500	&	-4370	&	3.0	\\
			550	&	-4370	&	3.0	\\
			500	&	-4158	&	3.0	\\
			460	&	-4037	&	3.0	\\
			410	&	-3912	&	3.0	\\
			360	&	-3825	&	3.0	\\
			\hline
		\end{tabular}
	\end{center}
\end{table}

\begin{table}[htb!]
	\caption{State points simulated for cyclohexane with the $\lambda^{(1)}_{\rm CH_2} = 20$ force field.}
	\begin{center}
		\begin{tabular}{|c|c|c|}
			\hline
			$T$ (K) & $\mu$ (K) & $L$ (nm) \\ \hline
			450	&	-4386	&	3.0	\\
			500	&	-4386	&	3.0	\\
			550	&	-4386	&	3.0	\\
			500	&	-4178	&	3.0	\\
			460	&	-4062	&	3.0	\\
			410	&	-3946	&	3.0	\\
			360	&	-3866	&	3.0	\\
			\hline
		\end{tabular}
	\end{center}
\end{table}

\begin{table}[htb!]
	\caption{State points simulated for 2-methylpropane with the MiPPE-gen force field.}
	\begin{center}
		\begin{tabular}{|c|c|c|}
			\hline
			$T$ (K) & $\mu$ (K) & $L$ (nm) \\ \hline
            350	&	-3150	&	3.0	\\
            380	&	-3150	&	3.0	\\
            410	&	-3145	&	3.0	\\
            380	&	-3010	&	3.0	\\
            350	&	-2910	&	3.0	\\
            320	&	-2830	&	3.0	\\
            290	&	-2760	&	3.0	\\
            260	&	-2700	&	3.0	\\
            230	&	-2670	&	3.0	\\
            200	&	-2640	&	3.0	\\
            \hline
		\end{tabular}
	\end{center}
\end{table}

\begin{table}[htb!]
	\caption{State points simulated for 2,2-dimethylpropane with the MiPPE-gen force field.}
	\begin{center}
		\begin{tabular}{|c|c|c|}
			\hline
			$T$ (K) & $\mu$ (K) & $L$ (nm) \\ \hline
			368	&	-3344	&	3.0	\\
			398	&	-3344	&	3.0	\\
			430	&	-3400	&	3.0	\\
			398	&	-3216	&	3.0	\\
			372	&	-3124	&	3.0	\\
			346	&	-3032	&	3.0	\\
			326	&	-2961	&	3.0	\\
			299	&	-2865	&	3.0	\\
			270	&	-2759	&	3.0	\\
			\hline
		\end{tabular}
	\end{center}
\end{table}

\begin{table}[htb!]
	\caption{State points simulated for 2,2-dimethylbutane with the MiPPE-gen force field.}
	\begin{center}
		\begin{tabular}{|c|c|c|}
			\hline
			$T$ (K) & $\mu$ (K) & $L$ (nm) \\ \hline
			415	&	-3873	&	3.5	\\
			445	&	-3873	&	3.5	\\
			480	&	-3895	&	3.5	\\
			450	&	-3756	&	3.5	\\
			420	&	-3654	&	3.5	\\
			400	&	-3588	&	3.5	\\
			380	&	-3521	&	3.5	\\
			360	&	-3454	&	3.5	\\
			340	&	-3384	&	3.5	\\
			310	&	-3380	&	3.5	\\
			\hline
		\end{tabular}
	\end{center}
\end{table}

\begin{table}[htb!]
	\caption{State points simulated for 2,3-dimethylbutane with the MiPPE-gen force field.}
	\begin{center}
		\begin{tabular}{|c|c|c|}
			\hline
			$T$ (K) & $\mu$ (K) & $L$ (nm) \\ \hline
			440	&	-4010	&	3.0	\\
			470	&	-4010	&	3.0	\\
			500	&	-4009	&	3.0	\\
			470	&	-3860	&	3.0	\\
			440	&	-3760	&	3.0	\\
			410	&	-3670	&	3.0	\\
			380	&	-3600	&	3.0	\\
			350	&	-3530	&	3.0	\\
			320	&	-3480	&	3.0	\\
			\hline
		\end{tabular}
	\end{center}
\end{table}

\begin{table}[htb!]
	\caption{State points simulated for 2,3,4-trimethylpentane with the MiPPE-gen force field.}
	\begin{center}
		\begin{tabular}{|c|c|c|}
			\hline
			$T$ (K) & $\mu$ (K) & $L$ (nm) \\ \hline
			480	&	-4720	&	3.5	\\
			520	&	-4720	&	3.5	\\
			565	&	-4713	&	3.5	\\
			530	&	-4540	&	3.5	\\
			500	&	-4360	&	3.5	\\
			470	&	-4355	&	3.5	\\
			440	&	-4275	&	3.5	\\
			410	&	-4205	&	3.5	\\
			380	&	-4165	&	3.5	\\
			350	&	-4115	&	3.5	\\
			\hline
		\end{tabular}
	\end{center}
\end{table}

\begin{table}[htb!]
	\caption{State points simulated for 2,2,4-trimethylpentane with the MiPPE-gen force field.}
	\begin{center}
		\begin{tabular}{|c|c|c|}
			\hline
			$T$ (K) & $\mu$ (K) & $L$ (nm) \\ \hline
			470	&	-4570	&	4.0	\\
			520	&	-4570	&	4.0	\\
			550	&	-4570	&	4.0	\\
			520	&	-4420	&	4.0	\\
			490	&	-4300	&	4.0	\\
			460	&	-4170	&	4.0	\\
			430	&	-4050	&	4.0	\\
			400	&	-3920	&	4.0	\\
			370	&	-3790	&	4.0	\\
			\hline
		\end{tabular}
	\end{center}
\end{table}

\section{Tabulated MBAR results} \label{Tabulated MBAR results}

\begin{table}[htb!]
	\caption{GCMC-MBAR results for 2-methylpropane .}
	\begin{center}
		\begin{tabular}{|c|c|c|c|c|c|}
			\hline
			$T_{\rm sat}$ (K) & $\rho_{\rm liq}^{\rm sat}$ (kg/m$^3$) & $\rho_{\rm vap}^{\rm sat}$ (kg/m$^3$) & $P_{\rm vap}^{\rm sat}$ (MPa) & $\Delta H_{\rm v}$ (kJ/mol) & $Z_{\rm vap}^{\rm sat}$ \\ \hline
			\hline
		\end{tabular}
	\end{center}
\end{table}

\section{Optimal $\psi$ values}



\end{document}